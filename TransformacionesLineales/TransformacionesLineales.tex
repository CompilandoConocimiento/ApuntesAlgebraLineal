% ****************************************************************************************
% ******************        TRANSFORMACIONES LINEALES     ********************************
% ****************************************************************************************


% =======================================================
% =======   ALL COMMANDS AND RULES FOR DOC   ============
% =======================================================
\documentclass[12pt]{report}                                %Type of docuemtn and size of font
\usepackage[margin=1.2in]{geometry}                         %Margins

\usepackage[spanish]{babel}                                 %Please use spanish
\usepackage[utf8]{inputenc}                                 %Please use spanish 
\usepackage[T1]{fontenc}                                    %Please use spanish

\usepackage{amsthm, amssymb, amsfonts, mathrsfs}            %Make math beautiful
\usepackage[fleqn]{amsmath}                                 %Please make equations left
\decimalpoint                                               %Make math beautiful
\setlength{\parindent}{0pt}                                 %Eliminate ugly indentation

\usepackage{graphicx}                                       %Allow to create graphics
\usepackage{wrapfig}                                        %Allow to create images
\graphicspath{ {Graphics/} }                                %Where are the images :D
\usepackage{listings}                                       %We will be using code here
\usepackage[inline]{enumitem}                               %We will need to enumarate

\usepackage{fancyhdr}                                       %Lets make awesome headers/footers
\renewcommand{\footrulewidth}{0.5pt}                        %We will need this!
\setlength{\headheight}{16pt}                               %We will need this!
\setlength{\parskip}{0.5em}                                 %We will need this!
\pagestyle{fancy}                                           %Lets make awesome headers/footers
\lhead{\footnotesize{\leftmark}}                            %Headers!
\rhead{\footnotesize{\rightmark}}                           %Headers!
\lfoot{Compilando Conocimiento}                             %Footers!
\rfoot{Oscar Rosas}                                         %Footers!

\author{Oscar Andrés Rosas}                                 %Who I am




% =====================================================
% ============        COVER PAGE       ================
% =====================================================
\begin{document}
\begin{titlepage}

    \center
    % ============ UNIVERSITY NAME AND DATA =========
    \textbf{\textsc{\Large Proyecto Compilando Conocimiento}}\\[1.0cm] 
    \textsc{\Large Algebra Lineal}\\[1.0cm] 

    % ============ NAME OF THE DOCUMENT  ============
    \rule{\linewidth}{0.5mm} \\[1.0cm]
        { \huge \bfseries Transformaciones Lineales}\\[1.0cm] 
    \rule{\linewidth}{0.5mm} \\[2.0cm]
    
    % ====== SEMI TITLE ==========
    {\LARGE Transformaciones Lineales}\\[7cm] 
    
    % ============  MY INFORMATION  =================
    \begin{center} \large
    \textbf{\textsc{Autor:}}\\
    Rosas Hernandez Oscar Andres
    \end{center}

    \vfill

\end{titlepage}


% =====================================================
% ========                INDICE              =========
% =====================================================
\tableofcontents{}
\clearpage

% ======================================================================================
% =============================    TRANSFORMACIONES LINEALES    ========================
% ======================================================================================
\chapter{Transformaciones Lineales}
    \clearpage

    % =====================================================
    % ============           DEFINICION            ========
    % =====================================================
    \section{Definición}
        Sea $V$ y $W$ dos espacios vectoriales sobre un \textbf{mismo} campo $K$. Una
        transformación lineal de $V \to W$ es una función que cumpla con esto:

        $\mathscr{T}: V \to W $ tal que $\forall v_1, v_2, v_3 \in V $ y $\forall \alpha \in K$
        tenemos que se cumple que:

        \begin{itemize}
            \item $\mathscr{T} (v_1 + v_2) = \mathscr{T}(v_1) + \mathscr{T}(v_2)$
            \item $\mathscr{T} (\alpha v_1) = \alpha \mathscr{T}(v_1)$ 
        \end{itemize}

        \subsubsection{Combinación Lineal}
        Podemos tambien tener que como consecuencia de lo que tenemos arriba que podemos
        encontrar que $\mathscr{T}$ es una transformación lineal si y solo si se cumple que:

        $\forall v_1, v_2 \in V$ y $\forall \alpha, \beta \in K$ se cumple que:
        \begin{equation}
            \mathscr{T}(\alpha v_1 + \beta v_2) = \alpha \mathscr{T}(v_1) + \beta \mathscr{T}(v_2)
        \end{equation}

        \subsubsection{Saber si algo es una $\mathscr{T}$}
        Así que para probar que una $\mathscr{T}$ es o no transformación lineal basta con verificar
        que se cumplan las 2 propiedades originales. 

        % ========================
        % =====   EJEMPLO   ======
        % ========================
        \clearpage
        \subsubsection{Ejemplos}
            Sea $\mathbb{R}^3 \to \mathbb{R}^2$ tal que:
            \begin{equation*}
                \mathscr{T}\left( \begin{matrix} x\\y\\z \end{matrix} \right) =
                \left( \begin{matrix} x-z\\y+z \end{matrix} \right)
            \end{equation*}

            Probemos la primera propiedad como:
            \begin{equation*}
            \begin{split}
                \mathscr{T} (v_1 + v_2) & =
                \mathscr{T} \left( \begin{matrix} x_1\\y_1\\z_1 \end{matrix} + \begin{matrix} x_2\\y_2\\z_2 \end{matrix} \right)
                \\
                & = \mathscr{T} \left( \begin{matrix} x_1+x_2\\y_1+y_2\\z_1+z_2 \end{matrix} \right)
                  = \begin{matrix} (x_1+x_2)-(z_1+z_2) \\ (y_1+y_2)+(z_1+z_2) \end{matrix}
                  = \begin{matrix}(x_1-z_1)\\(y_1+z_1)\end{matrix} + \begin{matrix}(x_2-z_2)\\(y_2+z_2)\end{matrix} 
                  = 
                  \mathscr{T} \left(\begin{matrix} x_1\\y_1\\z_1 \end{matrix} \right)
                  +
                  \mathscr{T} \left( \begin{matrix} x_2\\y_2\\z_2 \end{matrix} \right)            
                \\
                & = \mathscr{T}(v_1) + \mathscr{T}(v_2)
            \end{split}
            \end{equation*}

            Probemos la segunda propiedad:
            \begin{equation*}
            \begin{split}
                \mathscr{T} (\alpha v_1) & =
                \mathscr{T} \left( \alpha \cdot \begin{matrix} x\\y\\z \end{matrix} \right) \\
                & = \mathscr{T} \left( \begin{matrix} \alpha x\\ \alpha y\\ \alpha z\end{matrix} \right) =
                \begin{matrix} \alpha x - \alpha z\\ \alpha y + \alpha z\end{matrix}  =
                \alpha \cdot \begin{matrix}x-z\\y+z\end{matrix} =
                \alpha \mathscr{T} \left( \begin{matrix}x\\y\\z\end{matrix} \right) \\
                & = \alpha \mathscr{T}(v_1)
            \end{split}
            \end{equation*}

        Por lo tanto las 2 propiedades se cumplen así que si que es una transformación lineal.


    % =====================================================
    % ============       PROPIEDADES            ===========
    % =====================================================
    \clearpage
    \section{Propiedades}

        \subsubsection{El $0_v$ se preserva}
            Una Transformación Lineal debe llevar al $0_v$ de $V$ al $0_v$ de $W$

            Su demostración es muy sencilla, pues
            $\mathscr{T}(0_v)=\mathscr{T}(v_v-v_v)=\mathscr{T}(v_v)-\mathscr{T}(v_v)=0_w$

        \subsubsection{Operador Lineal}
            Decimos que $\mathscr{T}$ (alguna transformación lineal) es un
            operador lineal en V si y solo si su dominio y su contradominio son el mismo.


    % =====================================================
    % ============       KERNEL E IMAGEN        ===========
    % =====================================================
    \clearpage
    \section{Kernel e Imágen}

        % ===================================
        % ===========   KERNEL    ===========
        % ===================================
        \subsection{Kernel}
        \subsubsection{Definición}
        \textbf{El Kernel} de una Transformación Lineal o \textbf{Núcleo} es el conjunto 
        de todos los vectores originales (osea $v \in V$) tales que al momento de
        aplicarles la transformación estos son llevados al origen (osea $0_w$)

        O dicho con el bello lenguaje de matemáticas:
        \begin{equation}
            Kernel(\mathscr{T}) = \{v \in V |\quad \mathscr{T}(v) = 0_w\}
        \end{equation}

        Recuerda que un Kernel siempre siempre sera un Subespacio Vectorial y solemos
        llamar a su dimensión la 'Nulidad'.

        Podemos decir que el Kernel es el espacio solución del Sistema Homogeneo.
        \begin{equation*}
            \{x \in K^m |\quad Ax = 0_{m \times 1} \}
        \end{equation*}

        % ========================
        % =====   EJEMPLO    =====
        % ========================
            \clearpage
            \subsubsection{Ejemplo}
            Encuentra el Kernel de la siguiente Transformación Lineal:
            $\mathscr{T} : \mathbb{R}^3 \to \mathbb{R}_2[x]$ tal que: 
            $\mathscr{T}(a,b,c) = (a+b) + (a-c)x + (2a+b-c)x^2$

            Lo que nos estan pidiendo es:
            \begin{equation*}
                Kernel(\mathscr{T}) = 
                \{(a,b,c)\in \mathbb{R}^3 |\quad \mathscr{T}(a,b,c) = 0+0x+0x^2\}
            \end{equation*}

            Veamos que para hacerlo solo basta con que cumplan que:
            \begin{equation*}
            \begin{split}
                a + b       & = 0 \\
                a - c       & = 0 \\
                2a + b + c  & = 0 \\
            \end{split}
            \end{equation*}

            Podemos hacer Gauss - Jordan:
            \begin{equation*}
                \begin{pmatrix} 1&1&0 \\ 1&0&-1 \\ 2&1&-1 \\\end{pmatrix} \to
                \begin{pmatrix} 1&1&0 \\ 1&0&-1 \\ 1&0&-1 \\\end{pmatrix} \to
                \begin{pmatrix} 1&1&0 \\ 1&0&-1 \\ 0&0&0  \\\end{pmatrix}
            \end{equation*}

            Por lo tanto podemos ver que:
            \begin{equation*}
            \begin{split}
                a + b = 0 &\to a = -b  \\
                a - c = 0 &\to a = c   \\
             \end{split}
            \end{equation*}

            Por lo tanto podemos ver que:
            \begin{equation*}
                Kernel(\mathscr{T}) = \{(a,b,c)\in R^3 |\quad a = -b, a=c\}
            \end{equation*}

            Finalmente aplicamos la transformación con estas propiedades y tenemos que:
            \begin{equation*}
                Kernel(\mathscr{T}) = \{(a,-a,a)\in \mathbb{R}^3 |\quad a \in \mathbb{R}\}
            \end{equation*}

            Y si te das cuenta estas ya describiendo un espacio vectorial que esta definido como:
            \begin{equation*}
                Kernel(\mathscr{T}) = \{\alpha(1,-1,1) |\quad \alpha\in \mathbb{R}\}
            \end{equation*}

            Sera tal vez una linea, pero no deja de ser espacio vectorial, cuyo vector base es:
            \begin{equation*}
                Kernel(\mathscr{T}) = <(1,-1,1)>
            \end{equation*}




        % ===================================
        % ==========   IMAGEN    ============
        % ===================================
        \clearpage
        \subsection{Imágen}
        Tambien tenemos a la hermana perdida del Kernel, la llamamos la \textbf{Imágen},
        la cual la definimos así:

        \subsubsection{Definición}
        La imágen de una Transformación Lineal es el conjunto de todos los vectores
        nuevos (osea $w \in W$) que podemos 'crear' desde los vectores originales
        (osea $v \in V$) usando la Transformación Lineal.

        O dicho con el bello lenguaje de matemáticas:
        \begin{equation}
            Imagen(\mathscr{T}) = \{w \in W |\quad \exists v \in V ,\quad \mathscr{T}(v) = w\}
        \end{equation}

        Recuerda que una Imagen siempre siempre sera un Espacio Vectorial y solemos
        llamar a su dimensión 'Rango'.

        Podemos decir que el Imagen es el conjunto de terminos independientes para los cuales
        hay solución.
        \begin{equation*}
            \{b \in K^m |\quad \exists x \in K^m, Ax = b \}
        \end{equation*}


        % ========================
        % =====   EJEMPLO    =====
        % ========================
            \clearpage
            \subsubsection{Ejemplo}
            Encuentra la Imagen de la siguiente Transformación Lineal:
            $\mathscr{T} : \mathbb{R}^3 \to \mathbb{R}_2[x]$ tal que: 
            $\mathscr{T}(a,b,c) = (a+b) + (a-c)x + (2a+b-c)x^2$

            Lo que nos estan pidiendo es:
            \begin{equation*}
                Imagen(\mathscr{T}) = 
                \{a_0+a_1x+a_2x^2 \in R_2[x] |\quad \exists (a,b,c) \in R^3 ,\quad 
                \mathscr{T}(a,b,c) = a_0+a_1x+a_2x^2\}
            \end{equation*}

            Es decir, lo que se nos esta pidiendo es que:
            \begin{equation*}
            \begin{split}
                a + b           & = a_0 \\
                a - c           & = a_1 \\
                2a + b + c      & = a_2 \\
             \end{split}
            \end{equation*}

            Y pos preguntas para que valores de $a_0, a_1, a_2$ tiene solución el 
            sistema que planteamos allá arriba.

            Es decir lo que tenemos que hacer es ver las soluciones de
            este sistema de ecuaciones, podemos hacer Gauss - Jordan:
            \begin{equation*}
                \begin{pmatrix} 1&1&0 \\ 1&0&-1 \\ 2&1&-1 \\\end{pmatrix} 
                \begin{pmatrix} a_0 \\ a_1 \\ a_2 \\\end{pmatrix}
                \to_{Usando: Gauss-Jordan}
                \begin{pmatrix} 1&1&0 \\ 1&0&-1 \\ 0&0&0 \\\end{pmatrix}
                \begin{pmatrix} a_1 & \\ a_0-a_1 &\\ a_2-a_1-a_0 &\\\end{pmatrix}
            \end{equation*}

            Por lo tanto podemos ver que:
            \begin{equation*}
                a_2-a_1-a_0 = 0
                \quad \to \quad
                a_2 = a_1 + a_0
            \end{equation*}

            Y ya solo sustituyendo tenemos que:
            \begin{equation*}
            \begin{split}
                Imagen(\mathscr{T}) 
                & = \{a_0+a_1x+(a_0+a_1)x^2 \in R_2[x] |\quad a_2 = a_0 + a_1, |\quad a_0, a_1 \in \mathbb{R}\}  \\
                & = \{a_0(1+x^2) +a_1(x+x^2) \in R_2[x] |\quad a_0, a_1 \in \mathbb{R}\}
            \end{split}
            \end{equation*}

            Y si te das cuenta estas ya describiendo un espacio vectorial que esta definido como:
            \begin{equation*}
                Imagen(\mathscr{T}) = \{\alpha(1+x^2) + \beta(x+x^2) |\quad \alpha, \beta \in \mathbb{R}\}
            \end{equation*}

            Y cuyos vectores base son:
            \begin{equation*}
                Imagen(\mathscr{T}) = <(1+x^2) , (x+x^2)>
            \end{equation*}

        % ====================================
        % ============   PROPIEDADES    ======
        % ====================================
        \subsection{Propiedades de Ambas}
        Podemos hablar de que ambas paracen ser como hermanas perdidas,
        veamos que propiedades tenemos:

        \begin{itemize}
            \item Llamemos Rango a $Dim(Imagen(\mathscr{T}))$
            \item Llamemos Nulidad a $Dim(Kernel(\mathscr{T}))$
            \item Ambas \textbf{Son SubEspacios Vectoriales}.
            \item Estas deacuerdo que todos los vectores o bien son llevados al cero
            vector o no, así que tiene sentido hablar de que  \textbf{La Suma de la Nulidad
            con el Rango te da la dimensión de V}
        \end{itemize}

% =====================================================
% ============        BIBLIOGRAPHY   ==================
% =====================================================
\clearpage
\bibliographystyle{plain}
    \begin{thebibliography}{9}

    % ============ REFERENCE #1 ========
    \bibitem{Sitio1} 
        ProbRob
        \\\texttt{Youtube.com}


     

\end{thebibliography}



\end{document}