% ****************************************************************************************
% ************************      ESPACIOS VECTORIALES      ********************************
% ****************************************************************************************


% =======================================================
% =======   ALL COMMANDS AND RULES FOR DOC   ============
% =======================================================
\documentclass[12pt]{report}                                %Type of docuemtn and size of font
\usepackage[margin=1.2in]{geometry}                         %Margins

\usepackage[spanish]{babel}                                 %Please use spanish
\usepackage[utf8]{inputenc}                                 %Please use spanish 
\usepackage[T1]{fontenc}                                    %Please use spanish

\usepackage{amsthm, amssymb, amsfonts}                      %Make math beautiful
\usepackage[fleqn]{amsmath}                                 %Please make equations left
\decimalpoint                                               %Make math beautiful
\setlength{\parindent}{0pt}                                 %Eliminate ugly indentation

\usepackage{graphicx}                                       %Allow to create graphics
\usepackage{wrapfig}                                        %Allow to create images
\graphicspath{ {Graphics/} }                                %Where are the images :D
\usepackage{listings}                                       %We will be using code here
\usepackage[inline]{enumitem}                               %We will need to enumarate

\usepackage{fancyhdr}                                       %Lets make awesome headers/footers
\renewcommand{\footrulewidth}{0.5pt}                        %We will need this!
\setlength{\headheight}{16pt}                               %We will need this!
\setlength{\parskip}{0.5em}                                 %We will need this!
\pagestyle{fancy}                                           %Lets make awesome headers/footers
\lhead{\footnotesize{\leftmark}}                            %Headers!
\rhead{\footnotesize{\rightmark}}                           %Headers!
\lfoot{Compilando Conocimiento}                             %Footers!
\rfoot{Oscar Rosas}                                         %Footers!

\author{Oscar Andrés Rosas}                                 %Who I am




% =====================================================
% ============        COVER PAGE       ================
% =====================================================
\begin{document}
\begin{titlepage}

    \center
    % ============ UNIVERSITY NAME AND DATA =========
    \textbf{\textsc{\Large Proyecto Compilando Conocimiento}}\\[1.0cm] 
    \textsc{\Large Algebra Lineal}\\[1.0cm] 

    % ============ NAME OF THE DOCUMENT  ============
    \rule{\linewidth}{0.5mm} \\[1.0cm]
        { \huge \bfseries Espacios Vectoriales y Bases}\\[1.0cm] 
    \rule{\linewidth}{0.5mm} \\[2.0cm]
    
    % ====== SEMI TITLE ==========
    {\LARGE Espacios Vectoriales}\\[7cm] 
    
    % ============  MY INFORMATION  =================
    \begin{center} \large
    \textbf{\textsc{Autor:}}\\
    Rosas Hernandez Oscar Andres
    \end{center}

    \vfill

\end{titlepage}

% =====================================================
% ========                INDICE              =========
% =====================================================
\tableofcontents{}
\clearpage

% ======================================================================================
% =============================     ESPACIOS Y SUBESPACIOS    ==========================
% ======================================================================================
\chapter{Espacios y SubEspacios}
    \clearpage

    % =====================================================
    % ============    ESPACIOS VECTORIALES         ========
    % =====================================================
    \section{Espacios Vectoriales}
        Un espacio vectorial $V$ es un Conjunto de objetos
        llamados Vectores (Dahh!), junto con dos operaciones:

        \begin{itemize}
            \item \textbf{Suma de Vectores}: Recibe 2 vectores y regresa 1 vector
            \item \textbf{Producto Escalar}: Recibe 1 vector y 1 escalar y regresa 1 vector
        \end{itemize}

        Lo importantes es que estas operaciones, satisfascan los 10 Axiomas
        que se enumeran a continuacion:

        \begin{enumerate}
            \item \textbf{Es Cerrado en Suma}:\\
            Si $x \in V$ y $y \in V$, entonces $x+y \in V$

            \item \textbf{Es Asociativo}:\\
            Para todos $x,y,z \in V$, $(x+y) + z = x + (y+z)$

            \item \textbf{Existe en O Vector}:\\
            Existe un vector $0 \in V$ tal que todos $x \in V, x + 0 = 0 $

            \item \textbf{Inverso de un Vector}:\\
            Si $x \in V$, existe un vector $-x$ en V tal que $x+(-x) = 0$

            \item \textbf{Es Conmutativo}:\\
            Si $x,y \in V$ , entonces $x + y = y + x$

            \item \textbf{Multiplo de un Vector}:\\
            Si $x \in V$, y $\alpha \in K$, entonces $\alpha x \in V$

            \item \textbf{Existe un Uno}:\\
            Para todo vector $x \in V$, tenemos que $1x = x$  

            \item Si $x,y \in V$ y $\alpha \in K$, entonces $\alpha(x+y) = \alpha x + \alpha y$

            \item Si $x \in V$ y $\alpha \in K$, entonces $\alpha(\beta x) = \alpha \beta x$          

        \end{enumerate}


        % ==========================
        % ====   PROPIEDADES  ======
        % ==========================
        \subsection{Propiedades}
        Podemos ver algunas caracteristicas muy útiles de los Espacios vectoriales, sea $V$ 
        un Espacio vectorial, entonces:
        \begin{itemize}
            \item $\alpha 0 = 0 $, $\forall \alpha \in \mathbb{R}$\\
            \item $0x = 0$, $\forall x \in V$
            \item Si $\alpha x = 0$, entonces $\alpha = 0$ ó bien $x = 0$ ó ambos.
            \item $(-1)x = -x$ $\forall x \in V$
        \end{itemize}


    % =====================================================
    % ============    SUBESPACIOS VECTORIALES      ========
    % =====================================================
    \clearpage
    \section{SubEspacios Vectoriales}    
    
        Un Subconjunto no vacio $H$ de un Espacio vectorial V
        es un Subespacio de $V$ si se cumplen que:

        \begin{enumerate}
            \item \textbf{Cerradura de la Suma}
            Si $x \in H$ y $y \in H$, entonces $x + y \in H$

            \item \textbf{Cerradura de la Producto Escalar}
            Si $x \in H$, entonces $\alpha x \in H$, para
            todo escalar $\alpha$
        \end{enumerate}

        Otra forma de probar es checar la combinacion lineal $\alpha w_1 + \beta w_2 \in W$
        ya que se cumplen las dos condiciones.

        % ==========================
        % ====   PROPIEDADES  ======
        % ==========================
        \subsection{Propiedades}

        \begin{itemize}
            \item \{$0_v$\} es un Subespacio.

            \item $V$ es un Subespacio de $V$

            \item $W_1 inter W_2$ es un Subespacio de $V$

            \item $W_1 + W_2$ es un Subespacio de $V$
        \end{itemize}


    % =====================================================
    % ============    INDEPENDENCIA Y DEPENDENCIA     =====
    % =====================================================
    \clearpage
    \section{Independecia y Dependencia}   

        % ==========================
        % ====   INDEPENDECIA  =====
        % ==========================
        \subsection{Independencia Lineal}
        Sea A una matriz de $n \times n$. Entonces cada uno de los siguientes siete enunciados implica a los otros seis.
        \begin{itemize}
            \item A es Invertible.
            \item $Det(A) \neq 0$.
            \item La unica solución al Sistema Homogéneo $Ax=0$ es la solución $x=0$.
            \item El sistema $Ax=b$ posee una solución única para todo n-vector b.
            \item A es equivalente por filas a la Matriz Identidad.
            \item A puede ser escrita como el producto de matrices elementales.
            \item Las columnas y los renglones de A son Linealmente Independientes.
        \end{itemize}

        Podemos generalizar aún más esto de la siguiente manera como: 

        Sean A  = $\begin{bmatrix} F_{1} \\ F_{2} \\ F_{n} \end{bmatrix} $ 
                = $\begin{bmatrix} C_{1} &  C_{2} &  C_{n} \end{bmatrix} $
                pertenecen a $M_{m\times n} (K)$

        Es decir sea A un Vector de Vectores (estos últimos sean Vectores Fila
        o Columna, la verdad no importa), entonces los siguientes
        enunciados son equivalentes:

        \begin{itemize}
            \item A es Invertible
            \item $F_1, F_2, \cdots ,F_n$ generan a $K^n$
            \item $C_1, C_2, \cdots, C_n$ generan a $K^n$
            \item $F_1, F_2, \cdots, F_n$ son Linealmente Independientes en $K^n$
            \item $C_1, C_2, \cdots, C_n$ son Linealmente Independientes en $K^n$
            \item $B = (F_1, F_2, \cdots, F_n)$ es base de $K^n$
            \item B = ($C_1, C_2, \cdots, C_n)$ es base de $K^n$
        \end{itemize}
        

        % ==========================
        % ====   PROPIEDADES  ======
        % ==========================
        \subsection{Propiedades}
        \begin{itemize}
            \item Un conjunto de n vectores en $\mathbb{R}^m$ es siempre Linealmente
             \textbf{Dependiente} si $n > m$. (Si hay mas incognitas que ecuaciones).
        \end{itemize}

        % ==========================
        % ====    EJEMPLOS    ======
        % ==========================
        \subsection{Ejemplo}

            Tengamos el Sistema $\{3,2x,-x^2\}$ y veamos si es Linealmente Independiente:

            Sea $\alpha_1, \alpha_2, \alpha_3 \in \mathbb{R}$ tales que:
            \begin{equation*}
                \alpha_1(3) + \alpha_2(2x) + \alpha_3(-x^2) = 0
            \end{equation*}

            Entonces tenemos el sistema:
            \begin{equation*}
                \begin{bmatrix}
                    0 & 0 & -\alpha_3 & = 0\\
                    0 & 2\alpha_2 & 0 & = 0\\
                    3\alpha_1 & 0 & 0 & = 0\\
                \end{bmatrix}
            \end{equation*}

            Así que $\alpha_1=\alpha_2=\alpha_3 = 0$, por lo que $\{3,2x,-x^2\}$ son
            Linealmente Independientes.

        % ==========================
        % ====    EJEMPLOS    ======
        % ==========================
        \subsection{Ejemplo}

            Tengamos el Sistema $\{1+x,2+2x-3x^2,x^2\}$ y veamos si es Linealmente Independiente:

            Sea $\alpha_1, \alpha_2, \alpha_3 \in \mathbb{R}$ tales que:
            \begin{equation*}
                \alpha_1(1+x) + \alpha_2(2+2x) + \alpha_3(x^2) = 0
            \end{equation*}

            Entonces tenemos el sistema:
            \begin{equation*}
                \begin{bmatrix}
                    0 & -3\alpha_2 & \alpha_3 & = 0\\
                    \alpha_1 & 2\alpha_2 & 0 & = 0\\
                    \alpha_1 & 2\alpha_2 & 0 & = 0\\
                \end{bmatrix}
                \to
                \begin{bmatrix}
                    0 & -3 & 1 \\
                    1 & 2 & 0  \\
                    1 & 2 & 0  \\
                \end{bmatrix}
            \end{equation*}

            Donde este sistema tiene infinitas soluciones, por lo tanto $\{1+x,2+2x-3x^2,x^2\}$ son
            Linealmente Dependientes.
        

    % =====================================================
    % ============    GENERACION DE ESPACIO      ==========
    % ===================================================== 
    \clearpage
    \section{Generación de Espacios}
        Los vectores $v_1,v_2,\dots,v_n$ forman un Espacio Vectorial $V$, o se dice
        que  generan a $V$, si todo vector en $V$ puede expresarse como Combinacion lineal de ellos.


        Esto es, para todo $v \in V$, existen escalares $\{a_1,a_2,\dots,a_n\}$ tales que:

        \begin{equation}
            v = a_1v_1 + a_2v_2 + \cdots + a_nv_n
        \end{equation}


        % ==========================
        % ====   PROPIEDADES  ======
        % ==========================
        \subsection{Propiedades}
        \begin{itemize}
            \item Un Conjunto de n Vectores linealmente independientes en
            $\mathbb{R}^n$ genera a $\mathbb{R}^n$

            \item Sean $n+1$ Vectores: $\{ v_1, v_2, \cdots, v_n, v_{n+1} \}$, 
            de un espacio vectorial $V$. Si $\{ v_1, v_2, \cdots, v_n \}$ generan a $V$,
            entonces $\{ v_1, v_2, \cdots, v_n, v_{n+1} \}$ también generan a $V$.

            \item Si $\{ v_1, v_2, \cdots, v_n \}$ son linealmente independientes,
            entonces los $n-1$ vectores, $\{v_1, v_2, \cdots, v_{n-1}, v_n \}$ son
            Linealmente Independientes.

        \end{itemize}


        % ==========================
        % ====     EJEMPLO    ======
        % ==========================
            \subsubsection{Ejemplo}
            Determine si el siguiente conjunto de vectores $\{ 3, 2x, -x^2\}$ genera
            a $\mathbb{R}_2 [x]$, es decir que genera a todos los
            polinómios de máximo grado 2.

            Sea $ax^2 + bx +c \in \mathbb{R}_2 [x]$.

            Luego tenemos que
            $\alpha_1 (3) + \alpha_2 (2x) + \alpha_3 (-x^2) = ax^2 + bx +c$,
            entonces tenemos el Sistema que:

            \begin{equation*}
                \begin{bmatrix}
                    0 & 0 & -\alpha_3 & = a\\
                    0 & 2\alpha_2 & 0 & = b\\
                    3\alpha_1 & 0 & 0 & = c\\
                \end{bmatrix}
                \to
                \begin{bmatrix}
                    0 & 0 & -1 & | a\\
                    0 & 2 & 0 &  | b\\
                    3 & 0 & 0 &  | c\\
                \end{bmatrix}
            \end{equation*}

            Donde es obvio que su determinante no es 0, (es 6 :p), esto quiere decir que el
            sistema siempre tiene solución.

            Por lo tanto este conjunto si que genera a $\mathbb{R}_2 [x]$.

        % ==========================
        % ====     EJEMPLO    ======
        % ==========================
            \subsubsection{Ejemplo}
            Determine si el siguiente conjunto de vectores $\{ 1+x,2+2x-3x^2,x^2\}$ genera
            a $\mathbb{R}_2 [x]$, es decir que genera a todos los
            polinómios de máximo grado 2.

            Sea $ax^2 + bx +c \in \mathbb{R}_2 [x]$.

            Luego tenemos que
            $\alpha_1 (1+x) + \alpha_2 (2+2x-3x^2) + \alpha_3 (x^2) = ax^2 + bx +c$,
            entonces tenemos el Sistema que:
            
            \begin{equation*}
                \begin{bmatrix}
                    0 & -3\alpha_2 & \alpha_3 & = a\\
                    \alpha_1 & 2\alpha_2 & 0 & = b\\
                    \alpha_1 & 2\alpha_2 & 0 & = c\\
                \end{bmatrix}
                \to
                \begin{bmatrix}
                    0 & -3 & 1 & | a\\
                    1 & 2 & 0 &  | b\\
                    1 & 2 & 0 &  | c\\
                \end{bmatrix}
            \end{equation*}

            Donde es obvio que su determinante es 0, esto quiere decir que el
            sistema NO siempre tiene solución.

            Por lo tanto este conjunto NO genera a $\mathbb{R}_2 [x]$.


    % =====================================================
    % ============    BASES DE DIMENSIONES       ==========
    % ===================================================== 
    \clearpage
    \section{Bases y Dimensión}

        % ==========================
        % ====      BASES     ======
        % ==========================
        \subsection{Bases}
        Un conjunto de vectores forman una Base para V si:
        \begin{itemize}
            \item Dicho conjunto es Linealmente Independiente.

            \item Dicho conjunto es genera a V.
        \end{itemize}

        \subsubsection{Propiedades}
        Un Conjunto de vectores forman una Base para V si Todo conjunto de n vectores
        linealmente independientes en $\mathbb{R}^n$ es un Base en $\mathbb{R}^n$

        Si $\{ v_1, v_2,\cdots, v_n\}$ es una Base de $V$ y si $v \in V$, entonces
        existe un conjunto Único de escalares $c_1, c_2, \cdots, c_n$ tales que:

        \begin{equation}
            v = c_1v_1 + c_2v_2 + \cdots + c_nv_n
        \end{equation}

        Si $\{u_1, u_2, \cdots, u_n\}$ y $\{v_1, v_2, \cdots, v_n\}$ son bases del
        Espacio vectorial V, entonces $m = n$, cualesquiera dos bases en un espacio
        vectorial V poseen el mismo número de vectores.

        % ==========================
        % ====   PROPIEDADES  ======
        % ==========================
        \subsubsection{Observaciones}
        Sea $n = dim(V)$. Entonces los siguientes enunciados son equivalentes:
        \begin{itemize}
            \item $v_1, v_2, \cdots ,v_n$ Generan a V
            \item $v_1, v_2, \cdots ,v_n$ Son Linealmente Independientes
            \item $B = \{ v_1, v_2, \cdots ,v_n \}$ son una Base de V
        \end{itemize}

        Sea $n = dim(V)$. Entonces los siguientes enunciados son verdaderos:
        \begin{itemize}
            \item Todo conjunto $\{ v_1, v_2, \cdots ,v_n$ con $n < m$ que generan a V
            se puede reducir a una base de V.

            \item Todo conjunto$\{ v_1, v_2, \cdots ,v_n$ con $m < n$ que sea Linealmente
            Independiente se puede completar a una base
        \end{itemize}

        % ==========================
        % ====      EJEMPLO     ====
        % ==========================
        \subsubsection{Ejemplo}

            Para que valores de $a$ los siguentes vectores forman una base de $\mathbb{R}^3$:
            \begin{equation*}
                \left\{
                    \begin{bmatrix} 1 \\ 1 \\ a \end{bmatrix},
                    \begin{bmatrix} 1 \\ a \\ 1 \end{bmatrix},
                    \begin{bmatrix} a \\ 1 \\ 1 \end{bmatrix} 
                \right\}
            \end{equation*}


            Para poder resolver esto basta con seguir las propiedades:

            Forman una base en $\mathbb{R}^3$ ssi la siguiente Matriz es Invertible:
            \begin{equation*}
                A = \begin{bmatrix} 1 & 1 & a \\ 1 & a & 1 \\ a & 1 & 1 \end{bmatrix}
            \end{equation*}

            Ssi, la Determinante de A sea difetente de 0.

            \begin{equation*}
            \begin{split}
            det(A) & = 3a-a^3-2 = -(a^3+3a-2) \\
                   & = - (a-1)(a^2+a-2) = -(a-1)(a-1)(a+2) = -(a-1)^2(a+2)
            \end{split}
            \end{equation*}

            Con esto logramos ver que las raices de dicha expresión es 1 y -2.
            Pero sabemos que dicho Determinante no puede ser 0, por lo tanto tenemos que:

            \begin{equation*}
                A = \begin{bmatrix} 1 & 1 & a \\ 1 & a & 1 \\ a & 1 & 1 \end{bmatrix}
            \end{equation*}

            Genera a $\mathbb{R}^3$ siempre y cuando $a \neq 1$ ó $a \neq -2$ 


        % ==========================
        % ====    DIMENSION     ====
        % ==========================
        \clearpage
        \subsection{Dimensión}
        Si el Espacio vectorial $V$ posee una base finita, la dimension de V es el 
        número de vectores en la base, y V se llama Espacio vectorial de dimension finita.

        Cualesquiera n vectores linealmente independientes en un Espacio
        vectorial V de dimensión n, constituyen una base.

        \begin{itemize}
            \item De otra manera, $V$ se denomina Espacio vectorial de dimension infinita.
            Si $V=\{0\}$, entones V se dice que es de Dimensión 0.

            \item Supongase que $dim(V) = n$. Si $\{u_1, u_2, \cdots, u_n\}$ es un Conjunto de m vectores
            Linealmente Independientes en $V$, entonces $m \leq n$.

            \item Sea $H$ un Subespacio vectorial de V. Entonces H es de dimensión finita
            y $dim(H) \leq dim(V)$.

        \end{itemize}


        % ==========================
        % ====   PROPIEDADES  ======
        % ==========================
        \subsubsection{Dimensiones Comunes}
        Sea $ B = (V_1, V_2, \cdots, V_n)$, osea, sea B un Conjuntos de Vectores:

        \begin{itemize}
            \item $dim(K^n) = n$
            \item $dim(M_{m \times n}(K)) = mn$
            \item $dim(K_{n}{[X]}) = n + 1$
        \end{itemize}


% ======================================================================================
% =============================     SISTEMAS DE COORDENADAS   ==========================
% ======================================================================================
\chapter{Sistemas de Coordenadas}
    \clearpage

    % =====================================================
    % ============    SISTEMA DE COORDENADAS       ========
    % =====================================================
    \section{Sistemas de Coordenadas}

        Sea una $B = \{ v_!, v_2, \cdots, v_n\}$ una base de un Espacio Vectorial V.
        Sean $v \in V$.
        Sean $\alpha_1, \alpha_2, \cdots, \alpha_n \in K$ tales que:

        \begin{equation}
            v = \sum_{i=1}^{n} \alpha_i v_i
        \end{equation}

        Si esto pasa, entonces podemos decir que  $\alpha_1, \alpha_2, \cdots, \alpha_n$ son únicos.

        \subsection{Demostración}
        \begin{itemize}
            \item Propon otros escalares que cumplen con generar al mismo vector
            \item Pero como son base, son linealmente independientes, por lo tanto ambos escalares deben ser iguales
        \end{itemize}


        \subsection{¿Qué es un Sistema de Coordenadas?}
        Sea una $B = \{ v_!, v_2, \cdots, v_n\}$ una base de un Espacio Vectorial V.
        Sean $v \in V$.
        Sean $\alpha_1, \alpha_2, \cdots, \alpha_n \in K$ tales que $v = \sum_{i=1}^{n} \alpha_i v_i$.

        Entonces podemos definir las coordenadas de nuestro pequeño e inocente $v$ en la Base B como:

        \begin{equation}
            [v]_B = 
            \begin{pmatrix} 
                \alpha_1    \\
                \alpha_2    \\
                \cdots      \\
                \alpha_n 
            \end{pmatrix}
            \in K^n
        \end{equation}


        \clearpage

        % ==========================
        % ====   EJEMPLOS  =========
        % ==========================
        \subsubsection{Ejemplo:}
        Considerere a $B = \{ (1+x), (1+x^2) , (x + x^2) \}$ como una base de un Polinomio de $\mathbb{R}_2[x]$.

        Sea $p(x) = 1+8x+3x^2$.

        Luego podemos ver que podemos escribirlo como:
        $3(1+x) + (-2)(1+x^2) + (5)(x+x^2)$

        Es decir, podemos escribirlo como:
        $[p(x)]_B =  \begin{pmatrix} 3\\-2\\5\end{pmatrix}$
         

        Para encontrarlos lo que tuvimos que hacer fue plantear el siguiente sistema de ecuaciones:
        \begin{equation*}
            \alpha_1 + \alpha_2 = 1  \\
            \alpha_1 + \alpha_3 = 8  \\
            \alpha_2 + \alpha_3 = 3  \\
        \end{equation*}


        % ==========================
        % ====   PROPIEDADES  ======
        % ==========================
        \subsection{Propiedades}
        Podemos ver entonces que estas coordenadas se comporta de manera muy muy bonita:

        \begin{itemize}
            \item $[v_1 + v_2]_B = [v_1]_B + [v_2]_B$
            \item $[\alpha v_1]_B = \alpha [v_1]_B$
        \end{itemize}



    % =====================================================
    % ============    SISTEMA DE COORDENADAS       ========
    % =====================================================
    \clearpage
    \section{Cambio de Coordenadas}

        Sea $B_1 = \{v_1,v_2, \cdots, v_n\}$ y sean  $B_2 = \{u_1,u_2, \cdots, u_n\}$.

        Podemos cambiar de base usando la siguiente Matriz:

        \begin{equation*}
            C_{B_1 \to B_2}= C_{B_1}^{B_2} = C_{\frac{B_2}{B_1}} = \left( [v_1]_{B_2} + [v_2]_{B_2} + \cdots + [v_n]_{B_2}   \right)
        \end{equation*}

        Podemos ver entonces que:
        \begin{equation*}
            [v]_{B_2} = C_{B_1 \to B_2} [v]_{B_1}
        \end{equation*}

        Para encontrarla lo mas útil de la vida será:
        \begin{equation}
            \begin{pmatrix}  Base 2  \vert Base 1 \end{pmatrix} \to_{Gauss-Jordan}
            \begin{pmatrix}  I_n  \vert C_{B_1 \to B_2} \end{pmatrix}
        \end{equation}


        Podemos saber algunas cosas super interesantes como:
        \begin{itemize}
            \item Si tenemos ya una matriz de cambio de base podemos obtener el otro cambio simplemente sacando la inversa a la matriz:
            $ C^{-1}_{\frac{B_2}{B_1}} = C_{\frac{B_1}{B_2}} $
            
            \item Podemos ver que existe algo que me tienta a llamar 'inversos' o que 'se cancela': 
            $ C_{\frac{B_3}{B_2}} C_{\frac{B_2}{B_1}} = C_{\frac{B_3}{B_1}} $

        \end{itemize}

        % ==========================
        % ====   EJEMPLOS  =========
        % ==========================
        \subsubsection{Ejemplo 1}

        Por ejemplo, sea:
        \begin{equation*}
            B_1 = \left\{
                \begin{pmatrix} 1 \\ 1 \\0 \end{pmatrix}, \begin{pmatrix} 1 \\ 0 \\1 \end{pmatrix},
                \begin{pmatrix} 0 \\ 1 \\1 \end{pmatrix}
            \right\}
        \end{equation*}

        \begin{equation*}
            B_2 = \left\{
                \begin{pmatrix} 1 \\ 0 \\0 \end{pmatrix}, \begin{pmatrix} 1 \\ 1 \\0 \end{pmatrix},
                \begin{pmatrix} 1 \\ 1 \\1 \end{pmatrix}
            \right\}
        \end{equation*}

        Entonces, podemos encontrar la Matriz de Cambio de Coordenadas de $B_2$ al $B_1$ como:

        \begin{equation*}
            \left(
            \begin{matrix}
                1 & 1 & 1 \\
                0 & 1 & 1 \\
                0 & 0 & 1 \\
            \end{matrix}
            \Big|
            \begin{matrix}
                1 & 1 & 0 \\
                0 & 0 & 1 \\
                0 & 1 & 1 \\
            \end{matrix}
            \right)
            = \left(
            I_3
            \big|
            \begin{matrix}
                0 & 1 & -1 \\
                1 & -1 & 0 \\
                0 & 1 & 1 \\
            \end{matrix}
            \right)
        \end{equation*}

        Entonces ya al final podemos decir que:
        \begin{equation*}
            C_{B_2 \to B_1} =
            \begin{pmatrix}
                0 & 1 & -1 \\
                1 & -1 & 0 \\
                0 & 1 & 1 \\
            \end{pmatrix}
        \end{equation*}



        % ==========================
        % ====   EJEMPLOS  =========
        % ==========================
        \subsubsection{Ejemplo 2}

        Si queremos encontrar la matriz de cambio de base entre:
        \begin{itemize}
            \item $B_1 = < (1+x), (1+x^2), (x+x^2) >$ 
            \item $B_2 = < (1), (1+x), (1+x+x^2) >$ 
        \end{itemize}

        Entonces podemos tener esta Matriz de Cambio de Base:

        \begin{equation*}
            \begin{pmatrix}
                1 & 1 & 1 | & 1 & 1 & 0 \\
                0 & 1 & 1 | & 1 & 0 & 1 \\
                0 & 0 & 1 | & 0 & 1 & 1 \\
            \end{pmatrix}
            \to_{Gauss-Jordan}
            \begin{pmatrix}
                1 & 0 & 0 | & 0 & 1 & -1 \\
                0 & 1 & 0 | & 1 & -1 & 0 \\
                0 & 0 & 1 | & 0 & 1 & 1 \\
            \end{pmatrix}
        \end{equation*}

        Por lo tanto podemos concordar que:
        \begin{equation*}
            C_{\frac{B_2}{B_1}}
            \begin{pmatrix}
                0 & 1 & -1 \\
                1 & -1 & 0 \\
                0 & 1 & 1  \\
            \end{pmatrix}
        \end{equation*}



        % ==========================
        % ====   EJEMPLOS  =========
        % ==========================
        \subsubsection{Ejemplo 3}

        Por ejemplo, sea:
        \begin{equation*}
            B_1 = \left\{
                \begin{pmatrix}\bar{1}&\bar{2}\\\bar{4}&\bar{0}\end{pmatrix},
                \begin{pmatrix}\bar{0}&\bar{1}\\\bar{0}&\bar{1}\end{pmatrix},
                \begin{pmatrix}\bar{1}&\bar{1}\\\bar{3}&\bar{0}\end{pmatrix},
                \begin{pmatrix}\bar{1}&\bar{0}\\\bar{0}&\bar{0}\end{pmatrix},
            \right\}
        \end{equation*}

        Y la canonica:
        \begin{equation*}
            B_2 = \left\{
                \begin{pmatrix}\bar{1}&\bar{0}\\\bar{0}&\bar{0}\end{pmatrix},
                \begin{pmatrix}\bar{0}&\bar{1}\\\bar{0}&\bar{0}\end{pmatrix},
                \begin{pmatrix}\bar{0}&\bar{0}\\\bar{1}&\bar{0}\end{pmatrix},
                \begin{pmatrix}\bar{0}&\bar{0}\\\bar{0}&\bar{1}\end{pmatrix},
            \right\}
        \end{equation*}

        Entonces, podemos encontrar la Matriz de Cambio de Coordenadas de $B_2$ al $B_1$ como:


        \begin{equation*}
            \begin{pmatrix}
                \bar{1} & \bar{0} & \bar{0} & \bar{0} | & \bar{1} & \bar{0} & \bar{1} & \bar{1} \\
                \bar{0} & \bar{1} & \bar{0} & \bar{0} | & \bar{2} & \bar{1} & \bar{1} & \bar{0} \\
                \bar{0} & \bar{0} & \bar{1} & \bar{0} | & \bar{4} & \bar{0} & \bar{3} & \bar{0} \\
                \bar{0} & \bar{0} & \bar{0} & \bar{1} | & \bar{0} & \bar{1} & \bar{0} & \bar{0} \\
            \end{pmatrix}
        \end{equation*}

        Por lo tanto podemos concordar que:
        \begin{equation*}
            C_{\frac{B_2}{B_1}} = 
            \begin{pmatrix}
                \bar{1} & \bar{0} & \bar{1} & \bar{1} \\
                \bar{2} & \bar{1} & \bar{1} & \bar{0} \\
                \bar{4} & \bar{0} & \bar{3} & \bar{0} \\
                \bar{0} & \bar{1} & \bar{0} & \bar{0} \\
            \end{pmatrix}
        \end{equation*}



% ======================================================================================
% ===================== ESPACIOS EUCLIDEANOS Y PRODUCTO INTERNO ========================
% ======================================================================================
\chapter{Espacios Euclideanos}
    \clearpage

    % =====================================================
    % ===== ESPACIOS EUCLIDEANOS Y PRODUCTO INTERNO   =====
    % =====================================================
    \section{Espacios Euclideanos}
        Son un espacio vectorial, en nuestro caso lo vamos a considerar sobre los reales, la principal caracteristica de estos espacios es que cumplen con que tienen un producto interno:


    % =====================================================
    % ================ PRODUCTO INTERNO ===================
    % =====================================================
    \clearpage
    \section{Producto Interno}
        Un producto interno será aquella función $<,>$ tal que reciba 2 vectores y
        te regrese un escalar: $\vec v \times \vec v \to \mathbb{R}$ tal que para
        todo 3 vectores cuales quiera $v, w, u \in V$ y para todo
        $\alpha , \beta \in  \mathbb{R} $ tenemos que:

        \begin{itemize}
            \item $<\alpha v + \beta w, u> = \alpha <v, u> + \beta <w, u>$
            \item $<u, v> = <v, u>$
            \item $<v, v> \geq 0 $ y $ <v, v> = 0 \leftrightarrow v = 0$
        \end{itemize}

        En el caso de que tenga un producto interno que cumpla estas caracteristicas
        podemos decir que nuestro espacio vectorial es Euclidiano.

        \subsection{Producto Internos Comunes}

        \begin{itemize}
            \item Matrices: $<A, B> = traza( transpuesta(A)B)$ \\ Es decir, es la suma de
            todos los elementos de la diagonal
            principal de la matriz resultante de la multiplicación de la transpuesta de A con B.

            \item $\mathbb{R}^n$: $<v, u> = v_x u_x + v_y u_y \cdots$ \\ Es decir, lo que
            conocemos como el producto punto. 
        \end{itemize}


        \subsection{Propiedades del Producto Interno}
        Podemos saber que:
        \begin{itemize}
            \item $<v, 0_v> = 0$
            \item Si $<u, v> = <v, u>$ y $\forall n \in V$, entonces $v = 0_v$
        \end{itemize}


    % =====================================================
    % =============== NORMA DE UN VECTOR ==================
    % =====================================================
    \clearpage
    \section{Norma de un Vector}
        Podemos definir una norma de un vector $v \in V$ como:
        \begin{equation}
            ||v|| = \sqrt{<v,v>}
        \end{equation}

        \subsection{Propiedades de la Norma}
        \begin{itemize}
            \item $||v|| \geq 0$ y también $||v|| = 0$ ssi $v = 0_v$   
            \item $|| \alpha v||  = |\alpha| ||v|| $
            \item $||<v, u>||  \leq ||u|| ||v|| $ Esta es conocida como Desigualdad de Cauchy-Shuartz
            \item $||v + u||  \leq ||u|| + ||v|| $ Esta es conocida como Desigualdad del Triangulo
        \end{itemize}



    % =====================================================
    % ===============      ORTOGONAL     ==================
    % =====================================================
    \clearpage
    \section{Conjuntos Octogonales}

        Decimos que el conjunto $S = \{v_1, v_2, \cdots, v_n \}$ de vectores de un espacio euclidiano es: \\

        i) Ortogonal: Si $\forall i, j \in \{1, \cdots, n \}$ $(i\neq j) \to <v_i, v_j> = 0$ \\

        i) Ortonormal: Si ademas de Ortogonal tenemos que $||v_i|| = 1$ $\forall i\in \{1, \cdots, n \}$

        % ==========================
        % ====   EJEMPLO   =========
        % ==========================
        \subsubsection{Ejemplo}

        Este conjunto no es ni Ortogonal ni Ortonormal:
        \begin{equation*}
            S_1 = \left< \begin{bmatrix} 1\\1\\0\end{bmatrix} , \begin{bmatrix} 1\\0\\1\end{bmatrix}, \begin{bmatrix} 0\\1\\1\end{bmatrix} \right> 
        \end{equation*}

        Este conjunto es Ortogonal pero no Ortonormal:
        \begin{equation*}
            S_1 = \left< \begin{bmatrix} 1\\0\\1\end{bmatrix} , \begin{bmatrix} 0\\-3\\0\end{bmatrix}, \begin{bmatrix} 2\\0\\-2\end{bmatrix} \right> 
        \end{equation*}


        \subsection{Propiedades}
        Sea $S = \{ v_1 , v_2 , \cdots, v_n \} \subseteq V$

        \begin{itemize}
            \item Si S es Ortogonal y $v_i \neq 0_v$ $\forall i \in \{1, \cdots, n\}$
            entonces podemos concluir que S es Linealmente independiente.

            \item Si S es Ortonormal, entonces el vector de la forma:\\
            $ w = v - <v, v_1> v_1 - <v, v_2> v_2 - \cdots - <v, v_n> v_n $\\
            O visto mas bonito $w = v - \sum_{i = 1}^{n} <v, v_i>v_i$\\
            es ortogonal a S $\forall v \in V$
        \end{itemize}



% =====================================================
% ============        BIBLIOGRAPHY   ==================
% =====================================================
\clearpage
\bibliographystyle{plain}
    \begin{thebibliography}{9}

    % ============ REFERENCE #1 ========
    \bibitem{Sitio1} 
        ProbRob
        \\\texttt{Youtube.com}


     

\end{thebibliography}



\end{document}
