% ****************************************************************************************
% ************************      ALGEBRA LINEAL BASICA         ****************************
% ****************************************************************************************


% =======================================================
% =======         HEADER FOR DOCUMENT        ============
% =======================================================
    % *********   DOCUMENT ITSELF   **************
    \documentclass[12pt]{report}                                    %Type of docuemtn and size of font
    \usepackage[margin=1.2in]{geometry}                             %Margins and Geometry pacakge
    \usepackage{ifthen}                                             %Allow simple programming
    \usepackage{hyperref}                                           %Create MetaData for a PDF and LINKS!
    \setlength{\parindent}{0pt}                                     %Eliminate ugly indentation
    \author{Oscar Andrés Rosas}                                     %Who I am

    % *********   LANGUAJE AND UFT-8   *********
    \usepackage[spanish]{babel}                                     %Please use spanish
    \usepackage[utf8]{inputenc}                                     %Please use spanish - UFT
    \usepackage[T1]{fontenc}                                        %Please use spanish
    \usepackage{textcmds}                                           %Allow us to use quoutes
    \usepackage{changepage}                                         %Allow us to use identate paragraphs

    % *********   MATH AND HIS STYLE  *********
    \usepackage{amsthm, amssymb, amsfonts, mathrsfs}                %Make math beautiful
    \usepackage[fleqn]{amsmath}                                     %Please make equations left
    \usepackage{centernot}                                          %Allow me to negate a symbol
    \decimalpoint                                                   %Use decimal point

    % *********   GRAPHICS AND IMAGES *********
    \usepackage{graphicx}                                           %Allow to create graphics
    \usepackage{wrapfig}                                            %Allow to create images
    \graphicspath{ {Graphics/} }                                    %Where are the images :D

    % *********   LISTS AND TABLES ***********
    \usepackage{listings}                                           %We will be using code here
    \usepackage[inline]{enumitem}                                   %We will need to enumarate
    \usepackage{tasks}                                              %Horizontal lists
    \usepackage{longtable}                                          %Lets make tables awesome
    \usepackage{booktabs}                                           %Lets make tables awesome
    \usepackage{tabularx}                                           %Lets make tables awesome
    \usepackage{multirow}                                           %Lets make tables awesome
    \usepackage{multicol}                                           %Create multicolumns

    % *********   HEADERS AND FOOTERS ********
    \usepackage{fancyhdr}                                           %Lets make awesome headers/footers
    \pagestyle{fancy}                                               %Lets make awesome headers/footers
    \setlength{\headheight}{16pt}                                   %Top line
    \setlength{\parskip}{0.5em}                                     %Top line
    \renewcommand{\footrulewidth}{0.5pt}                            %Bottom line

    \lhead{                                                         %Left Header
        \hyperlink{chapter.\arabic{chapter}}                        %Make a link to the current chapter
        {\normalsize{\textsc{\nouppercase{\leftmark}}}}             %And fot it put the name
    }

    \rhead{                                                         %Right Header
        \hyperlink{section.\arabic{chapter}.\arabic{section}}       %Make a link to the current chapter
            {\footnotesize{\textsc{\nouppercase{\rightmark}}}}      %And fot it put the name
    }

    \rfoot{\textsc{\small{\hyperref[sec:Index]{Ve al Índice}}}}     %This will always be a footer  

    \fancyfoot[L]{                                                  %Algoritm for a changing footer
        \ifthenelse{\isodd{\value{page}}}                           %IF ODD PAGE:
            {\href{https://compilandoconocimiento.com/yo/}          %DO THIS:
                {\footnotesize                                      %Send the page
                    {\textsc{Oscar Andrés Rosas}}}}                 %Send the page
            {\href{https://compilandoconocimiento.com}              %ELSE DO THIS: 
                {\footnotesize                                      %Send the author
                    {\textsc{Compilando Conocimiento}}}}            %Send the author
    }
    
    
    
% ========================================
% ===========   COMMANDS    ==============
% ========================================

    % =====  GENERAL TEXT  ==========
    \newcommand \Quote {\qq}                                        %Use: \Quote to use quotes
    \newenvironment{Indentation}[1][0.75em]                         %Use: \begin{Inde...}[Num]...\end{Inde...}
    {\begin{adjustwidth}{#1}{}}                                     %If you dont put nothing i will use 0.75 em
    {\end{adjustwidth}}                                             %This indentate a paragraph
    \newenvironment{SmallIndentation}[1][0.75em]                    %Use: The same that we upper one, just 
    {\begin{adjustwidth}{#1}{}\begin{footnotesize}}                 %footnotesize size of letter by default
    {\end{footnotesize}\end{adjustwidth}}                           %that's it
        
    % =====  GENERAL MATH  ==========
    \DeclareMathOperator \Space {\quad}                             %Use: \Space for a cool mega space
    \DeclareMathOperator \MiniSpace {\;}                            %Use: \Space for a cool mini space
    \newcommand \Such {\MiniSpace|\MiniSpace}                       %Use: \Such like in sets

    % =====  LOGIC  ==================
    \DeclareMathOperator \doublearrow {\leftrightarrow}             %Use: \doublearrow for a double arrow
    \newcommand \lequal {\MiniSpace \Leftrightarrow \MiniSpace}     %Use: \lequal for a double arrow
    \newcommand \linfire {\MiniSpace \Rightarrow \MiniSpace}        %Use: \lequal for a double arrow
    \newcommand \longto {\longrightarrow}                           %Use: \longto for a long arrow

    % =====  NUMBER THEORY  ==========
    \DeclareMathOperator \Naturals  {\mathbb{N}}                     %Use: \Naturals por Notation
    \DeclareMathOperator \Primes    {\mathbb{P}}                     %Use: \Naturals por Notation
    \DeclareMathOperator \Integers  {\mathbb{Z}}                     %Use: \Integers por Notation
    \DeclareMathOperator \Racionals {\mathbb{Q}}                     %Use: \Racionals por Notation
    \DeclareMathOperator \Reals     {\mathbb{R}}                     %Use: \Reals por Notation
    \DeclareMathOperator \Complexs  {\mathbb{C}}                     %Use: \Complex por Notation

    % === LINEAL ALGEBRA & VECTORS ===
    \DeclareMathOperator \LinealTransformation {\mathcal{T}}        %Use: \LinealTransformation for a cool T

    \newcommand{\pVector}[1]{                                       %Use: \pVector {Matrix Notation} use parentesis
        \ensuremath{\begin{pmatrix}#1\end{pmatrix}}                 %Example: \pVector{a\\b\\c} or \pVector{a&b&c} 
    }
    \newcommand{\lVector}[1]{                                       %Use: \lVector {Matrix Notation} use a abs 
        \ensuremath{\begin{vmatrix}#1\end{vmatrix}}                 %Example: \lVector{a\\b\\c} or \lVector{a&b&c} 
    }
    \newcommand{\bVector}[1]{                                       %Use: \bVector {Matrix Notation} use a brackets 
        \ensuremath{\begin{bmatrix}#1\end{bmatrix}}                 %Example: \bVector{a\\b\\c} or \bVector{a&b&c} 
    }
    \newcommand{\Vector}[1]{                                        %Use: \Vector {Matrix Notation} no parentesis
        \ensuremath{\begin{matrix}#1\end{matrix}}                   %Example: \Vector{a\\b\\c} or \Vector{a&b&c}
    }

    % MATRIX
    \makeatletter                                                   %Example: \begin{matrix}[cc|c]
    \renewcommand*\env@matrix[1][*\c@MaxMatrixCols c] {             %WTF! IS THIS
        \hskip -\arraycolsep                                        %WTF! IS THIS
        \let\@ifnextchar\new@ifnextchar                             %WTF! IS THIS
        \array{#1}                                                  %WTF! IS THIS
    }                                                               %WTF! IS THIS
    \makeatother                                                    %WTF! IS THIS




% =====================================================
% ============     	  COVER PAGE	   ================
% =====================================================
\begin{document}
\begin{titlepage}

	\center
	% ============ UNIVERSITY NAME AND DATA =========
	\textbf{\textsc{\Large Proyecto Compilando Conocimiento}}\\[1.0cm] 
	\textsc{\Large Matemáticas Discretas}\\[1.0cm] 

	% ============ NAME OF THE DOCUMENT  ============
	\rule{\linewidth}{0.5mm} \\[1.0cm]
		{ \huge \bfseries Algebra Lineal}\\[1.0cm] 
	\rule{\linewidth}{0.5mm} \\[2.0cm]
	
	% ====== SEMI TITLE ==========
	{\LARGE Una Pequeña (Gran) Introducción}\\[7cm] 
	
	% ============  MY INFORMATION  =================
	\begin{center} \large
	\textbf{\textsc{Autor:}}\\
	Rosas Hernandez Oscar Andrés
	\end{center}

	\vfill

\end{titlepage}

% =====================================================
% ========                INDICE              =========
% =====================================================
\tableofcontents{}
\label{sec:Index}

\clearpage




% //////////////////////////////////////////////////////////////////////////////////////////////////////////
% //////////////////////////////////////           MATRICES        /////////////////////////////////////////
% //////////////////////////////////////////////////////////////////////////////////////////////////////////
\part{Matrices}
\clearpage


    % ===============================================================================
    % ===================    ENTENDAMOS A LAS MATRICES         ======================
    % ===============================================================================
    \chapter{Conozcamos las Matrices}

        % ==============================================
        % ========          DEFINICION            ======
        % ==============================================
        \clearpage
        \section{Definición}

            Siendo formales una Matriz es un arreglo rectangular de $m \times n$ elementos 
            (donde $m,n \in \Naturals$), es decir es un objecto matemático de $m$ filas y
            de $n$ columnas.

            Las entradas de matrices pueden ser números u objetos más complicados.
            
            Sea $\mathbb{F}$ un Campo, entonces decimos que $M_{m \times n}(\mathbb{F})$
            al conjunto de todas las matrices de tamaños $m \times n$ cuyas entradas
            pertenecen a $\mathbb{F}$.
            \begin{equation}
                A = 
                \begin{bmatrix}[ccc]
                    a _{1, 1}   & \cdots & a_{1,n}   \\
                    \cdots      &        & \cdots    \\
                    a _{m, 1}   & \cdots & a_{m,n}   \\
                \end{bmatrix}
            \end{equation}


            % ====================================
            % =====   DEFINICION FORMAL     ======
            % ====================================
            \subsection*{Definición más Formal}
                Una matriz de tamaño $m \times n$ con elementos del campo $\mathbb{F}$ se puede
                definir como una función: 
                $\{1, \dots, m\} \times \{1, \dots , n\} \to \mathbb{F}$



            % ====================================
            % =====   SIMBOLOGIA HERMOSA    ======
            % ====================================
            \subsection{Notación de Matrices mediante Función}

                La notación más rara y al mismo tiempo más increíble es:
                \begin{equation}
                    A   = [f(i,j)]_{i, j = 1}^{m, n}
                        =
                        \begin{bmatrix}[ccc]
                            f(1,1)  & \cdots & f(1,n)   \\
                            \cdots  &        & \cdots   \\
                            f(m, 1) & \cdots & f(m,n)   \\
                        \end{bmatrix}
                \end{equation}

                Significa $A$ es una matriz de tamaño $m \times n$ tal que su entrada
                ubicada en la fila número $i$ y en la columna $j$ es igual a la función
                $f: \{1, \dots, m\} \times \{1, \dots, n\} \to \mathbb{F}$.
                
                Aquí $f$ es una función de dos argumentos.
                
                Para hablar de un elemento cualquiera de la Matriz $A$ decimos de manera
                informal $[A]_{i,j}$


        % ==============================================
        % ========          SIMBOLOGIA            ======
        % ==============================================
        \clearpage
        \section{Simbología}

            Solemos denotar con letras mayúsculas a las matrices y con letras miniscúlas
            a cada uno de los elementos.

            Para hablar de un elemento en específico usamos $a_{i,j}$ donde $i$ es el
            número de fila y $j$ es el número de columnas.


            % =============================
            % ========   EJEMPLO     ======
            % =============================
            \subsubsection*{Ejemplo}

                Por ejemplo, una matriz sería:
                \begin{equation*}
                    A =
                    \begin{bmatrix}[ccc]
                        a & b & c   \\
                        d & e & f   \\
                    \end{bmatrix}
                \end{equation*}

                y $a_{1,3}$ es el elemento $c$.


        % ==============================================
        % =======       DELTA DE KRONECKER        ======
        % ==============================================
        \section{Delta de Kronecker}

            Esta es una función demasiado sencilla $\delta(i,j): \Naturals^2 \to \{0,1\}$
            pero muy importante a lo largo de Algebra Lineal, podemos definirla como:
            \begin{equation}
                \delta(i,j) =
                \begin{cases}
                    1 \Space \text{ si } i = j \\
                    0 \Space \text{ si } i \neq j
                \end{cases}
            \end{equation}



        % ==============================================
        % =======   CLASIFICACION DE MATRICES     ======
        % ==============================================
        \clearpage
        \section{Clasificación y Matrices Famosas}

            % ===================================
            % ===  MATRICES RECTANGULARES   =====
            % ===================================
            \subsection{Matrices Rectangulares}

                Son aquellas matrices de $m \times n$ si es que $m \neq n$ 

                Por ejemplo: 
                \begin{equation*}
                    A_{m \times n} =
                    \begin{bmatrix}[ccc]
                        1 & 2 & 3   \\
                        4 & 5 & 6   \\
                    \end{bmatrix}
                \end{equation*}

            % ===================================
            % =======  MATRICES CUADRADAS   =====
            % ===================================
            \subsection{Matrices Cuadradas}

                Son aquellas matrices de $m \times n$ si es que $m = n$.
                Solemos decir que el orden de estas matrices es $n$.

                Por ejemplo: 
                \begin{equation*}
                    A_{n \times n} =
                    \begin{bmatrix}[ccc]
                        1 & 2 \\
                        4 & 5 \\
                    \end{bmatrix}
                \end{equation*}

            % ===================================
            % =======  MATRICES DIAGONALES   ====
            % ===================================
            \subsection{Matrices Diagonales}

                Son todas las matrices cuadradas donde cada elemento cumple que:
                \begin{equation}
                    [A]_{i,j} = [A]_{i,j} \cdot \delta(i,j)
                \end{equation}

                O más formalmente como cualquier matriz que cumple con que:
                \begin{equation}
                    [f(i,j)]_{i, j = 1}^{m, n} = [ f(i,j) \cdot \delta(i,j) ]_{i, j = 1}^{m, n}  
                \end{equation}

                Es decir es una matriz en la que a cualquier elemento lo puedes multiplicar 
                por la Delta de Kronecker correspondiente y no se vera afectado.

                Una matriz diagonal tiene el siguiente aspecto:
                \begin{equation*}
                    A_n =
                    \begin{bmatrix}[cccc]
                        a_{1,1} & 0         & \dots & 0         \\
                        0       & a_{2,2}   & \dots & 0         \\
                        \vdots                                  \\
                        0       & 0         & \dots & a_{n,n}   \\
                    \end{bmatrix}
                \end{equation*}



            % ===================================
            % =======  MATRICES IDENTIDAD    ====
            % ===================================
            \clearpage
            \subsection{Matriz Identidad: $I_n$}

                Son todas las matrices cuadradas donde cada elemento cumple que:
                \begin{equation}
                    [I]_{i,j} = \delta(i,j)
                \end{equation}

                O más formalmente podemos definir a la Matriz identidad de órden $n$ como:
                \begin{equation}
                    [\delta(i,j)]_{i, j = 1}^{n, n}
                \end{equation}

                Se ve algo así:
                \begin{equation*}
                    I_n =
                    \begin{bmatrix}[cccc]
                        1 & 0 & \dots & 0   \\
                        0 & 1 & \dots & 0   \\
                        \vdots              \\
                        0 & 0 & \dots & 1   \\
                    \end{bmatrix}
                \end{equation*}



            % ===================================
            % =======  MATRICES CERO         ====
            % ===================================
            \subsection{Matriz Cero: $0_{m \times n}$}

                Son todas aquellas matrices $m \times n$ que cumplen que para cada elemento:
                \begin{equation}
                    [0]_{i,j} = 0_K
                \end{equation}

                O más formalmente podemos definir a la Matriz de Ceros de órden $n$ como:
                \begin{equation}
                    [0_{\mathbb{F}}]_{i, j = 1}^{n, n}
                \end{equation}

                Se ven algo así:
                \begin{equation*}
                    0_{m \times n} =
                    \begin{bmatrix}[cccc]
                        0 & 0 & \dots & 0   \\
                        0 & 0 & \dots & 0   \\
                        \vdots              \\
                        0 & 0 & \dots & 0   \\
                    \end{bmatrix}
                \end{equation*}












    % ===============================================================================
    % ===================    OPERACIONES CON MATRICES          ======================
    % ===============================================================================
    \clearpage
    \chapter{Álgebra Matricial}

        % ==============================================
        % ==========     SUMA DE MATRICES      =========
        % ==============================================
        \clearpage
        \section{Suma de Matrices}

            Definimos la suma de dos Matrices $A, B \in M_{m \times n}(\mathbb{F})$ como una relación 
            \\ $+: (M_{m \times n}, M_{m \times n}) \to M_{m \times n}$

            Entonces definamos la suma de dos matrices $A, B \in M_{m \times n}(\mathbb{F})$ como:
            \begin{equation}
                A + B = [A_{i, j} + B_{i, j}]_{i, j = 1}^{m, n}
            \end{equation}

            O visto de otra manera $A + B \in M_{m \times n}(\mathbb{F})$ y cumple que:
            \begin{equation}
                \forall i \in \{1, \dots, m\} ,\MiniSpace
                    \forall j \in \{1, \dots, n\} ,\Space
                        (A+B)_{i, j} = A_{i, j} + B_{i, j}
            \end{equation}


            % ================================
            % ===      PROPIEDADES       =====
            % ================================
            \subsection{Propiedades de Suma}

                Sea $A, B \in M_{m \times n}(\mathbb{F})$ y $\alpha, \beta \in \mathbb{F}$
                y con la suma y producto por escalar previamente definido tenemos que:

                \begin{itemize}

                    \item \textbf{Cerradura Aditiva:}\\
                        Si $A, B \in M_{m \times n}(\mathbb{F})$ entonces 
                        $(A+B) \in M_{m \times n}(\mathbb{F})$


                    \item \textbf{Ley Conmutativa:}\\
                        Si $A, B \in M_{m \times n}(\mathbb{F})$ entonces $A+B = B+A$

                    \item \textbf{Ley Asociativa para la Suma:}\\
                        Si $A, B, C \in M_{m \times n}(\mathbb{F})$ entonces 
                        $A + (B+C) = (A+B) + C$

                    \item \textbf{Existencia del Neutro Aditivo:}\\
                        Existe una matriz $\varnothing \in M_{m \times n}(\mathbb{F})$
                        tal que $\forall A \in M_{m \times n}(\mathbb{F}), \MiniSpace A + \varnothing = A$

                    \item \textbf{Existencia del Inverso Aditivo:}\\
                        Existe una matriz $-A \in M_{m \times n}(\mathbb{F})$ para toda 
                        $A \in M_{m \times n}(\mathbb{F})$ tal que $ A + (-A) = \varnothing$

                \end{itemize}




        % ==============================================
        % ====  PRODUCTO DE ESCALAR POR MATRIZ    ======
        % ==============================================
        \clearpage
        \section{Producto de Escalar por Matriz}

            Sea $A \in M_{m \times n}(\mathbb{F})$ y $\alpha \in \mathbb{F}$ entonces 
            definimos a $ \alpha A$ como:
            \begin{equation}
                A \alpha = \alpha A = [\alpha A_{i, j}]_{i, j = 1}^{m, n}
            \end{equation}

            O visto de otra manera $\alpha A \in M_{m \times n}(\mathbb{F})$ y cumple que:
            \begin{equation}
                \forall i \in \{1, \dots, m\} ,\MiniSpace
                    \forall j \in \{1, \dots, n\} ,\Space
                        (\alpha A)_{i, j} = \alpha A_{i, j}
            \end{equation}

            % ================================
            % ===      PROPIEDADES       =====
            % ================================
            \subsection{Propiedades del Producto Escalar}

                Sea $A, B \in M_{m \times n}(\mathbb{F})$ y $\alpha, \beta \in \mathbb{F}$
                y con la suma y producto por escalar previamente definido tenemos que:

                \begin{itemize}

                    \item \textbf{Cerradura Escalar:}\\
                        Si $A\in M_{m \times n}(\mathbb{F})$ y $\alpha \in \mathbb{F}$ entonces 
                        $(\alpha A) \in M_{m \times n}(\mathbb{F})$

                    \item \textbf{Ley Asociativa para la Multiplicación Escalar:}\\
                        Sea $A \in M_{m \times n}(\mathbb{F})$ y $\alpha, \beta \in \mathbb{F}$
                        entonces $\alpha(\beta A) = (\alpha \beta)A$

                    \item \textbf{Ley Distributiva en la Suma y Producto Escalar:}\\
                        Sea $A, B \in M_{m \times n}(\mathbb{F})$ y $\alpha \in \mathbb{F}$
                        entonces $\alpha(A + B) = (\alpha A) + (\alpha B)$

                    \item \textbf{Ley Distributiva en los Escalares:}\\
                        Sea $A \in M_{m \times n}(\mathbb{F})$ y $\alpha, \beta \in \mathbb{F}$
                        entonces $(\alpha + \beta)A = (\alpha A) + (\beta A)$

                    \item \textbf{Existencia del Neutro Multiplicativo Escalar:}\\
                        Existe un elemento $1 \in \mathbb{F}$ tal que para toda
                            $A \in M_{m \times n}(\mathbb{F})$ tenemos que $1A = A$

                \end{itemize}




        % ==============================================
        % ====        PRODUCTO DE MATRICES        ======
        % ==============================================
        \clearpage
        \section{Producto de Matrices}

            Sea $A \in M_{m \times n}(\mathbb{F})$ y $B \in M_{n \times p}(\mathbb{F})$ entonces 
            definimos a $AB$ como:
            \begin{equation}
                AB = \bVector{\sum_{k=1}^{n} A_{i, k}B_{k, j} }_{i, j = 1}^{m, p}
            \end{equation}

            O visto de otra manera $AB \in M_{m \times p}(\mathbb{F})$ y cumple que:
            \begin{equation}
                \forall i \in \{1, \dots, m\} ,\MiniSpace
                    \forall j \in \{1, \dots, n\} ,\Space
                        (AB)_{i, j} = \sum_{k=1}^{n} A_{i, k}B_{k, j}
            \end{equation}


            % ===============================
            % =========   PROPIEDADES =======
            % ===============================
            \clearpage
            \subsection{Propiedades}

                \begin{itemize}

                    \item Sea $A \in M_{m \times n}(\mathbb{F})$ y $B,C \in M_{n \times p}(\mathbb{F})$
                        entonces tenemos que:
                        $A(B+C) = AB+AC$

                        % ======== DEMOSTRACION ========
                        \begin{SmallIndentation}[1em]
                            \textbf{Demostración}:

                            Empecemos por ver que tienen el mismo tamaño:
                            La matriz $(B+C) \in M_{n \times p}(\mathbb{F})$, por lo que 
                            $A(B+C) \in M_{m \times p}(\mathbb{F})$.
                            También tenemos que $AB, AC \in M_{m \times p}(\mathbb{F})$
                            Por lo tanto tienen el mismo tamaño.

                            Ahora veamos que un cualquier elemento arbitrario de ambas matrices es igual:
                            \begin{equation*}
                            \begin{split}
                                [A(B+C)]_{i, j}    
                                    &= \sum_{k=1}^{n} A_{i, k}(B_{k, j}+C_{k, j})  \\            
                                    &= \sum_{k=1}^{n} (A_{i, k}B_{k, j}) + (A_{i, k}C_{k, j}) 
                                    = \sum_{k=1}^{n} (A_{i, k}B_{k, j}) + \sum_{k=1}^{n} (A_{i, k}C_{k, j}) \\
                                    &= [AB]_{i, j} + [AC]_{i, j} = [AB + AC]_{i, j} 
                            \end{split}
                            \end{equation*}

                        \end{SmallIndentation}

                    \item Sea $A \in M_{m \times n}(\mathbb{F})$ y $B \in M_{n \times p}(\mathbb{F})$
                        entonces tenemos que: $\alpha(AB) = A(\alpha B)$

                    % ======== DEMOSTRACION ========
                    \begin{SmallIndentation}[1em]
                        \textbf{Demostración}:

                        Creo que es más que obvio que tienen el mismo tamaño

                        Ahora veamos que un cualquier elemento arbitrario de ambas matrices es igual:
                        \begin{equation*}
                        \begin{split}
                            [\alpha(AB)]_{i, j}    
                                = \alpha \sum_{k=1}^{n} A_{i, k} B_{k, j}              
                                = \sum_{k=1}^{n} A_{i, k} (\alpha B_{k, j} )            
                                = [A(\alpha B)]_{i, j}
                        \end{split}
                        \end{equation*}

                    \end{SmallIndentation}

                    \item Sea $A \in M_{m \times n}(\mathbb{F})$, $B \in M_{n \times p}(\mathbb{F})$
                        y $C \in M_{p \times q}(\mathbb{F})$ entonces tenemos que:  \\
                        $A(BC) = (AB)C$

                        % ======== DEMOSTRACION ========
                        \begin{SmallIndentation}[1em]
                            \textbf{Demostración}:

                            Empecemos por ver que tienen el mismo tamaño:
                            La matriz $(BC) \in M_{n \times q}(\mathbb{F})$, por lo que 
                            $A(BC) \in M_{m \times q}(\mathbb{F})$.
                            También tenemos que $(AB) \in M_{m \times p}(\mathbb{F})$, por lo que tenemos
                            que $(AB)C \in M_{m \times q}(\mathbb{F})$.
                            Por lo tanto tienen el mismo tamaño.

                            Ahora veamos que un cualquier elemento arbitrario de ambas matrices es igual:
                            \begin{equation*}
                            \begin{split}
                                [A(BC)]_{i, j}    
                                    &= \sum_{k=1}^n A_{i, k} (BC)_{k, j}  
                                     = \sum_{k=1}^n A_{i, k} \pVector{\sum_{k'=1}^p B_{k, k'} C_{k', j} }   \\
                                    &= \sum_{k'=1}^n A_{i, k'} \pVector{\sum_{k=1}^p B_{k', k} C_{k, j} }   
                                     = \sum_{k'=1}^n \pVector{\sum_{k=1}^p A_{i, k'} B_{k', k} C_{k, j} }   \\
                                    &= \sum_{k=1}^p \pVector{\sum_{k'=1}^n A_{i,k'} B_{k',k} } C_{k, j}  
                                     = \sum_{k=1}^p AB_{i, k} C_{k, j}
                                     = [(AB)C]_{i, j}
                            \end{split}
                            \end{equation*}

                        \end{SmallIndentation}

                \end{itemize}




        % ==============================================
        % ====        TRAZA DE UNA MATRIZ         ======
        % ==============================================
        \clearpage
        \section{Traza de una Matriz}

            Sea $A \in M_{n \times n}(\mathbb{F})$ entonces definimos a $traza(A)$ como:
            \begin{equation}
                traza(A) = tr(A) = \sum_{k=1}^{n} A_{k, k}
            \end{equation}


            % ===============================
            % =========   PROPIEDADES =======
            % ===============================
            \subsection{Propiedades}

                \begin{itemize}
                    \item Si $A, B \in M_{m \times n}(\mathbb{F})$ entonces $traza(AB) = traza(BA)$
                \end{itemize}




        % ==============================================
        % ====   TRANSPUESTA DE UNA MATRIZ        ======
        % ==============================================
        \clearpage
        \section{Transpuesta de una Matriz}

            % ===============================
            % ======   DEFINICION     =======
            % ===============================
            \subsection{Definición}

                Sea $A \in M_{m \times n}(\mathbb{F})$ entonces definimos a $transpuesta(A)$ como:
                \begin{equation}
                    A^T = \bVector{ A_{j, i} }_{i, j = 1}^{n, m}
                \end{equation}

                Es decir $A^T \in M_{n \times m}(\mathbb{F})$

                O visto de otra manera $A^T \in M_{n \times m}(\mathbb{F})$ y cumple que:
                \begin{equation}
                    \forall i \in \{1, \dots, n\} ,\MiniSpace
                        \forall j \in \{1, \dots, m\} ,\Space
                            (A^T)_{i, j} = A_{j, i}
                \end{equation}


            % ===============================
            % =========   PROPIEDADES =======
            % ===============================
            \clearpage
            \subsection{Propiedades}

                \begin{itemize}

                    \item Sea $A\in M_{m \times n}(\mathbb{F})$ entonces $(A^T)^T = A^T$
                        % ======== DEMOSTRACION ========
                        \begin{SmallIndentation}[1em]
                            \textbf{Demostración}:

                            Empecemos por ver que tienen el mismo tamaño:
                            La matriz $(A^T) \in M_{n \times m}(\mathbb{F})$, por lo que 
                            $(A^T)^T \in M_{m \times n}(\mathbb{F})$.
                            Por lo tanto tienen el mismo tamaño.

                            Ahora veamos que un cualquier elemento arbitrario de ambas matrices es igual:
                            \begin{equation*}
                            \begin{split}
                                [(A^T)^T]_{i, j}    
                                    = [(A^T)]_{j, i}               
                                = [A]_{i, j}
                            \end{split}
                            \end{equation*}

                        \end{SmallIndentation}

                    \item Sea $A,B \in M_{m \times n}(\mathbb{F})$ entonces 
                        $(A+B)^T = A^T + B^T$

                        % ======== DEMOSTRACION ========
                        \begin{SmallIndentation}[1em]
                            \textbf{Demostración}:

                            Empecemos por ver que tienen el mismo tamaño:
                            La matriz $(A+B)^T \in M_{n \times m}(\mathbb{F})$, y también tenemos que
                            $(A^T+B^T) \in M_{n \times m}(\mathbb{F})$. Por lo tanto tienen el mismo
                            tamaño.

                            Ahora veamos que un cualquier elemento arbitrario de ambas matrices es igual:
                            \begin{equation*}
                            \begin{split}
                                [(A+B)^T]_{i, j}    
                                    = [(A+B)]_{j, i}               
                                    = [A]_{j, i} + [B]_{j, i}      
                                    = [A^T]_{i, j} + [B^T]_{i, j}
                                = [A^T + B^T]_{i, j}
                            \end{split}
                            \end{equation*}

                        \end{SmallIndentation}

                    \item Sea $A \in M_{m \times n}(\mathbb{F})$ y $\alpha \in \mathbb{F}$ entonces:
                        $(\alpha A)^T = \alpha A^T$
                        
                        % ======== DEMOSTRACION ========
                        \begin{SmallIndentation}[1em]
                            \textbf{Demostración}:

                            Es (creo) más que obvio que tendrán el mismo tamaño

                            Ahora veamos que un cualquier elemento arbitrario de ambas matrices es igual:
                            \begin{equation*}
                            \begin{split}
                                [(\alpha A)^T]_{i, j}    
                                    = [\alpha A]_{j, i}               
                                    = \alpha [A]_{j, i}
                                = \alpha [A^T]_{i, j}
                            \end{split}
                            \end{equation*}

                        \end{SmallIndentation}
                                    


                    \item Sea $A \in M_{m \times n}(\mathbb{F})$ y $B \in M_{n \times p}(\mathbb{F})$
                        entonces tenemos que: $(AB)^T = B^T A^T$

                        % ======== DEMOSTRACION ========
                        \begin{SmallIndentation}[1em]
                            \textbf{Demostración}:

                            Veamos que ambas matrices tienen el mismo tamaño: 
                            La matriz $AB \in M_{m \times p}(\mathbb{F})$, por lo tanto la matriz
                            $(AB)^T \in M_{p \times m}(\mathbb{F})$, mientra que la matriz 
                            $B^T \in M_{p \times n}(\mathbb{F})$ y $A^T \in M_{n \times m}(\mathbb{F})$
                            por lo tanto $B^T A^T \in M_{p \times m}(\mathbb{F})$, así que si te das
                            cuenta ¡Tienen el mismo tamaño!

                            Ahora veamos que un cualquier elemento arbitrario de ambas matrices es igual:
                            \begin{equation*}
                            \begin{split}
                                [(AB)^T]_{i, j}     
                                    = [(AB)]_{j, i} = \sum_{k=1}^{n} A_{j, k}B_{k, i} 
                                    = \sum_{k=1}^{n} B_{k, i} A_{j, k}   
                                    = \sum_{k=1}^{n} B^T_{i, k} A^T_{k, j}            
                                = [B^T A^T]_{i, j}
                            \end{split}
                            \end{equation*}

                        \end{SmallIndentation}

                \end{itemize}


                        
            % ===============================
            % ===   MATRICES SIMETRICAS  ====
            % ===============================
            \clearpage
            \subsection{Matrices Simétricas}

                Una matriz $A \in M_{n}(\mathbb{F})$ se dice simétrica si cumple la propiedad:
                \begin{equation}
                    A = A^T
                \end{equation}

                Podemos ver que entrada








        % ==============================================
        % =====     OPERACIONES ELEMENTALES       ======
        % ==============================================
        \clearpage
        \section{Operaciones Elementales}

            % ==============================================
            % ======== SWAP: INTERCAMBIAR FILAS / COL ======
            % ==============================================
            \subsection{Swap: Intercambiar Filas ó Columnas}

                \begin{itemize}
                    \item
                        Decimos que vamos a intercambiar la Fila $i$ por la Fila $j$ de
                        esta manera:\\
                        $\overset{F_i \lequal F_j}{\longto}$

                    \item
                        Decimos que vamos a intercambiar la Columna $i$ por la Columna $j$ de
                        esta manera:\\
                        $\overset{C_i \lequal C_j}{\longto}$
                \end{itemize}

                % ==================================
                % =======  MATRIZ ELEMENTAL    =====
                % ==================================
                \subsubsection{Matriz Elemental}

                    Podemos si queremos expresar esta operación como una \Quote{Matriz Elemental}:
                    $E_{a, b}$

                    Donde $E_{a, b}$ es casi la identidad, pero estan intercambiadas la Fila
                    $a$ por la Fila $b$.












% //////////////////////////////////////////////////////////////////////////////////////////////////////////
% //////////////////////////////      SISTEMA DE ECUACIONES LINEALES        ////////////////////////////////
% //////////////////////////////////////////////////////////////////////////////////////////////////////////
\part{Sistema de Ecuaciones Lineales}

    % ===============================================================================
    % ===================    SISTEMAS DE ECUACIONES LINEALES        =================
    % ===============================================================================
    \chapter{Sistemas de Ecuaciones Lineales}
        \clearpage


        % =====================================================
        % ==============      GENERALIDADES     ===============
        % =====================================================
        \section{Generalidades}

            Podemos usar las matrices y álgebra lineal para encontrar las soluciones
            de un sistema de ecuaciones lineales dentro de cualquier campo (eso quiere
            decir que podemos ocuparla incluso para resolver sistemas en el campo de
            los complejos o el campo enteros módulo n).

            % ==================================
            % ===   ECUACIONES LINEALES   ======
            % ==================================
            \subsection{Definición: Ecuaciones Lineales}

                Este es muy obvio pero mejor lo digo, TODAS las ecuaciones debe ser lineales,
                es decir estar escritas de la forma:

                \begin{equation}
                    a_1x_1 + a_2x_2 + a_3x_3 + \dots + a_nx_n = b
                \end{equation}


                Por lo tanto podemos definir un sistema de $m \times n$ (es decir $m$ ecuaciones
                con $n$ incognitas) ecuaciones lineales como:

                \begin{equation*}
                    m \text{ ecuaciones}
                    \begin{cases}
                        & a_{1,1}x_1 + a_{1,2}x_2 + a_{1,3}x_3 + \dots + a_{1,n}x_n = b_1 \\
                        & a_{2,1}x_1 + a_{2,2}x_2 + a_{2,3}x_3 + \dots + a_{2,n}x_n = b_2 \\
                        & \cdots \\
                        & a_{m,1}x_1 + a_{m,2}x_2 + a_{m,3}x_3 + \dots + a_{m,n}x_n = b_m \\
                    \end{cases}      
                \end{equation*}

                \begin{equation*}
                    \begin{split}
                        &a_{1,1}x_1 + a_{1,2}x_2 + a_{1,3}x_3 + \dots + a_{1,n}x_n = b_1 \\
                        &a_{2,1}x_1 + a_{2,2}x_2 + a_{2,3}x_3 + \dots + a_{2,n}x_n = b_2 \\
                        &\cdots \\
                        &\underbrace{a_{m,1}x_1 + a_{m,2}x_2 + a_{m,3}x_3 + \dots + a_{m,n}x_n = b_m}_\text{$n$ incognitas}
                    \end{split}
                \end{equation*}


            % ==================================
            % ===   MATRIZ AMPLIADA       ======
            % ==================================
            \clearpage
            \subsection{Matriz Ampliada}
                La forma en la que Algebra Lineal nos ayuda a resolver nuestro sistema de ecuaciones
                es mediante una matriz ampliada, que no es más que convertir nuestro sistema
                de ecuaciones de esta manera:

                Desde algo así:
                \begin{equation}
                    \begin{split}
                        & a_{1,1}x_1 &&+ a_{1,2}x_2 &&+ a_{1,3}x_3 &&+ \dots &&+ a_{1,n}x_n &&= b_1 \\
                        & a_{2,1}x_1 &&+ a_{2,2}x_2 &&+ a_{2,3}x_3 &&+ \dots &&+ a_{2,n}x_n &&= b_2 \\
                        & \cdots                                                                    \\
                        & a_{m,1}x_1 &&+ a_{m,2}x_2 &&+ a_{m,3}x_3 &&+ \dots &&+ a_{m,n}x_n &&= b_m \\
                    \end{split}
                \end{equation}

                Hasta algo así:
                \begin{equation}
                    \begin{bmatrix}[ccccc|c]
                        a_{1,1} & a_{1,2} & a_{1,3} & \dots  & a_{1,n} & b_1     \\
                        a_{2,1} & a_{2,2} & a_{2,3} & \dots  & a_{2,n} & b_2     \\
                        \cdots  & \cdots  & \cdots  & \cdots & \cdots  & \cdots  \\
                        a_{m,1} & a_{m,2} & a_{m,3} & \dots  & a_{m,n} & b_m     \\
                    \end{bmatrix}
                \end{equation}


            % ==================================
            % ===         EJEMPLOS        ======
            % ==================================
            \clearpage
            \subsection{Ejemplos}

                Supongamos que tenengamos este sistema:
                \begin{equation*}
                    \begin{split}
                        &(2)x_1  &&+  (3)x_2 &&+ (-1)x_3 &&= 0    \\
                        &(-1)x_1 &&+  (2)x_2 &&+ (-3)x_3 &&= -3   \\
                        &(3)x_1  &&+  (5)x_2 &&+ (7)x_3  &&= 5    \\
                    \end{split}
                \end{equation*}

                Entonces la Matriz Ampliada es:
                \begin{equation*}
                    \begin{bmatrix}[rrr|r]
                        2  & 3 & -1 & 0  \\
                        -1 & 2 & -3 & -3 \\
                        3  & 5 & 7  & 5  \\
                    \end{bmatrix}
                \end{equation*}



        % =====================================================
        % ==========    TIPOS DE SOLUCIONES     ===============
        % =====================================================
        \clearpage
        \section{Tipos de Soluciones}

            Recordemos antes que nada sobre estas ecuaciones, cada una de ellas representa
            algo en el espacio y podemos \Quote{solucionarlas} al dibujarlas en el espacio:

            Y podemos separar nuestras soluciones en 2 (ó 3) amplias zonas:


            % ==================================
            % ===   SISTEMAS CONSISTENTES  =====
            % ==================================
            \subsection{Sistemas Consistentes (Mínimo 1 Solución)}

                Podemos tener primeramente sistemas consistentes, es decir que tienen
                \textbf{mínimo} una solución.

                Aquí hay dos opciones:

                \begin{itemize}

                    \item
                        \textbf{Sistemas Consistentes Independientes: Tocan en un Punto}

                        % ======== EXPLICACION ========
                        \begin{SmallIndentation}[1em]
                            Que es lo esperado y a lo que yo llamaría normal.
                            Por lo tanto si tocan en un punto solo hay una única solución.
                        \end{SmallIndentation}


                    \item
                        \textbf{Sistemas Consistentes Dependientes: Son las Mismas}

                        % ======== EXPLICACION ========
                        \begin{SmallIndentation}[1em]

                            Este caso es muy especial, pues nos dice que el sistema esta
                            dado por ecuaciones que son múltiplos de la otra o otra forma
                            de verlo es que esta dado por vectores linealmente dependientes.

                            Así que de forma numérica cuando tengamos este caso llegamos a
                            algo que siempre es verdad, a una tautología.

                            Te muestro como se ve:
                            \begin{equation*}
                            \begin{split}
                                &a_1x_1    &&+ a_2x_2    &&+ a_3x_3    &&+ \dots &&+ a_nx_n    &&= b    \\
                                &C_1a_1x_1 &&+ C_1a_2x_2 &&+ C_1a_3x_3 &&+ \dots &&+ C_1a_nx_n &&= C_1b \\
                                &\cdots                                                                 \\
                                &C_ma_1x_1 &&+ C_ma_2x_2 &&+ C_ma_3x_3 &&+ \dots &&+ C_ma_nx_n &&= C_mb \\
                            \end{split}
                            \end{equation*}

                            Si es intentas resolver esto llegarás a esto:
                            \begin{equation*}
                            \begin{split}
                                0x_1   + 0x_2 + 0x_3 + \dots + 0x_n &= 0    \\
                                0x_1   + 0x_2 + 0x_3 + \dots + 0x_n &= 0    \\
                                &\cdots                                     \\
                                0x_1   + 0x_2 + 0x_3 + \dots + 0x_n &= 0    \\
                            \end{split}
                            \end{equation*}

                            Si llega a pasar esto es que nuestro sistema tiene infinitas soluciones.

                        \end{SmallIndentation}

                \end{itemize}



            % ==================================
            % === SISTEMAS NO CONSISTENTES  ====
            % ==================================
            \clearpage
            \subsection{Sistemas No Consistentes (No Solución)}

                Estos son los feos.
                Ocurren cuando llegamos una contradicción, como este estilo:
                \begin{equation}
                \begin{split}
                    & a_{1,1}x_1 &&+ a_{1,2}x_2 &&+ a_{1,3}x_3 &&+ \dots &&+ a_{1,n}x_n &&= b_1 \\
                    & a_{2,1}x_1 &&+ a_{2,2}x_2 &&+ a_{2,3}x_3 &&+ \dots &&+ a_{2,n}x_n &&= b_2 \\
                    &\cdots                                                                     \\
                    & 0x_1       &&+ 0x_2       &&+ 0x_3       &&+ \dots &&+ 0x_n       &&= b_p \\
                    &\cdots                                                                     \\
                    & a_{m,1}x_1 &&+ a_{m,2}x_2 &&+ a_{m,3}x_3 &&+ \dots &&+ a_{m,n}x_n &&= b_m \\
                \end{split}
                \end{equation}

                Esto nos indica que no tienen solución.


























        % =====================================================
        % ==========    SISTEMAS HOMOGENEOS     ===============
        % =====================================================
        \clearpage
        \section{Sistemas Homogéneos}

            Además algo muy interesante es que siempre es es consistente, es decir
            siempre habrá mínimo una solución.

            Donde la solución mas obvia (o trivial) es en la que $a_1, a_2, \dots, a_3$
            valen CERO.



\end{document}