% ****************************************************************************************
% ************************          TAREA 1           ************************************
% ****************************************************************************************


% =======================================================
% =======         HEADER FOR DOCUMENT        ============
% =======================================================
    
    % *********   HEADERS AND FOOTERS ********
    \def\ProjectAuthorLink{https://github.com/SoyOscarRH}           %Just to keep it in line
    \def\ProjectNameLink{\ProjectAuthorLink/Proyect}                %Link to Proyect

    % *********   DOCUMENT ITSELF   **************
    \documentclass[12pt, fleqn]{article}                             %Type of docuemtn and size of font and left eq
    \usepackage[spanish]{babel}                                     %Please use spanish
    \usepackage[utf8]{inputenc}                                     %Please use spanish - UFT
    \usepackage[margin = 1.2in]{geometry}                           %Margins and Geometry pacakge
    \usepackage{ifthen}                                             %Allow simple programming
    \usepackage{hyperref}                                           %Create MetaData for a PDF and LINKS!
    \usepackage{pdfpages}                                           %Create MetaData for a PDF and LINKS!
    \hypersetup{pageanchor = false}                                 %Solve 'double page 1' warnings in build
    \setlength{\parindent}{0pt}                                     %Eliminate ugly indentation
    \author{Oscar Andrés Rosas}                                     %Who I am

    % *********   LANGUAJE    *****************
    \usepackage[T1]{fontenc}                                        %Please use spanish
    \usepackage{textcmds}                                           %Allow us to use quoutes
    \usepackage{changepage}                                         %Allow us to use identate paragraphs
    \usepackage{anyfontsize}                                        %All the sizes

    % *********   MATH AND HIS STYLE  *********
    \usepackage{ntheorem, amsmath, amssymb, amsfonts}               %All fucking math, I want all!
    \usepackage{mathrsfs, mathtools, empheq}                        %All fucking math, I want all!
    \usepackage{cancel}                                             %Negate symbol
    \usepackage{centernot}                                          %Allow me to negate a symbol
    \decimalpoint                                                   %Use decimal point

    % *********   GRAPHICS AND IMAGES *********
    \usepackage{graphicx}                                           %Allow to create graphics
    \usepackage{float}                                              %For images
    \usepackage{wrapfig}                                            %Allow to create images
    \graphicspath{ {Graphics/} }                                    %Where are the images :D

    % *********   LISTS AND TABLES ***********
    \usepackage{listings, listingsutf8}                             %We will be using code here
    \usepackage[inline]{enumitem}                                   %We will need to enumarate
    \usepackage{tasks}                                              %Horizontal lists
    \usepackage{longtable}                                          %Lets make tables awesome
    \usepackage{booktabs}                                           %Lets make tables awesome
    \usepackage{tabularx}                                           %Lets make tables awesome
    \usepackage{multirow}                                           %Lets make tables awesome
    \usepackage{multicol}                                           %Create multicolumns

    % *********   HEADERS AND FOOTERS ********
    \usepackage{fancyhdr}                                           %Lets make awesome headers/footers
    \pagestyle{fancy}                                               %Lets make awesome headers/footers
    \setlength{\headheight}{16pt}                                   %Top line
    \setlength{\parskip}{0.5em}                                     %Top line
    \renewcommand{\footrulewidth}{0.5pt}                            %Bottom line

    \lhead{                                                         %Left Header
        \hyperlink{section.\arabic{section}}                        %Make a link to the current chapter
        {\normalsize{\textsc{\nouppercase{\leftmark}}}}             %And fot it put the name
    }

    \rhead{                                                         %Right Header
        \hyperlink{section.\arabic{section}.\arabic{subsection}}    %Make a link to the current chapter
            {\footnotesize{\textsc{\nouppercase{\rightmark}}}}      %And fot it put the name
    }

    \rfoot{\textsc{\small{\hyperref[sec:Index]{Ve al Índice}}}}     %This will always be a footer  

    \fancyfoot[L]{                                                  %Algoritm for a changing footer
        \ifthenelse{\isodd{\value{page}}}                           %IF ODD PAGE:
            {\href{https://compilandoconocimiento.com/nosotros/}    %DO THIS:
                {\footnotesize                                      %Send the page
                    {\textsc{Oscar Andrés Rosas}}}}                 %Send the page
            {\href{https://compilandoconocimiento.com}              %ELSE DO THIS: 
                {\footnotesize                                      %Send the author
                    {\textsc{Algebra Lineal 1}}}}                   %Send the author
    }
    
    
    
% =======================================================
% ===================   COMMANDS    =====================
% =======================================================

    % =========================================
    % =======   NEW ENVIRONMENTS   ============
    % =========================================
    \newenvironment{Indentation}[1][0.75em]                         %Use: \begin{Inde...}[Num]...\end{Inde...}
        {\begin{adjustwidth}{#1}{}}                                 %If you dont put nothing i will use 0.75 em
        {\end{adjustwidth}}                                         %This indentate a paragraph
    \newenvironment{SmallIndentation}[1][0.75em]                    %Use: The same that we upper one, just 
        {\begin{adjustwidth}{#1}{}\begin{footnotesize}}             %footnotesize size of letter by default
        {\end{footnotesize}\end{adjustwidth}}                       %that's it

    \newenvironment{MultiLineEquation}[1]                           %Use: To create MultiLine equations
        {\begin{equation}\begin{alignedat}{#1}}                     %Use: \begin{Multi..}{Num. de Columnas}
        {\end{alignedat}\end{equation}}                             %And.. that's it!
    \newenvironment{MultiLineEquation*}[1]                          %Use: To create MultiLine equations
        {\begin{equation*}\begin{alignedat}{#1}}                    %Use: \begin{Multi..}{Num. de Columnas}
        {\end{alignedat}\end{equation*}}                            %And.. that's it!
    

    % =========================================
    % == GENERAL TEXT & SYMBOLS ENVIRONMENTS ==
    % =========================================
    
    % =====  TEXT  ======================
    \newcommand \Quote {\qq}                                        %Use: \Quote to use quotes
    \newcommand \Over {\overline}                                   %Use: \Bar to use just for short
    \newcommand \ForceNewLine {$\Space$\\}                          %Use it in theorems for example

    % =====  SPACES  ====================
    \DeclareMathOperator \Space {\quad}                             %Use: \Space for a cool mega space
    \DeclareMathOperator \MegaSpace {\quad \quad}                   %Use: \MegaSpace for a cool mega mega space
    \DeclareMathOperator \MiniSpace {\;}                            %Use: \Space for a cool mini space
    
    % =====  MATH TEXT  =================
    \newcommand \Such {\MiniSpace | \MiniSpace}                     %Use: \Such like in sets
    \newcommand \Also {\MiniSpace \text{y} \MiniSpace}              %Use: \Also so it's look cool
    \newcommand \Remember[1]{\Space\text{\scriptsize{#1}}}          %Use: \Remember so it's look cool
    
    % =====  THEOREMS  ==================
    \newtheorem{Theorem}{Teorema}[section]                          %Use: \begin{Theorem}[Name]\label{Nombre}...
    \newtheorem{Corollary}{Colorario}[Theorem]                      %Use: \begin{Corollary}[Name]\label{Nombre}...
    \newtheorem{Lemma}[Theorem]{Lemma}                              %Use: \begin{Lemma}[Name]\label{Nombre}...
    \newtheorem{Definition}{Definición}[section]                    %Use: \begin{Definition}[Name]\label{Nombre}...
    \theoremstyle{break}                                            %THEOREMS START 1 SPACE AFTER

    % =====  LOGIC  =====================
    \newcommand \lIff    {\leftrightarrow}                          %Use: \lIff for logic iff
    \newcommand \lEqual  {\MiniSpace \Leftrightarrow \MiniSpace}    %Use: \lEqual for a logic double arrow
    \newcommand \lInfire {\MiniSpace \Rightarrow \MiniSpace}        %Use: \lInfire for a logic infire
    \newcommand \lLongTo {\longrightarrow}                          %Use: \lLongTo for a long arrow

    % =====  FAMOUS SETS  ===============
    \DeclareMathOperator \Naturals     {\mathbb{N}}                 %Use: \Naturals por Notation
    \DeclareMathOperator \Primes       {\mathbb{P}}                 %Use: \Primes por Notation
    \DeclareMathOperator \Integers     {\mathbb{Z}}                 %Use: \Integers por Notation
    \DeclareMathOperator \Racionals    {\mathbb{Q}}                 %Use: \Racionals por Notation
    \DeclareMathOperator \Reals        {\mathbb{R}}                 %Use: \Reals por Notation
    \DeclareMathOperator \Complexs     {\mathbb{C}}                 %Use: \Complex por Notation
    \DeclareMathOperator \GenericField {\mathbb{F}}                 %Use: \GenericField por Notation
    \DeclareMathOperator \VectorSet    {\mathbb{V}}                 %Use: \VectorSet por Notation
    \DeclareMathOperator \SubVectorSet {\mathbb{W}}                 %Use: \SubVectorSet por Notation
    \DeclareMathOperator \Polynomials  {\mathbb{P}}                 %Use: \Polynomials por Notation
    \DeclareMathOperator \VectorSpace  {\VectorSet_{\GenericField}} %Use: \VectorSpace por Notation
    \DeclareMathOperator \LinealTransformation {\mathcal{T}}        %Use: \LinealTransformation for a cool T
    \DeclareMathOperator \LinTrans {\mathcal{T}}                    %Use: \LinTrans for a cool T


    % =====  CONTAINERS   ===============
    \newcommand{\Set}[1]    {\left\{ \; #1 \; \right\}}             %Use: \Set {Info} for INTELLIGENT space 
    \newcommand{\bigSet}[1] {\big\{  \; #1 \; \big\}}               %Use: \bigSet  {Info} for space 
    \newcommand{\BigSet}[1] {\Big\{  \; #1 \; \Big\}}               %Use: \BigSet  {Info} for space 
    \newcommand{\biggSet}[1]{\bigg\{ \; #1 \; \bigg\}}              %Use: \biggSet {Info} for space 
    \newcommand{\BiggSet}[1]{\Bigg\{ \; #1 \; \Bigg\}}              %Use: \BiggSet {Info} for space 
    
    \newcommand{\Brackets}[1]    {\left[ #1 \right]}                %Use: \Brackets {Info} for INTELLIGENT space
    \newcommand{\bigBrackets}[1] {\big[ \; #1 \; \big]}             %Use: \bigBrackets  {Info} for space 
    \newcommand{\BigBrackets}[1] {\Big[ \; #1 \; \Big]}             %Use: \BigBrackets  {Info} for space 
    \newcommand{\biggBrackets}[1]{\bigg[ \; #1 \; \bigg]}           %Use: \biggBrackets {Info} for space 
    \newcommand{\BiggBrackets}[1]{\Bigg[ \; #1 \; \Bigg]}           %Use: \BiggBrackets {Info} for space 
    
    \newcommand{\Wrap}[1]    {\left( #1 \right)}                    %Use: \Wrap {Info} for INTELLIGENT space
    \newcommand{\bigWrap}[1] {\big( \; #1 \; \big)}                 %Use: \bigBrackets  {Info} for space 
    \newcommand{\BigWrap}[1] {\Big( \; #1 \; \Big)}                 %Use: \BigBrackets  {Info} for space 
    \newcommand{\biggWrap}[1]{\bigg( \; #1 \; \bigg)}               %Use: \biggBrackets {Info} for space 
    \newcommand{\BiggWrap}[1]{\Bigg( \; #1 \; \Bigg)}               %Use: \BiggBrackets {Info} for space 
    
    \newcommand{\Generate}[1]{\left\langle #1 \right\rangle}        %Use: \Wrap {Info} for INTELLIGENT space

    % =====  BETTERS MATH COMMANDS   =====
    \newcommand{\pfrac}[2]{\Wrap{\dfrac{#1}{#2}}}                   %Use: Put fractions in parentesis

    % =========================================
    % ====   LINEAL ALGEBRA & VECTORS    ======
    % =========================================

    % ===== UNIT VECTORS  ================
    \newcommand{\hati} {\hat{\imath}}                               %Use: \hati for unit vector    
    \newcommand{\hatj} {\hat{\jmath}}                               %Use: \hatj for unit vector    
    \newcommand{\hatk} {\hat{k}}                                    %Use: \hatk for unit vector

    % ===== FN LINEAL TRANSFORMATION  ====
    \newcommand{\FnLinTrans}[1]{\mathcal{T}\Wrap{#1}}               %Use: \FnLinTrans for a cool T
    \newcommand{\VecLinTrans}[1]{\mathcal{T}\pVector{#1}}           %Use: \LinTrans for a cool T
    \newcommand{\FnLinealTransformation}[1]{\mathcal{T}\Wrap{#1}}   %Use: \FnLinealTransformation

    % ===== MAGNITUDE  ===================
    \newcommand{\abs}[1]{\left\lvert #1 \right\lvert}               %Use: \abs{expression} for |x|
    \newcommand{\Abs}[1]{\left\lVert #1 \right\lVert}               %Use: \Abs{expression} for ||x||
    \newcommand{\Mag}[1]{\left| #1 \right|}                         %Use: \Mag {Info} 
    
    \newcommand{\bVec}[1]{\mathbf{#1}}                              %Use for bold type of vector
    \newcommand{\lVec}[1]{\overrightarrow{#1}}                      %Use for a long arrow over a vector
    \newcommand{\uVec}[1]{\mathbf{\hat{#1}}}                        %Use: Unitary Vector Example: $\uVec{i}

    % ===== ALL FOR DOT PRODUCT  =========
    \makeatletter                                                   %WTF! IS THIS
    \newcommand*\dotP{\mathpalette\dotP@{.5}}                       %Use: \dotP for dot product
    \newcommand*\dotP@[2] {\mathbin {                               %WTF! IS THIS            
        \vcenter{\hbox{\scalebox{#2}{$\m@th#1\bullet$}}}}           %WTF! IS THIS
    }                                                               %WTF! IS THIS
    \makeatother                                                    %WTF! IS THIS

    % === WRAPPERS FOR COLUMN VECTOR ===
    \newcommand{\pVector}[1]                                        %Use: \pVector {Matrix Notation} use parentesis
        { \ensuremath{\begin{pmatrix}#1\end{pmatrix}} }             %Example: \pVector{a\\b\\c} or \pVector{a&b&c} 
    \newcommand{\lVector}[1]                                        %Use: \lVector {Matrix Notation} use a abs 
        { \ensuremath{\begin{vmatrix}#1\end{vmatrix}} }             %Example: \lVector{a\\b\\c} or \lVector{a&b&c} 
    \newcommand{\bVector}[1]                                        %Use: \bVector {Matrix Notation} use a brackets 
        { \ensuremath{\begin{bmatrix}#1\end{bmatrix}} }             %Example: \bVector{a\\b\\c} or \bVector{a&b&c} 
    \newcommand{\Vector}[1]                                         %Use: \Vector {Matrix Notation} no parentesis
        { \ensuremath{\begin{matrix}#1\end{matrix}} }               %Example: \Vector{a\\b\\c} or \Vector{a&b&c}

    % === MAKE MATRIX BETTER  =========
    \makeatletter                                                   %Example: \begin{matrix}[cc|c]
    \renewcommand*\env@matrix[1][*\c@MaxMatrixCols c] {             %WTF! IS THIS
        \hskip -\arraycolsep                                        %WTF! IS THIS
        \let\@ifnextchar\new@ifnextchar                             %WTF! IS THIS
        \array{#1}                                                  %WTF! IS THIS
    }                                                               %WTF! IS THIS
    \makeatother                                                    %WTF! IS THIS

    % =========================================
    % =======   FAMOUS FUNCTIONS   ============
    % =========================================

    % == TRIGONOMETRIC FUNCTIONS  ====
    \newcommand{\Cos}[1] {\cos\Wrap{#1}}                            %Simple wrappers
    \newcommand{\Sin}[1] {\sin\Wrap{#1}}                            %Simple wrappers
    \newcommand{\Tan}[1] {tan\Wrap{#1}}                             %Simple wrappers
    
    \newcommand{\Sec}[1] {sec\Wrap{#1}}                             %Simple wrappers
    \newcommand{\Csc}[1] {csc\Wrap{#1}}                             %Simple wrappers
    \newcommand{\Cot}[1] {cot\Wrap{#1}}                             %Simple wrappers

    % === COMPLEX ANALYSIS TRIG ======
    \newcommand \Cis[1]  {\Cos{#1} + i \Sin{#1}}                    %Use: \Cis for cos(x) + i sin(x)
    \newcommand \pCis[1] {\Wrap{\Cis{#1}}}                          %Use: \pCis for the same with parantesis
    \newcommand \bCis[1] {\Brackets{\Cis{#1}}}                      %Use: \bCis for the same with Brackets


    % =========================================
    % ===========     CALCULUS     ============
    % =========================================

    % ====== TRANSFORMS =============
    \newcommand{\FourierT}[1]{\mathscr{F} \left\{ #1 \right\} }     %Use: \FourierT {Funtion}
    \newcommand{\InvFourierT}[1]{\mathscr{F}^{-1}\left\{#1\right\}} %Use: \InvFourierT {Funtion}

    % ====== DERIVATIVES ============
    \newcommand \MiniDerivate[1][x] {\dfrac{d}{d #1}}               %Use: \MiniDerivate[var] for simple use [var]
    \newcommand \Derivate[2] {\dfrac{d \; #1}{d #2}}                %Use: \Derivate [f(x)][x]
    \newcommand \MiniUpperDerivate[2] {\dfrac{d^{#2}}{d#1^{#2}}}    %Mini Derivate High Orden Derivate -- [x][pow]
    \newcommand \UpperDerivate[3] {\dfrac{d^{#3} \; #1}{d#2^{#3}}}  %Complete High Orden Derivate -- [f(x)][x][pow]
    
    \newcommand \MiniPartial[1][x] {\dfrac{\partial}{\partial #1}}  %Use: \MiniDerivate for simple use [var]
    \newcommand \Partial[2] {\dfrac{\partial \; #1}{\partial #2}}   %Complete Partial Derivate -- [f(x)][x]
    \newcommand \MiniUpperPartial[2]                                %Mini Derivate High Orden Derivate -- [x][pow] 
        {\dfrac{\partial^{#2}}{\partial #1^{#2}}}                   %Mini Derivate High Orden Derivate
    \newcommand \UpperPartial[3]                                    %Complete High Orden Derivate -- [f(x)][x][pow]
        {\dfrac{\partial^{#3} \; #1}{\partial#2^{#3}}}              %Use: \UpperDerivate for simple use

    \DeclareMathOperator \Evaluate  {\Big|}                         %Use: \Evaluate por Notation

    % =========================================
    % ========    GENERAL STYLE     ===========
    % =========================================
    
    % =====  COLORS ==================
    \definecolor{RedMD}{HTML}{F44336}                               %Use: Color :D        
    \definecolor{Red100MD}{HTML}{FFCDD2}                            %Use: Color :D        
    \definecolor{Red200MD}{HTML}{EF9A9A}                            %Use: Color :D        
    \definecolor{Red300MD}{HTML}{E57373}                            %Use: Color :D        
    \definecolor{Red700MD}{HTML}{D32F2F}                            %Use: Color :D 

    \definecolor{PurpleMD}{HTML}{9C27B0}                            %Use: Color :D        
    \definecolor{Purple100MD}{HTML}{E1BEE7}                         %Use: Color :D        
    \definecolor{Purple200MD}{HTML}{EF9A9A}                         %Use: Color :D        
    \definecolor{Purple300MD}{HTML}{BA68C8}                         %Use: Color :D        
    \definecolor{Purple700MD}{HTML}{7B1FA2}                         %Use: Color :D 

    \definecolor{IndigoMD}{HTML}{3F51B5}                            %Use: Color :D        
    \definecolor{Indigo100MD}{HTML}{C5CAE9}                         %Use: Color :D        
    \definecolor{Indigo200MD}{HTML}{9FA8DA}                         %Use: Color :D        
    \definecolor{Indigo300MD}{HTML}{7986CB}                         %Use: Color :D        
    \definecolor{Indigo700MD}{HTML}{303F9F}                         %Use: Color :D 

    \definecolor{BlueMD}{HTML}{2196F3}                              %Use: Color :D        
    \definecolor{Blue100MD}{HTML}{BBDEFB}                           %Use: Color :D        
    \definecolor{Blue200MD}{HTML}{90CAF9}                           %Use: Color :D        
    \definecolor{Blue300MD}{HTML}{64B5F6}                           %Use: Color :D        
    \definecolor{Blue700MD}{HTML}{1976D2}                           %Use: Color :D        
    \definecolor{Blue900MD}{HTML}{0D47A1}                           %Use: Color :D  

    \definecolor{CyanMD}{HTML}{00BCD4}                              %Use: Color :D        
    \definecolor{Cyan100MD}{HTML}{B2EBF2}                           %Use: Color :D        
    \definecolor{Cyan200MD}{HTML}{80DEEA}                           %Use: Color :D        
    \definecolor{Cyan300MD}{HTML}{4DD0E1}                           %Use: Color :D        
    \definecolor{Cyan700MD}{HTML}{0097A7}                           %Use: Color :D        
    \definecolor{Cyan900MD}{HTML}{006064}                           %Use: Color :D 

    \definecolor{TealMD}{HTML}{009688}                              %Use: Color :D        
    \definecolor{Teal100MD}{HTML}{B2DFDB}                           %Use: Color :D        
    \definecolor{Teal200MD}{HTML}{80CBC4}                           %Use: Color :D        
    \definecolor{Teal300MD}{HTML}{4DB6AC}                           %Use: Color :D        
    \definecolor{Teal700MD}{HTML}{00796B}                           %Use: Color :D        
    \definecolor{Teal900MD}{HTML}{004D40}                           %Use: Color :D 

    \definecolor{GreenMD}{HTML}{4CAF50}                             %Use: Color :D        
    \definecolor{Green100MD}{HTML}{C8E6C9}                          %Use: Color :D        
    \definecolor{Green200MD}{HTML}{A5D6A7}                          %Use: Color :D        
    \definecolor{Green300MD}{HTML}{81C784}                          %Use: Color :D        
    \definecolor{Green700MD}{HTML}{388E3C}                          %Use: Color :D        
    \definecolor{Green900MD}{HTML}{1B5E20}                          %Use: Color :D

    \definecolor{AmberMD}{HTML}{FFC107}                             %Use: Color :D        
    \definecolor{Amber100MD}{HTML}{FFECB3}                          %Use: Color :D        
    \definecolor{Amber200MD}{HTML}{FFE082}                          %Use: Color :D        
    \definecolor{Amber300MD}{HTML}{FFD54F}                          %Use: Color :D        
    \definecolor{Amber700MD}{HTML}{FFA000}                          %Use: Color :D        
    \definecolor{Amber900MD}{HTML}{FF6F00}                          %Use: Color :D

    \definecolor{BlueGreyMD}{HTML}{607D8B}                          %Use: Color :D        
    \definecolor{BlueGrey100MD}{HTML}{CFD8DC}                       %Use: Color :D        
    \definecolor{BlueGrey200MD}{HTML}{B0BEC5}                       %Use: Color :D        
    \definecolor{BlueGrey300MD}{HTML}{90A4AE}                       %Use: Color :D        
    \definecolor{BlueGrey700MD}{HTML}{455A64}                       %Use: Color :D        
    \definecolor{BlueGrey900MD}{HTML}{263238}                       %Use: Color :D        

    \definecolor{DeepPurpleMD}{HTML}{673AB7}                        %Use: Color :D

    \newcommand{\Color}[2]{\textcolor{#1}{#2}}                      %Simple color environment
    \newenvironment{ColorText}[1]                                   %Use: \begin{ColorText}
        { \leavevmode\color{#1}\ignorespaces }                      %That's is!

    % =====  CODE EDITOR =============
    \lstdefinestyle{CompilandoStyle} {                              %This is Code Style
        backgroundcolor     = \color{BlueGrey900MD},                %Background Color  
        basicstyle          = \tiny\color{white},                   %Style of text
        commentstyle        = \color{BlueGrey200MD},                %Comment style
        stringstyle         = \color{Green300MD},                   %String style
        keywordstyle        = \color{Blue300MD},                    %keywords style
        numberstyle         = \tiny\color{TealMD},                  %Size of a number
        frame               = shadowbox,                            %Adds a frame around the code
        breakatwhitespace   = true,                                 %Style   
        breaklines          = true,                                 %Style   
        showstringspaces    = false,                                %Hate those spaces                  
        breaklines          = true,                                 %Style                   
        keepspaces          = true,                                 %Style                   
        numbers             = left,                                 %Style                   
        numbersep           = 10pt,                                 %Style 
        xleftmargin         = \parindent,                           %Style 
        tabsize             = 4,                                    %Style
        inputencoding       = utf8/latin1                           %Allow me to use special chars
    }
 
    \lstset{style = CompilandoStyle}                                %Use this style







% =====================================================
% ============        COVER PAGE       ================
% =====================================================
\begin{document}
\begin{titlepage}
    
    % ============ TITLE PAGE STYLE  ================
    \definecolor{TitlePageColor}{cmyk}{1,.60,0,.40}                 %Simple colors
    \definecolor{ColorSubtext}{cmyk}{1,.50,0,.10}                   %Simple colors
    \newgeometry{left=0.25\textwidth}                               %Defines an Offset
    \pagecolor{TitlePageColor}                                      %Make it this Color to page
    \color{white}                                                   %General things should be white

    % ===== MAKE SOME SPACE =========
    \vspace                                                         %Give some space
    \baselineskip                                                   %But we need this to up command

    % ============ NAME OF THE PROJECT  ============
    \makebox[0pt][l]{\rule{1.3\textwidth}{3pt}}                     %Make a cool line
    
    \href{https://compilandoconocimiento.com}                       %Link to project
    {\textbf{\textsc{\Huge Facultad de Ciencias - UNAM}}}\\[2.7cm]  %Name of project   

    % ============ NAME OF THE BOOK  ===============
    \href{\ProjectNameLink/LibroAlgebraLineal}                      %Link to Author
    {\fontsize{65}{78}\selectfont \textbf{1 Tarea-Examen}\\[0.5cm]  %Name of the book
    \textcolor{ColorSubtext}{\textsc{\Huge Algebra Lineal 1 }}}     %Name of the general theme
    
    \vfill                                                          %Fill the space
    
    % ============ NAME OF THE AUTHOR  =============
    \href{\ProjectAuthorLink}                                       %Link to Author
    {\LARGE \textsf{Oscar Andrés Rosas Hernandez}}                  %Author

    % ===== MAKE SOME SPACE =========
    \vspace                                                         %Give some space
    \baselineskip                                                   %But we need this to up command
    
    {\large \textsf{Febrero 2018}}                                  %Date

\end{titlepage}


% =====================================================
% ==========      RESTORE TO DOCUMENT      ============
% =====================================================
\restoregeometry                                                    %Restores the geometry
\nopagecolor                                                        %Use to restore the color to white




% =====================================================
% ========                INDICE              =========
% =====================================================
\tableofcontents{}
\label{sec:Index}

\clearpage




% ==============================================================
% =================          PROBLEMA 1       ==================
% ==============================================================
\clearpage
\section{1 Problema}

    Dado un espacio vectorial cualquiera, $\VectorSpace$, y $\vec u, \vec v \in \VectorSpace$
    entonces:\\
    $\Generate{\Set{\vec u, \vec v}} = \Generate{\Set{\vec u}} \oplus \Generate{\Set{\vec v}}$

    % ======== DEMOSTRACION ========
    \begin{SmallIndentation}[1em]
        \textbf{Contraejemplo}:
        
        Considera a $\VectorSpace = \Reals^2_{\Reals}$, entonces toma a $\vec v = (1, 0)$ y $\vec u = (2, 0)$
        entonces $<{(1, 0), (2, 0)}>$ no es una suma directa de $<\Set{(1, 0)}>$ y $<\Set{(2, 0)}>$ porque 
        $<\Set{(1, 0)}> \cap <\Set{(2, 0)}> \neq \Set{\vec 0}$ porque por ejemplo esta en ambas $(4, 0)$

    \end{SmallIndentation}

    \vspace{2em}

    Ahora podemos probar algo parecido que si es cierto:
    Dado un espacio vectorial cualquiera, $\VectorSpace$, y $\vec u, \vec v \in \VectorSpace$ tal que
    $\Set{\vec u, \vec v}$ sea linealmente independiente, entonces:\\
    $\Generate{\Set{\vec u, \vec v}} = \Generate{\Set{\vec u}} \oplus \Generate{\Set{\vec v}}$


    % ======== DEMOSTRACION ========
    \begin{SmallIndentation}[1em]
        \textbf{Demostración}:
        
        Este es un colorario muy facil de ver del \hyperref[sec:10]{\underline{Ejercicio 10, haciendo click}}
        donde:
        \begin{itemize}
            \item $\VectorSet = \Generate{\Set{\vec u, \vec v}}$
            \item $\SubVectorSet_1 = \Generate{\Set{\vec u}}$
            \item $\SubVectorSet_1 = \Generate{\Set{\vec v}}$
            \item $B_1 = \Set{\vec u}$
            \item $B_2 = \Set{\vec v}$
        \end{itemize}
    
    \end{SmallIndentation}
        


% ==============================================================
% =================          PROBLEMA 2      ==================
% ==============================================================
\clearpage
\section{2 Problema}

    Demuestre o refute que los siguientes conjuntos son espacios vectoriales
    sobre $\Reals$

    \begin{itemize}

        \item
            $\Reals$

            Ok, creo que es logico desde ahora que si que lo es, pero vamos a comprobarlo, 
            ahora como estamos usando la suma y el producto comun en los reales entonces
            por definición de los mismos son cerrados, digo $\Reals$ es un campo.

            Todas las $8$ propiedades son consecuencias directas de que los $\Reals$
            son un campo.

            \begin{SmallIndentation}[1em]

                \begin{enumerate}
                
                    \item 
                        \textbf{Ley Aditiva Asociativa:}
                        $\forall \vec{v_1}, \vec{v_2}, \vec{v_3} \in \Reals, \MiniSpace
                            (\vec{v_1} + \vec{v_2}) + \vec{v_3} = \vec{v_1} + (\vec{v_2} + \vec{v_3})$

                    \item 
                        \textbf{Ley Aditiva Conmutativa:}
                        $\forall \vec{v_1}, \vec{v_2} \in \Reals, \MiniSpace
                                \vec{v_1} + \vec{v_2} = \vec{v_2} + \vec{v_1}$

                    \item 
                        \textbf{Elemento Indentidad Aditivo:}
                        $\exists \vec{0} \in \Reals, \MiniSpace
                            \forall \vec{v} \in \Reals, \MiniSpace \vec{0} + \vec{v} = \vec{v}$

                    \item 
                        \textbf{Existen Inversos Aditivos:}
                        $\forall \vec{v} \in \Reals, \MiniSpace
                                \exists \vec{-v} \in \Reals, \MiniSpace
                                    \vec{v} + (\vec{-v}) = (\vec{-v}) + \vec{v} = \vec{0}$

                    \item 
                        \textbf{Ley Aditiva Distributiva:}
                        $\forall \alpha \in \GenericField \MiniSpace
                            \forall \vec{v_1}, \vec{v_2} \in \Reals \MiniSpace
                                \alpha \cdot (\vec{v_1} + \vec{v_2}) = 
                                    (\alpha \cdot \vec{v_1}) + (\alpha \cdot \vec{v_2})$

                    \item 
                        \textbf{Ley Multiplicativa Asociativa:}
                        $\forall \alpha, \beta \in \GenericField, \MiniSpace
                            \forall \vec{v} \in \Reals, \MiniSpace
                                \alpha \cdot (\beta \cdot \vec{v}) = (\alpha \beta) \cdot \vec{v}$

                    \item 
                        \textbf{Ley Multiplicativa Distributiva:}
                        $\forall \alpha, \beta \in \GenericField, \MiniSpace
                            \forall \vec{v} \in \Reals, \MiniSpace
                                (\alpha + \beta) \cdot \vec{v} = 
                                        (\alpha \cdot \vec{v}) + (\beta \cdot \vec{v})$

                    \item 
                        \textbf{Elemento Indentidad Multiplicativo:}
                        $\exists 1 \in \GenericField, \MiniSpace
                            \forall \vec{v} \in \Reals, \MiniSpace 1 \cdot \vec{v} = \vec{v}$

                \end{enumerate}

            \end{SmallIndentation}

        \item 

            $\Racionals^n$

            Sea $\vec x = \pVector{1_1\\\dots\\1_n}$
            y sea $\pi \in \Reals$, pero $\pi \vec x \not \in \Racionals^n$
            por lo tanto no es cerrada bajo el producto escalar.


    \end{itemize}



% ==============================================================
% =================          PROBLEMA 3       ==================
% ==============================================================
\clearpage
\section{3 Problema}

    \begin{itemize}

        \item Sea $A,B \in M_{m \times n}(\GenericField)$ entonces 
            $(A+B)^T = A^T + B^T$

            % ======== DEMOSTRACION ========
            \begin{SmallIndentation}[1em]
                \textbf{Demostración}:

                \emph{Empecemos por ver que tienen el mismo tamaño:}

                La matriz $(A+B)$ (por como la definimos a la suma) siguen estando en 
                $M_{m \times n}(\GenericField)$, por lo tanto tenemos que la transpuesta de la 
                matriz anteriormente dicha, es decir $(A+B)^T$ esta en $M_{n \times m}(\GenericField)$.

                Ahora por otro lado tenemos que $A^T, B^T \in M_{n \times m}(\GenericField)$ por la
                definición de transpuesta, ahora como definimos la suma tenemos que 
                $(A^T+B^T) \in M_{n \times m}(\GenericField)$.
                Por lo tanto tienen el mismo tamaño.

                \emph{Ahora veamos que un cualquier elemento arbitrario de ambas matrices es igual:}
                \begin{align*}
                    [(A+B)^T]_{i, j}    
                        = [A + B]_{j, i}               
                        = [A]_{j, i} + [B]_{j, i}      
                        = [A^T]_{i, j} + [B^T]_{i, j}
                        = \Brackets{A^T + B^T}_{i, j}
                \end{align*}

            \end{SmallIndentation}

        \item Sea $A \in M_{m \times n}(\GenericField)$ y $\alpha \in \GenericField$ entonces:
            $(\alpha A)^T = \alpha A^T$
            
            % ======== DEMOSTRACION ========
            \begin{SmallIndentation}[1em]
                \textbf{Demostración}:

                Es (creo) más que obvio que tendrán el mismo tamaño, por como definimos el producto
                por un escalar.

                Ahora veamos que un cualquier elemento arbitrario de ambas matrices es igual:
                \begin{equation*}
                \begin{split}
                    [(\alpha A)^T]_{i, j}    
                        = [\alpha A]_{j, i}               
                        = \alpha [A]_{j, i}
                        = \alpha [A^T]_{i, j}
                \end{split}
                \end{equation*}

            \end{SmallIndentation}

        \item
            Por los dos teoremas anteriores podemos decir que la transpuesta se parece mucho a 
            un operador lineal, me refiero a que:

            Sea $A,B \in M_{m \times n}(\GenericField)$ y $\alpha \in \GenericField$ entonces 
            $(\alpha A + \beta B)^T = \alpha(A^T) + \beta(B^T)$

            % ======== DEMOSTRACION ========
            \begin{SmallIndentation}[1em]
                \textbf{Demostración}:
                
                Es sencillo, mira:
                \begin{align*}
                    (\alpha A + \beta B)^T
                        &= (\alpha A)^T + (\beta B)^T
                            && \Remember{Por teorema anterior} \\
                        &= \alpha (A^T) + \beta (B^T)
                            && \Remember{Por teorema anterior}
                \end{align*}
            
            \end{SmallIndentation}
                

    \end{itemize}





% ==============================================================
% =================          PROBLEMA 4      ==================
% ==============================================================
\clearpage
\section{4 Problema}

    Muestre que el conjunto $\Set{\Sin{x},\Cos{x}}$ es un conjunto linealmente
    independiente del espacio $\mathcal{C}_\infty$

    % ======== DEMOSTRACION ========
    \begin{SmallIndentation}[1em]
        \textbf{Demostración}:
        
        Si ambas son linealmente independientes, entonces la unica combinación lineal
        de ellos que nos da la función cero es la solución trivial. Vamos poco a poco
        propongamos que:
        \begin{align*}
            f_0 = a \Sin{x} + b \Cos{x}
        \end{align*}

        Entonces piensa en una $x \neq 0$ o $n\pi$ con $n \in \Integers$.
        Entonces podemos decir que $\frac{\Cos{x}}{\Sin{x}} = -\frac{b}{a}$

        Ahora, sea $x = \frac{\pi}{2}$ por lo tanto $b = 0$, pero entonces
        $a \Sin{x} = 0$, entonces tenemos que $a (1) = 0$, es decir $a = 0$.

        Por lo tanto la unica combinación lineal es la trivial, por lo tnato
        son linealmente independientes.
    
    \end{SmallIndentation}
        


% ==============================================================
% =================          PROBLEMA 5       ==================
% ==============================================================
\clearpage
\section{5 Problema}

    \begin{itemize}

    \item Si $A \in M_{n}(\GenericField)$ entonces $A+A^T$ es una matriz simétrica. 

        % ======== DEMOSTRACION ========
        \begin{SmallIndentation}[1em]
            \textbf{Demostración}:
            \begin{align*}
                [A + A^T]_{i,j}  
                    &=  [A]_{i,j} + [A^T]_{i,j}     \\ 
                    &=  [A]_{i,j} + [A]_{j,i}       \\
                    &=  [A^T]_{j,i} + [A]_{j,i}     \\
                    &=  [A]_{j,i} + [A^T]_{j, i}    \\
                    &=  [A + A^T]_{j, i}             
            \end{align*}

        \end{SmallIndentation}

    \item Si $A \in M_{n}(\GenericField)$ entonces $A-A^T$ es una matriz antesimétrica. 

        % ======== DEMOSTRACION ========
        \begin{SmallIndentation}[1em]
            \textbf{Demostración}:
            \begin{align*}
                [A-A^T]_{i,j}   
                    &=  [A]_{i,j} - [A^T]_{i,j}     \\ 
                    &=  [A]_{i,j} - [A]_{j,i}       \\
                    &=  [A^T]_{j,i} - [A]_{j,i}     \\
                    &=  [-A+A^T]_{j, i}             \\
                    &= -[A - A^T]_{j, i}
            \end{align*}

        \end{SmallIndentation}

    \item Si $A \in M_{n}(\GenericField)$ es simetrica y $k \in \GenericField$ entonces 
        $KA$ también es simétrica.

        % ======== DEMOSTRACION ========
        \begin{SmallIndentation}[1em]
            \textbf{Demostración}:

            Esta esta sencilla, mira:
            \begin{align*}
                [KA]_{i, j}
                    &= K [A]_{i, j}     
                        && \Remember{por la definición de producto escalar} \\
                    &= K [A]_{j, i}
                        && \Remember{Porque $A$ es simetrica}               \\
                    &= [KA]_{j, i}
                        && \Remember{Porque definición de producto escalar}
            \end{align*}

        \end{SmallIndentation}

    \item Si $A \in M_{n}(\GenericField)$ es simetrica y $k \in \GenericField$ entonces 
        $KA$ también es antisimétrica.

        % ======== DEMOSTRACION ========
        \begin{SmallIndentation}[1em]
            \textbf{Demostración}:

            Esta esta sencilla, mira:
            \begin{align*}
                [KA]_{i, j}
                    &= K [A]_{i, j}     
                        && \Remember{por la definición de producto escalar} \\
                    &= (K)(-[A]_{j, i})
                        && \Remember{Porque $A$ es antisimétrica}           \\
                    &= -[KA]_{j, i}
                        && \Remember{Porque definición de producto escalar}
            \end{align*}

        \end{SmallIndentation}

    \end{itemize}

% ==============================================================
% =================          PROBLEMA 6       ==================
% ==============================================================
\clearpage
\section{6 Problema}

    Demuestre o refute que los siguientes conjuntos son espacios vectoriales
    sobre $\Reals$:

        \begin{itemize}
            

            \item 
                Prueba que $X = \Set{(a, b) \in \Reals^2 \Such a + 3b = 0}$, con 
                $\VectorSpace = \Reals^2$ y $\GenericField = \Reals$


                % ======== DEMOSTRACION ========
                \begin{SmallIndentation}[1em]
                    \textbf{Solución}:

                    Ok, veamos:
                    \begin{itemize}
                        
                        \item Probemos que $\vec 0 \in X$

                            Esta porque $(0, 0)$, es decir cuando $a = b = 0$ cumple que $0 + 3(0) = 0$, por
                            lo tanto $\vec 0 \in X$

                        \item
                            Veamos que sea cerrada bajo la suma:

                            Tomemos $\vec x, \vec y \in X$ donde $\vec x = (x_a, x_b)$ y $\vec y = (y_a, y_b)$
                            y como estan en $X$ tenemos que $x_a + 3x_b = 0$ y $y_a + 3y_b = 0$ entonces
                            tenemos que $\vec x + \vec y = (x_a + y_a, y_a + y_b)$

                            Y ve que:
                            \begin{align*}
                                x_a + y_a + 3(x_b + y_b)
                                    &= (x_a + 3x_b) + (y_a + 3y_b)      \\
                                    &= 0 + 0                            \\
                                    &= 0
                            \end{align*}

                            Por lo tanto $\vec x + \vec y \in X$, por lo tanto es cerrado bajo la suma

                        \item
                            Veamos que sea cerrada bajo el producto por escalar:

                            Tomemos $\vec x \in X$ y $\alpha \in \Reals$ donde $\vec x = (x_a, x_b)$ y como
                            esta en $X$ tenemos que $x_a + 3x_b = 0$ entonces tenemos que:
                            $\alpha \vec x = (\alpha x_a + \alpha x_b)$

                            Y ve que:
                            \begin{align*}
                                \alpha x_a + 3(\alpha x_b)
                                    &= \alpha (x_a + 3x_b)              \\
                                    &= \alpha (0)                       \\
                                    &= 0
                            \end{align*}

                            Por lo tanto $\alpha \vec x \in X$, por lo tanto es cerrado bajo el producto escalar

                    \end{itemize}

                \end{SmallIndentation}

            \clearpage

            \item 
                Prueba que $X = \Set{ f\in \Reals^{\Reals} \Such f(x) = -f(-x) \forall x \in \Reals}$, con 
                $\VectorSpace = \mathcal{C}_\infty$ y $\GenericField = \Reals$


                % ======== DEMOSTRACION ========
                \begin{SmallIndentation}[1em]
                    \textbf{Solución}:

                    Ok, veamos:
                    \begin{itemize}
                        
                        \item Probemos que $\vec 0 \in X$

                            Esta porque $g(x) = 0$, es decir una función que para cada real regresa el cero esta en $X$ pues
                            $g(x) = 0 = - (0) = -(g(-x))$
                            lo tanto $g \in X$

                        \item
                            Veamos que sea cerrada bajo la suma:

                            Tomemos $f, g \in X$ y un real arbitrario $x$, y que $f, g$ por estar en $X$ tenemos que
                            $f(x) = -f(-x) \forall x \in \Reals$ y $g(x) = -g(-x) \forall x \in \Reals$ 
                            entonces tenemos que:
                            \begin{align*}
                                f(x) + g(x)
                                    &= -f(-x) + -g(-x)                  \\
                                    &= - [f(-x) + g(-x)]                \\
                            \end{align*}

                            Nota que como acabamos de ver $f(x) + g(x)$ sigue en $X$ porque $f(x) + g(x) = - [f(-x) + g(-x)]$.
                            Por lo tanto es cerrado bajo la suma.

                        \item
                            Veamos que sea cerrada bajo el producto por escalar:

                            Tomemos $f \in X$ y $\alpha \in \Reals$ y un real arbitrario $x$ y que $f$ por estar en $X$
                            tenemos que $f(x) = -f(-x) \forall x \in \Reals$ entonces tenemos que:

                            Y ve que:
                            \begin{align*}
                                \alpha f(x)
                                    &= \alpha -f(-x)                    \\
                                    &= - [\alpha f(-x)]                 \\
                            \end{align*}

                            Nota que como acabamos de ver $\alpha f(x)$ sigue en $X$ porque $\alpha f(x) = - [\alpha f(-x)]$.
                            Por lo tanto es cerrado bajo el producto escalar.

                    \end{itemize}

                \end{SmallIndentation}

            \clearpage

            \item 
                Prueba que $X = \Set{(a, b) \in \Complexs^2 \Such a - b = 0}$, con 
                $\VectorSpace = \Complexs^2$ y $\GenericField = \Reals$


                % ======== DEMOSTRACION ========
                \begin{SmallIndentation}[1em]
                    \textbf{Solución}:

                    Ok, veamos:
                    \begin{itemize}
                        
                        \item Probemos que $\vec 0 \in X$

                            Esta porque $(0, 0)$, es decir cuando $a = b = 0$ cumple que $0 - (0) = 0$, por
                            lo tanto $\vec 0 \in X$

                        \item
                            Veamos que sea cerrada bajo la suma:

                            Tomemos $\vec x, \vec y \in X$ donde $\vec x = (x_a, x_b)$ y $\vec y = (y_a, y_b)$
                            y como estan en $X$ tenemos que $x_a - x_b = 0$ y $y_a - y_b = 0$ entonces
                            tenemos que $\vec x + \vec y = (x_a + y_a, y_a + y_b)$

                            Y ve que:
                            \begin{align*}
                                x_a + y_a - (x_b + y_b)
                                    &= (x_a - x_b) + (y_a - y_b)        \\
                                    &= 0 + 0                            \\
                                    &= 0
                            \end{align*}

                            Por lo tanto $\vec x + \vec y \in X$, por lo tanto es cerrado bajo la suma

                        \item
                            Veamos que sea cerrada bajo el producto por escalar:

                            Tomemos $\vec x \in X$ y $\alpha \in \Reals$ donde $\vec x = (x_a, x_b)$ y como
                            esta en $X$ tenemos que $x_a - x_b = 0$ entonces tenemos que:
                            $\alpha \vec x = (\alpha x_a + \alpha x_b)$

                            Y ve que:
                            \begin{align*}
                                \alpha x_a - (\alpha x_b)
                                    &= \alpha (x_a - x_b)               \\
                                    &= \alpha (0)                       \\
                                    &= 0
                            \end{align*}

                            Por lo tanto $\alpha \vec x \in X$, por lo tanto es cerrado bajo el producto escalar

                    \end{itemize}

                \end{SmallIndentation}

        \end{itemize}





% ==============================================================
% =================          PROBLEMA 7       ==================
% ==============================================================
\clearpage
\section{7 Problema}

    Demuestre que cualquier conjunto de k+1 vectores en un espacio de
    dimensión k es linealmente dependiente.

    \begin{itemize}
        \item
            Dado $S$ una base de $\VectorSpace$ cualquier otro conjunto linealmente
            independiente de $n$ elementos es una base de $\VectorSpace$.

            % ======== DEMOSTRACION ========
            \begin{SmallIndentation}[1em]
                \textbf{Demostración}:
                
                Usemos el teorema que dice que \Quote{
                    Dado $\VectorSpace = <G>$ donde $G = \Set{\vec u_1, \dots, \vec u_n}$ donde $G$ es base.
                    Ademas, dado a $L$ como subconjunto linealmente independiente de $\VectorSpace$
                    tal que $|L| = m$, y $m \leq n$, entonces existe otro conjunto $H$ tal que
                    $|H| = n - m$ tal que $<L \cup H> = \VectorSpace$
                }

                Entonces, supongamos un conjunto $S = \Set{\vec u_1, \dots, \vec u_n}$, entonces por ese teorema
                existe otro conjunto de $n - n$ elementos que al unirlo con $S$ puede generar a $\VectorSpace$,
                pero un conjunto de $0$ elementos es el vacío, por lo tanto $<S> = \VectorSpace$, por lo tanto
                por definición es base.
            
            \end{SmallIndentation}

        \item
            Todas las bases de $\VectorSpace$ tiene la misma cardinalidad

            % ======== DEMOSTRACION ========
            \begin{SmallIndentation}[1em]
                \textbf{Demostración}:
                
                Ok, este esta bueno, sea $S$ un conjunto de más de $n$ elementos.

                Ahora vamos a pensar que es linealmente independiente, veamos que pasa:

                Suponte un subconjunto de $S$, llamada $miniS$ que tenga ahora si $n$ elementos, 
                además como supusimos que $S$ es linealmente independiente, entonces todos sus subconjuntos
                en especial $miniS$ también es linealmente independiente.
                Ahora bien por el teorema anterior tenemos que cualquier conjunto linealmente independiente
                de $n$ elementos es una base, por lo tanto $miniS$ es base. 
                Entonces recuerda que podremos escribir a todos los elementos de $\VectorSpace$ como combinación
                lineal de $miniS$, eso incluye a todos los elementos de $S - miniS$, por lo tanto $S$ no puede ser
                linealmente independiente, pero dijimos que si, es decir contradicción.

                Si algun subconjunto del espacio vectorial tiene mas de $n$ elementos entonces no puede ser linealmente
                independiente y por lo tanto no puede ser base. 

                Ahora bien si $S$ tiene menos de $n$ elementos, digamos que tiene $m$ elementos, entonces
                tampoco puede ser base pues por el teorema:
                \Quote{
                    Dado $\VectorSpace = <G>$ donde $G = \Set{\vec u_1, \dots, \vec u_n}$ donde $G$ es base.
                    Ademas, dado a $L$ como subconjunto linealmente independiente de $\VectorSpace$
                    tal que $|L| = m$, y $m \leq n$, entonces existe otro conjunto $H$ tal que
                    $|H| = n - m$ tal que $<L \cup H> = \VectorSpace$
                }
                necesitamos agregarle otro conjunto de $n - m$ elementos para que pueda generar a $\VectorSpace$
                entonces tampoco puede ser base.
            
            \end{SmallIndentation}

        \item
            Cualquier conjunto de n+1 vectores en un espacio de
            dimensión n es linealmente dependiente. 

            % ======== DEMOSTRACION ========
            \begin{SmallIndentation}[1em]
                \textbf{Demostración}:
                
                Suponte que no, que encontramos un conjunto $S \subseteq \VectorSpace$ donde $dim(\VectorSpace) = n$
                que tenga $n + 1$ elementos y supongamos que sea linealmente independiente. 

                Suponte un subconjunto de $S$, llamada $miniS$ que tenga ahora si $n$ elementos, 
                además como supusimos que $S$ es linealmente independiente, entonces todos sus subconjuntos
                en especial $miniS$ también lo son, por lo tanto $miniS$ es linealmente independiente.

                Ahora bien por el teorema anterior tenemos que cualquier conjunto linealmente independiente
                de $n$ elementos es una base, por lo tanto $miniS$ es base.
                Entonces recuerda que podremos escribir a todos los elementos de $\VectorSpace$ como combinación
                lineal de $miniS$, eso incluye al elemento que esta en $S$ pero no en $miniS$, es decir que esta en
                $S - miniS$, por lo tanto (y gracias a un teorema anterior) $S$ no puede ser linealmente independiente
                porque podemos escribir a uno de sus elementos como combinación lineal de otros, pero dijimos que si era
                linealmente independiente, es decir contradicción.

                Podemos ahora generalizar un poco el resultado y ver que hemos dicho también que sin importar que elemento
                añadas a un subconjunto linealmente dependiente, este será linealmente dependiente, por lo tanto de manera
                general tenemos que:
                Si algun subconjunto del espacio vectorial tiene mas de $n$ elementos entonces no puede ser linealmente
                independiente.
            
            \end{SmallIndentation}

    \end{itemize}



% ==============================================================
% =================          PROBLEMA 8       ==================
% ==============================================================
\clearpage
\section{8 Problema}

    Demuestre que un conjunto de vectores S es linealmente independiente
    syss cada subconjunto finito de S es linealmente independiente
    
    \begin{itemize}
                    
        \item 
            Si $\vec 0 \in S$ entonces $S$ es linealmente dependiente

            % ======== DEMOSTRACION ========
            \begin{SmallIndentation}[1em]
                \textbf{Demostración}:
                
                Esta es muy fácil, considera el conjunto $\Set{\vec 0}$ entonces lo puedes
                escribir como $\vec 0 = a \vec 0$ con $a \neq 0$ entonces ya encontraste 
                una combinación lineal no trivial, y como demostraré en los siguientes temas
                veré que sin importar que le añada a un conjunto linealmente dependiente este
                seguirá siendo linealmente dependiente.
            
            \end{SmallIndentation}
                

        \item 
            Sin importar que le añada a un conjunto linealmente dependiente este
            seguirá siendo linealmente dependiente.

            Es decir:
            Si $S_1 \subseteq S_2$ y $S_1$ es linealmente dependiente entonces
            $S_2$ también es linealmente dependiente

            % ======== DEMOSTRACION ========
            \begin{SmallIndentation}[1em]
                \textbf{Idea de la Demostración}:
                
                Considera que como $S_1$ sin perdida de generalidad decimos que 
                $S_1 = \Set{\vec v_1, \dots, \vec v_n}$ es l. d. Entonces existe una combinación lineal
                tal que $\vec 0 = \sum_{i=1}^n a_i \vec v_i$, donde minímo un $a_0$ no es cero.

                Decimos que $S_2 - S_1 = \Set{\vec u_1, \dots, \vec u_k}$

                Entonces decimos que:
                $\vec 0 = \sum_{i=0}^n a_i \vec v_i +  \sum_{i=0}^k 0 \vec u_i$, bingo, una combinación
                no trivial, es un conjunto linealmente dependiente.
            
            \end{SmallIndentation}
                

        \item 
            Sin importar que le elimine a un conjunto linealmente independiente este
            seguirá siendo linealmente independiente.

            Es decir:
            Si $S_1 \subseteq S_2$ y $S_2$ es linealmente independiente entonces
            $S_1$ también es linealmente independiente

            % ======== DEMOSTRACION ========
            \begin{SmallIndentation}[1em]
                \textbf{Idea Demostración}:

                Considera que como $S_2$ sin perdida de generalidad decimos que 
                $S_2 = \Set{\vec v_1, \dots, \vec v_n}$ es l. i.

                Ahora pensemos en un subconjunto propio de $S_2$ llamado $S_1$. 
                Vamos a suponer que ese subconjunto es linealmente dependiente,
                entonces por el teorema pasado:
                \Quote{
                    Si $S_1 \subseteq S_2$ y $S_1$ es linealmente dependiente entonces
                    $S_2$ también es linealmente dependiente
                }
                pero espera, sabemos por hipotesis que $S_2$ es linealmente independiente
                por lo tanto llegamos a una contradicción si suponemos que $S_1$ es linealmente
                dependiente, por lo tanto solo le queda una opción, ser linealmente independiente

            \end{SmallIndentation}

        \clearpage

        \item
            Si cada subconjunto finito de $S$ es linealmente independiente, entonces $S$ es independiente.

            % ======== DEMOSTRACION ========
            \begin{SmallIndentation}[1em]
                \textbf{Demostración}:
                
                Probemos por contrapositiva, es decir, vamos mejor a probar que si $S$ no es linealmente independiente
                es decir, si $S$ es linealmente dependiente entonces no cada subconjunto finito de $S$ es linealmente
                independiente.

                Es decir, basta ver que si $S$ es linealmente dependiente, existe un subconjunto finito que es linealmente
                dependiente.

                $\dots$, esto va a estar feo.

                Considera $\vec x \in S$, ahora, si $\vec x = \vec 0$ ya acabamos porque $\Set{\vec 0}$ es linealmenete dependiente
                entonces por otro teorema anterior sin importar que le añada todo superconjunto de $S$ es linealmente dependiente
                incluyendo a $S$.

                Ahora, si $\vec x \neq \vec 0$ entonces $S' = \Set{\vec x}$ es linealmente independiente, ahora vamos a empezar
                a añadir cada uno de los elementos de $S$ a $S'$ hasta que el añadir a otro elemento nos oblige a que $S'$ 
                sea linealmente dependiente. Ahora, si podemos tomar todos los elementos de $S$ antes de que eso pase, entonces
                $S$ es Linealmente independiente, contradicción, por lo tanto tenemos acabar antes de tomar a todos los elementos
                de $S$, ahora, lo que nos hemos creado es un subconjunto de $S$ que es linealmente dependiente, y ya, sin importar
                que le agreges, seguirá siendo linealmente dependiente.
            
            \end{SmallIndentation}

    \end{itemize}



% ==============================================================
% =================          PROBLEMA 9       ==================
% ==============================================================
\clearpage
\section{9 Problema}

    Dados $\VectorSpace$ y $\vec u, \vec v \in \VectorSet$. Si $\Set{ \vec u, \vec v}$
    es base de $\VectorSpace$, entonces para cada $a, b \in \GenericField - \Set{\vec 0}$, $\Set{a\vec u, b \vec v}$
    también lo es.

    % ======== DEMOSTRACION ========
    \begin{SmallIndentation}[1em]
        \textbf{Demostración}:
        
        Ok, sabemos que $\Set{ \vec u, \vec v}$ es base de $\VectorSpace$, por lo tanto, ese pequeño conjunto
        cumple que $\Set{ \vec u, \vec v}$ es linealmente independiente y que genera a $\VectorSpace$.

        Ahora veamos que pasa para algunas $a, b$ arbitrarias (pero que no sean cero) con este conjunto: 
        $\Set{a \vec u, b \vec v}$, veamos si es linealmente dependiente, es decir existe una combinación lineal
        no trivial que te da el cero vector.
        Es decir si existe $k_1, k_2$ con alguno mínimo diferente de cero para los cuales la ecuación 
        $\vec 0 = k_1 a \vec u + k_2 b \vec v$  tiene solución.

        Ahora, sabemos que $\Set{\vec u, \vec u}$ es linealmente independiente por hipotesis, por lo tanto
        podemos decir que la ecuación $\vec 0 = q_1 \vec u + q_2 \vec v$ implica que $q_1 = q_2 = 0$.

        Entonces ve que por lo anterior $\vec 0 = (k_1 a) \vec u + (k_2 b) \vec v$ nos obliga a que $k_1a$ y $k_2b$ 
        sean cero, pero es que $a, b$ no pueden ser cero por hipotesis, por lo tanto tenemos que ve que $k_1, k_2$ son cero.

        Por lo tanto $\Set{a \vec u, b \vec v}$ es linealmente independiente.

        Ahora, veamos que $\Set{\vec u, \vec u}$ sigue generando a $\VectorSet$.

        A ver, por un lado tenemos que $\Set{\vec u, \vec v}$ genera a $\VectorSpace$, es decir
        $\forall \vec x \in \VectorSpace$ podemos decir que $\vec x = k_1 \vec u + k_2 \vec v$

        Ahora hagamos magia:
        \begin{align*}
            \vec x  
                = k_1 \vec u + k_2 \vec v  
                    &&\Remember{ por lo de arriba}                                                      \\
                = \frac{k_1}{a} a \vec u + \frac{k_2}{b} b\vec v
                    &&\Remember{Podemos dividir porque ni a ni b son cero}                              \\
                = k_1' a \vec u + k_2' b\vec v
                    &&\Remember{Mira, lo pude escribir como combinacion lineal de $a\vec u, b \vec v$}   
        \end{align*}

        Mira, como aun puedo escribir cualquier elemento de $\VectorSpace$ como combinación lineal de los
        elementos de $\Set{a \vec u, b \vec v}$. Por lo tanto sigue generando a $\VectorSpace$.

        Por lo tanto $\Set{a \vec u, b \vec v}$ es base. 
    
    \end{SmallIndentation}



% ==============================================================
% =================          PROBLEMA 10       =================
% ==============================================================
\clearpage
\section{10 Problema}
\label{sec:10}

    Sean $B_1, B_2$ dos bases ajenas de dos subespacios vectoriales $\SubVectorSet_1, \SubVectorSet_2$
    de $\VectorSpace$, entonces si $B_1 \cup B_2$ es base de $\VectorSpace$ entonces 
    $\SubVectorSet_1 \oplus \SubVectorSet_2 = \VectorSpace$

    % ======== DEMOSTRACION ========
    \begin{SmallIndentation}[1em]
        \textbf{Demostración}:
        
        Ok, este teorema parece tener mucho sentido, veamos porque:
        Por un lado si $B_1 \cup B_2$ es base de $\VectorSpace$ entonces vemos que 
        $B_1 \cup B_2$ es linealmente independiente. Ahora por ser bases ellas tienen que ser 
        linealmente independientes, ahora, además nos dicen que son bases ajenas, es decir
        que no tienen elementos en común.

        Sin perdida de generalidad tenemos que $B_1 = \Set{\vec v_1, \dots, \vec v_n}$ y 
        $B_2 = \Set{\vec u_1, \dots, \vec u_m}$ y que $dim(\VectorSpace) = n + m$

        Ahora probemos las 3 propiedades para ver que ambos subespacios son una suma directa:
        \begin{itemize}
            \item 
                $\SubVectorSet_1, \SubVectorSet_2$ son subespacios vectoriales de $\VectorSpace$

                Por hipotesis tanto $\SubVectorSet_1$ como $\SubVectorSet_2$ son subespacios de 
                $\VectorSpace$.

            \item
                $\SubVectorSet_1 \cap \SubVectorSet_2 = \Set{\vec 0}$

                Ok, para demostar eso, tomemos a $\vec x \in \SubVectorSet_1 \cap \SubVectorSet_2$
                ahora a fin de cuentas $\vec x \in \VectorSpace$ por lo tanto como sabemos que $B_1 \cup B_2$
                es base tenemos por un teorema anterior que un conjunto es linealmente independiente si y solo si solo hay
                una manera de escribir a cada elemento de su generado como combinación lineal, es
                decir $\vec x = \sum_{i=0}^{n} c_i \vec v_i + \sum_{i=0}^{m} c_i \vec u_i$

                Ahora, como $\vec x \in \SubVectorSet_1$ entonces $\vec x = \sum_{i=1}^n a_i \vec v_i$
                y como $\vec x \in \SubVectorSet_2$ entonces $\vec x = \sum_{i=1}^m b_i \vec u_i$

                Ahora, si te das cuenta parece que tenemos a 3 maneras distintas de escribir a $\vec x$ como
                combinación lineal de elementos de su base, pero sabemos que dicha combinación lineal debe ser unica por
                hipotesis de que $B_1 \cup B_2$ es base, por lo tanto solo nos queda que $c_i = a_i = b_i = 0$
                por lo tanto $\vec x = \vec 0$.

                Por lo tanto acabamos de demostrar que si tomamos algún elemento de la intersección de dichos subespacios
                este tiene que ser el $\vec 0$.

            \item
                Probemos finalmente que $\SubVectorSet_1 + \SubVectorSet_2 = \VectorSet$.

                Ahora hagamos esto por doble contención, por un lado sea
                $\vec x \in \SubVectorSet_1 + \SubVectorSet_2$ entonces
                $\vec x = \vec x_1 + \vec x_2$ con $\vec x_1 \in \SubVectorSet_1$ y $\vec x_2 \in \SubVectorSet_2$,
                como estamos hablando de espacios vectoriales, tienen que ser cerrado bajo la suma, por lo tanto
                $\vec x \in \VectorSet$.

                Ahora tomemos a un elemento $\vec y \in \VectorSet$, entonces se puede expresar como combinación
                lineal de la base que es $B_1 \cup B_2$, es decir 
                $\vec y = \sum_{i=0}^{n} c_i \vec v_i + \sum_{i=0}^{m} c_i \vec u_i$

                Ahora, podemos reacomodar esto y ver que $\vec y_1 = \sum_{i=}^{n} c_i \vec v_i$ y
                $\vec y_2 = \sum_{i=1}^{m} c_i \vec u_i$.
                Ahora creo que es obvio que $y_1 \in \SubVectorSet_1$ y $y_2 \in \SubVectorSet_2$
                por lo tanto hemos podido escribir a un elemento arbitrario de $\VectorSet$ como 
                suma de dos elementos $\vec y = \vec y_1 + \vec y_2$. Por lo tanto ambos conjuntos son iguales 
                $\SubVectorSet_1 + \SubVectorSet_2 = \VectorSet$.

                Por lo tanto la suma de dichos espacios, $\SubVectorSet_1, \SubVectorSet_2$ si es $\VectorSet$                                   

        \end{itemize}


    \end{SmallIndentation}




% ==============================================================
% =================          PROBLEMA 11       =================
% ==============================================================
\clearpage
\section{11 Problema}

    Si $\Set{\vec u, \vec v, \vec w}$ son base de $\VectorSpace$ entonces 
    $\Set{\vec v + \vec u + \vec w, \vec u + \vec w, \vec w}$

    % ======== DEMOSTRACION ========
    \begin{SmallIndentation}[1em]
        \textbf{Demostración}:
    
        Esta demostración se ve buena, tomemos
        la ecuación:
        \begin{align*}
            \vec 0 
                &= a_1 (v + \vec u + \vec w) + a_2(\vec u + \vec w) + a_3 \vec w        \\
                &= (a_1)\vec v + (a_1 + a_2)\vec u  + (a_1 + a_2 + a_3)\vec w 
        \end{align*}

        Ahora como $\Set{\vec u, \vec v, \vec w}$ es base, es linealmente independiente
        por lo tanto la unica combinación lineal que nos da el $\vec 0$ es la trivial
        por lo tanto $\vec 0 = (a_1)\vec v + (a_1 + a_2)\vec u  + (a_1 + a_2 + a_3)\vec w$
        nos obliga a que $a_1 = a_1 + a_2 = a_1 + a_2 + a_3 = 0$, por lo tanto
        todos son cero, por lo tanto $\Set{\vec v + \vec u + \vec w, \vec u + \vec w, \vec w}$
        sigue siendo linealmente independiente y sabemos por hipotesis que si 
        $\Set{\vec u, \vec v, \vec w}$ son base, entonces la dimensión del espacio vectorial
        es 3, por lo tanto cualquier conjunto linealmente independiente de 3 elementos
        es base, en este caso $\Set{\vec v + \vec u + \vec w, \vec u + \vec w, \vec w}$

    \end{SmallIndentation}
                         

% ==============================================================
% =================          PROBLEMA 12       =================
% ==============================================================
\vspace{1em}
\section{12 Problema}

    ¿El conjunto $\Set{(1, 1, 1, 1), (1, 1, 1, 0), (1, 1, 0, 0), (1, 0, 0, 0)}$ es una base
    para $\GenericField^4$ con $\GenericField$ siendo un campo cualquiera?.

    Si lo es, entonces encontremos la representación de $(a_1, a_2, a_3, a_4)$ como combinación
    lineal del primer conjunto.

    % ======== DEMOSTRACION ========
    \begin{SmallIndentation}[1em]
        \textbf{Solución}:

        Veamos si es primero base, para eso veamos que tiene cuatro elementos por lo tanto
        solo nos falta por ver si es linealmente independiente:
        \begin{align*}
            (0, 0, 0, 0)
                &= k_1(1, 1, 1, 1) + k_2(1, 1, 1, 0) + k_3(1, 1, 0, 0) + k_4(1, 0, 0, 0)    \\
                &= (k_1+k_2+k_3+k_4, k_1+k_2+k_3, k_1+k_2, k_1)                             \\
        \end{align*}

        Por lo tanto tenemos que $k_1 = 0$, ahora sabemos tambien que $k_1+k_2 = 0$, por lo tanto $k_2$
        es $0$, ahora $k_1+k_2+k_3 = 0$, por lo tanto $k_3=0 $y finalmente $k_1+k_2+k_3+k_4 = 0$, por
        lo tanto $k_4 = 0$

        Por lo tanto es linealmente independiente, pero como también tiene 4 elementos, podemos concluir que 
        es una base de $\GenericField^4$.

        Ahora podemos ver que:
        \begin{align*}
            (a_1, a_2, a_3, a_4)
                &= 
                    + c_1 (1, 1, 1, 1)
                    + c_2 (1, 1, 1, 0)
                    + c_3 (1, 1, 0, 0)
                    + c_4 (1, 0, 0, 0)                                  \\
                &= 
                    (c_1+c_2+c_3+c_4, c_1+c_2+c_3, c_1+c_2, c_1)        \\
                &= 
                    + [a_4]       (1, 1, 1, 1)
                    + [a_3 - a_4] (1, 1, 1, 0)
                    + [a_2 - a_3] (1, 1, 0, 0)
                    + [a_1 - a_2] (1, 0, 0, 0)
        \end{align*}

    \end{SmallIndentation}





\end{document}

