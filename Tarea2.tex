% ****************************************************************************************
% ************************          TAREA 1           ************************************
% ****************************************************************************************


% =======================================================
% =======         HEADER FOR DOCUMENT        ============
% =======================================================
    
    % *********   HEADERS AND FOOTERS ********
    \def\ProjectAuthorLink{https://github.com/SoyOscarRH}           %Just to keep it in line
    \def\ProjectNameLink{\ProjectAuthorLink/Proyect}                %Link to Proyect

    % *********   DOCUMENT ITSELF   **************
    \documentclass[12pt, fleqn]{article}                             %Type of docuemtn and size of font and left eq
    \usepackage[spanish]{babel}                                     %Please use spanish
    \usepackage[utf8]{inputenc}                                     %Please use spanish - UFT
    \usepackage[margin = 1.2in]{geometry}                           %Margins and Geometry pacakge
    \usepackage{ifthen}                                             %Allow simple programming
    \usepackage{hyperref}                                           %Create MetaData for a PDF and LINKS!
    \usepackage{pdfpages}                                           %Create MetaData for a PDF and LINKS!
    \hypersetup{pageanchor = false}                                 %Solve 'double page 1' warnings in build
    \setlength{\parindent}{0pt}                                     %Eliminate ugly indentation
    \author{Oscar Andrés Rosas}                                     %Who I am

    % *********   LANGUAJE    *****************
    \usepackage[T1]{fontenc}                                        %Please use spanish
    \usepackage{textcmds}                                           %Allow us to use quoutes
    \usepackage{changepage}                                         %Allow us to use identate paragraphs
    \usepackage{anyfontsize}                                        %All the sizes

    % *********   MATH AND HIS STYLE  *********
    \usepackage{ntheorem, amsmath, amssymb, amsfonts}               %All fucking math, I want all!
    \usepackage{mathrsfs, mathtools, empheq}                        %All fucking math, I want all!
    \usepackage{cancel}                                             %Negate symbol
    \usepackage{centernot}                                          %Allow me to negate a symbol
    \decimalpoint                                                   %Use decimal point

    % *********   GRAPHICS AND IMAGES *********
    \usepackage{graphicx}                                           %Allow to create graphics
    \usepackage{float}                                              %For images
    \usepackage{wrapfig}                                            %Allow to create images
    \graphicspath{ {Graphics/} }                                    %Where are the images :D

    % *********   LISTS AND TABLES ***********
    \usepackage{listings, listingsutf8}                             %We will be using code here
    \usepackage[inline]{enumitem}                                   %We will need to enumarate
    \usepackage{tasks}                                              %Horizontal lists
    \usepackage{longtable}                                          %Lets make tables awesome
    \usepackage{booktabs}                                           %Lets make tables awesome
    \usepackage{tabularx}                                           %Lets make tables awesome
    \usepackage{multirow}                                           %Lets make tables awesome
    \usepackage{multicol}                                           %Create multicolumns

    % *********   HEADERS AND FOOTERS ********
    \usepackage{fancyhdr}                                           %Lets make awesome headers/footers
    \pagestyle{fancy}                                               %Lets make awesome headers/footers
    \setlength{\headheight}{16pt}                                   %Top line
    \setlength{\parskip}{0.5em}                                     %Top line
    \renewcommand{\footrulewidth}{0.5pt}                            %Bottom line

    \lhead{                                                         %Left Header
        \hyperlink{section.\arabic{section}}                        %Make a link to the current chapter
        {\normalsize{\textsc{\nouppercase{\leftmark}}}}             %And fot it put the name
    }

    \rhead{                                                         %Right Header
        \hyperlink{section.\arabic{section}.\arabic{subsection}}    %Make a link to the current chapter
            {\footnotesize{\textsc{\nouppercase{\rightmark}}}}      %And fot it put the name
    }

    \rfoot{\textsc{\small{\hyperref[sec:Index]{Ve al Índice}}}}     %This will always be a footer  

    \fancyfoot[L]{                                                  %Algoritm for a changing footer
        \ifthenelse{\isodd{\value{page}}}                           %IF ODD PAGE:
            {\href{https://compilandoconocimiento.com/nosotros/}    %DO THIS:
                {\footnotesize                                      %Send the page
                    {\textsc{Oscar Andrés Rosas}}}}                 %Send the page
            {\href{https://compilandoconocimiento.com}              %ELSE DO THIS: 
                {\footnotesize                                      %Send the author
                    {\textsc{Algebra Lineal 1}}}}                   %Send the author
    }
    
    
    
% =======================================================
% ===================   COMMANDS    =====================
% =======================================================

    % =========================================
    % =======   NEW ENVIRONMENTS   ============
    % =========================================
    \newenvironment{Indentation}[1][0.75em]                         %Use: \begin{Inde...}[Num]...\end{Inde...}
        {\begin{adjustwidth}{#1}{}}                                 %If you dont put nothing i will use 0.75 em
        {\end{adjustwidth}}                                         %This indentate a paragraph
    \newenvironment{SmallIndentation}[1][0.75em]                    %Use: The same that we upper one, just 
        {\begin{adjustwidth}{#1}{}\begin{footnotesize}}             %footnotesize size of letter by default
        {\end{footnotesize}\end{adjustwidth}}                       %that's it

    \newenvironment{MultiLineEquation}[1]                           %Use: To create MultiLine equations
        {\begin{equation}\begin{alignedat}{#1}}                     %Use: \begin{Multi..}{Num. de Columnas}
        {\end{alignedat}\end{equation}}                             %And.. that's it!
    \newenvironment{MultiLineEquation*}[1]                          %Use: To create MultiLine equations
        {\begin{equation*}\begin{alignedat}{#1}}                    %Use: \begin{Multi..}{Num. de Columnas}
        {\end{alignedat}\end{equation*}}                            %And.. that's it!
    

    % =========================================
    % == GENERAL TEXT & SYMBOLS ENVIRONMENTS ==
    % =========================================
    
    % =====  TEXT  ======================
    \newcommand \Quote {\qq}                                        %Use: \Quote to use quotes
    \newcommand \Over {\overline}                                   %Use: \Bar to use just for short
    \newcommand \ForceNewLine {$\Space$\\}                          %Use it in theorems for example

    % =====  SPACES  ====================
    \DeclareMathOperator \Space {\quad}                             %Use: \Space for a cool mega space
    \DeclareMathOperator \MegaSpace {\quad \quad}                   %Use: \MegaSpace for a cool mega mega space
    \DeclareMathOperator \MiniSpace {\;}                            %Use: \Space for a cool mini space
    
    % =====  MATH TEXT  =================
    \newcommand \Such {\MiniSpace | \MiniSpace}                     %Use: \Such like in sets
    \newcommand \Also {\MiniSpace \text{y} \MiniSpace}              %Use: \Also so it's look cool
    \newcommand \Remember[1]{\Space\text{\scriptsize{#1}}}          %Use: \Remember so it's look cool
    
    % =====  THEOREMS  ==================
    \newtheorem{Theorem}{Teorema}[section]                          %Use: \begin{Theorem}[Name]\label{Nombre}...
    \newtheorem{Corollary}{Colorario}[Theorem]                      %Use: \begin{Corollary}[Name]\label{Nombre}...
    \newtheorem{Lemma}[Theorem]{Lemma}                              %Use: \begin{Lemma}[Name]\label{Nombre}...
    \newtheorem{Definition}{Definición}[section]                    %Use: \begin{Definition}[Name]\label{Nombre}...
    \theoremstyle{break}                                            %THEOREMS START 1 SPACE AFTER

    % =====  LOGIC  =====================
    \newcommand \lIff    {\leftrightarrow}                          %Use: \lIff for logic iff
    \newcommand \lEqual  {\MiniSpace \Leftrightarrow \MiniSpace}    %Use: \lEqual for a logic double arrow
    \newcommand \lInfire {\MiniSpace \Rightarrow \MiniSpace}        %Use: \lInfire for a logic infire
    \newcommand \lLongTo {\longrightarrow}                          %Use: \lLongTo for a long arrow

    % =====  FAMOUS SETS  ===============
    \DeclareMathOperator \Naturals     {\mathbb{N}}                 %Use: \Naturals por Notation
    \DeclareMathOperator \Primes       {\mathbb{P}}                 %Use: \Primes por Notation
    \DeclareMathOperator \Integers     {\mathbb{Z}}                 %Use: \Integers por Notation
    \DeclareMathOperator \Racionals    {\mathbb{Q}}                 %Use: \Racionals por Notation
    \DeclareMathOperator \Reals        {\mathbb{R}}                 %Use: \Reals por Notation
    \DeclareMathOperator \Complexs     {\mathbb{C}}                 %Use: \Complex por Notation
    \DeclareMathOperator \GenericField {\mathbb{F}}                 %Use: \GenericField por Notation
    \DeclareMathOperator \VectorSet    {\mathbb{V}}                 %Use: \VectorSet por Notation
    \DeclareMathOperator \SubVectorSet {\mathbb{W}}                 %Use: \SubVectorSet por Notation
    \DeclareMathOperator \Polynomials  {\mathbb{P}}                 %Use: \Polynomials por Notation
    \DeclareMathOperator \VectorSpace  {\VectorSet_{\GenericField}} %Use: \VectorSpace por Notation
    \DeclareMathOperator \LinealTransformation {\mathcal{T}}        %Use: \LinealTransformation for a cool T
    \DeclareMathOperator \LinTrans {\mathcal{T}}                    %Use: \LinTrans for a cool T
    \DeclareMathOperator \Laplace {\mathcal{L}}                     %Use: \LinTrans for a cool T


    % =====  CONTAINERS   ===============
    \newcommand{\Set}[1]    {\left\{ \; #1 \; \right\}}             %Use: \Set {Info} for INTELLIGENT space 
    \newcommand{\bigSet}[1] {\big\{  \; #1 \; \big\}}               %Use: \bigSet  {Info} for space 
    \newcommand{\BigSet}[1] {\Big\{  \; #1 \; \Big\}}               %Use: \BigSet  {Info} for space 
    \newcommand{\biggSet}[1]{\bigg\{ \; #1 \; \bigg\}}              %Use: \biggSet {Info} for space 
    \newcommand{\BiggSet}[1]{\Bigg\{ \; #1 \; \Bigg\}}              %Use: \BiggSet {Info} for space 
    
    \newcommand{\Brackets}[1]    {\left[ #1 \right]}                %Use: \Brackets {Info} for INTELLIGENT space
    \newcommand{\bigBrackets}[1] {\big[ \; #1 \; \big]}             %Use: \bigBrackets  {Info} for space 
    \newcommand{\BigBrackets}[1] {\Big[ \; #1 \; \Big]}             %Use: \BigBrackets  {Info} for space 
    \newcommand{\biggBrackets}[1]{\bigg[ \; #1 \; \bigg]}           %Use: \biggBrackets {Info} for space 
    \newcommand{\BiggBrackets}[1]{\Bigg[ \; #1 \; \Bigg]}           %Use: \BiggBrackets {Info} for space 
    
    \newcommand{\Wrap}[1]    {\left( #1 \right)}                    %Use: \Wrap {Info} for INTELLIGENT space
    \newcommand{\bigWrap}[1] {\big( \; #1 \; \big)}                 %Use: \bigBrackets  {Info} for space 
    \newcommand{\BigWrap}[1] {\Big( \; #1 \; \Big)}                 %Use: \BigBrackets  {Info} for space 
    \newcommand{\biggWrap}[1]{\bigg( \; #1 \; \bigg)}               %Use: \biggBrackets {Info} for space 
    \newcommand{\BiggWrap}[1]{\Bigg( \; #1 \; \Bigg)}               %Use: \BiggBrackets {Info} for space 
    
    \newcommand{\Generate}[1]{\left\langle #1 \right\rangle}        %Use: \Wrap {Info} for INTELLIGENT space

    % =====  BETTERS MATH COMMANDS   =====
    \newcommand{\pfrac}[2]{\Wrap{\dfrac{#1}{#2}}}                   %Use: Put fractions in parentesis

    % =========================================
    % ====   LINEAL ALGEBRA & VECTORS    ======
    % =========================================

    % ===== UNIT VECTORS  ================
    \newcommand{\hati} {\hat{\imath}}                               %Use: \hati for unit vector    
    \newcommand{\hatj} {\hat{\jmath}}                               %Use: \hatj for unit vector    
    \newcommand{\hatk} {\hat{k}}                                    %Use: \hatk for unit vector

    % ===== FN LINEAL TRANSFORMATION  ====
    \newcommand{\FnLinTrans}[1]{\mathcal{T}\Wrap{#1}}               %Use: \FnLinTrans for a cool T
    \newcommand{\VecLinTrans}[1]{\mathcal{T}\pVector{#1}}           %Use: \LinTrans for a cool T
    \newcommand{\FnLinealTransformation}[1]{\mathcal{T}\Wrap{#1}}   %Use: \FnLinealTransformation

    % ===== MAGNITUDE  ===================
    \newcommand{\abs}[1]{\left\lvert #1 \right\lvert}               %Use: \abs{expression} for |x|
    \newcommand{\Abs}[1]{\left\lVert #1 \right\lVert}               %Use: \Abs{expression} for ||x||
    \newcommand{\Mag}[1]{\left| #1 \right|}                         %Use: \Mag {Info} 
    
    \newcommand{\bVec}[1]{\mathbf{#1}}                              %Use for bold type of vector
    \newcommand{\lVec}[1]{\overrightarrow{#1}}                      %Use for a long arrow over a vector
    \newcommand{\uVec}[1]{\mathbf{\hat{#1}}}                        %Use: Unitary Vector Example: $\uVec{i}

    % ===== ALL FOR DOT PRODUCT  =========
    \makeatletter                                                   %WTF! IS THIS
    \newcommand*\dotP{\mathpalette\dotP@{.5}}                       %Use: \dotP for dot product
    \newcommand*\dotP@[2] {\mathbin {                               %WTF! IS THIS            
        \vcenter{\hbox{\scalebox{#2}{$\m@th#1\bullet$}}}}           %WTF! IS THIS
    }                                                               %WTF! IS THIS
    \makeatother                                                    %WTF! IS THIS

    % === WRAPPERS FOR COLUMN VECTOR ===
    \newcommand{\pVector}[1]                                        %Use: \pVector {Matrix Notation} use parentesis
        { \ensuremath{\begin{pmatrix}#1\end{pmatrix}} }             %Example: \pVector{a\\b\\c} or \pVector{a&b&c} 
    \newcommand{\lVector}[1]                                        %Use: \lVector {Matrix Notation} use a abs 
        { \ensuremath{\begin{vmatrix}#1\end{vmatrix}} }             %Example: \lVector{a\\b\\c} or \lVector{a&b&c} 
    \newcommand{\bVector}[1]                                        %Use: \bVector {Matrix Notation} use a brackets 
        { \ensuremath{\begin{bmatrix}#1\end{bmatrix}} }             %Example: \bVector{a\\b\\c} or \bVector{a&b&c} 
    \newcommand{\Vector}[1]                                         %Use: \Vector {Matrix Notation} no parentesis
        { \ensuremath{\begin{matrix}#1\end{matrix}} }               %Example: \Vector{a\\b\\c} or \Vector{a&b&c}

    % === MAKE MATRIX BETTER  =========
    \makeatletter                                                   %Example: \begin{matrix}[cc|c]
    \renewcommand*\env@matrix[1][*\c@MaxMatrixCols c] {             %WTF! IS THIS
        \hskip -\arraycolsep                                        %WTF! IS THIS
        \let\@ifnextchar\new@ifnextchar                             %WTF! IS THIS
        \array{#1}                                                  %WTF! IS THIS
    }                                                               %WTF! IS THIS
    \makeatother                                                    %WTF! IS THIS

    % =========================================
    % =======   FAMOUS FUNCTIONS   ============
    % =========================================

    % == TRIGONOMETRIC FUNCTIONS  ====
    \newcommand{\Cos}[1] {\cos\Wrap{#1}}                            %Simple wrappers
    \newcommand{\Sin}[1] {\sin\Wrap{#1}}                            %Simple wrappers
    \newcommand{\Tan}[1] {tan\Wrap{#1}}                             %Simple wrappers
    
    \newcommand{\Sec}[1] {sec\Wrap{#1}}                             %Simple wrappers
    \newcommand{\Csc}[1] {csc\Wrap{#1}}                             %Simple wrappers
    \newcommand{\Cot}[1] {cot\Wrap{#1}}                             %Simple wrappers

    % === COMPLEX ANALYSIS TRIG ======
    \newcommand \Cis[1]  {\Cos{#1} + i \Sin{#1}}                    %Use: \Cis for cos(x) + i sin(x)
    \newcommand \pCis[1] {\Wrap{\Cis{#1}}}                          %Use: \pCis for the same with parantesis
    \newcommand \bCis[1] {\Brackets{\Cis{#1}}}                      %Use: \bCis for the same with Brackets


    % =========================================
    % ===========     CALCULUS     ============
    % =========================================

    % ====== TRANSFORMS =============
    \newcommand{\FourierT}[1]{\mathscr{F} \left\{ #1 \right\} }     %Use: \FourierT {Funtion}
    \newcommand{\InvFourierT}[1]{\mathscr{F}^{-1}\left\{#1\right\}} %Use: \InvFourierT {Funtion}

    % ====== DERIVATIVES ============
    \newcommand \MiniDerivate[1][x] {\dfrac{d}{d #1}}               %Use: \MiniDerivate[var] for simple use [var]
    \newcommand \Derivate[2] {\dfrac{d \; #1}{d #2}}                %Use: \Derivate [f(x)][x]
    \newcommand \MiniUpperDerivate[2] {\dfrac{d^{#2}}{d#1^{#2}}}    %Mini Derivate High Orden Derivate -- [x][pow]
    \newcommand \UpperDerivate[3] {\dfrac{d^{#3} \; #1}{d#2^{#3}}}  %Complete High Orden Derivate -- [f(x)][x][pow]
    
    \newcommand \MiniPartial[1][x] {\dfrac{\partial}{\partial #1}}  %Use: \MiniDerivate for simple use [var]
    \newcommand \Partial[2] {\dfrac{\partial \; #1}{\partial #2}}   %Complete Partial Derivate -- [f(x)][x]
    \newcommand \MiniUpperPartial[2]                                %Mini Derivate High Orden Derivate -- [x][pow] 
        {\dfrac{\partial^{#2}}{\partial #1^{#2}}}                   %Mini Derivate High Orden Derivate
    \newcommand \UpperPartial[3]                                    %Complete High Orden Derivate -- [f(x)][x][pow]
        {\dfrac{\partial^{#3} \; #1}{\partial#2^{#3}}}              %Use: \UpperDerivate for simple use

    \DeclareMathOperator \Evaluate  {\Big|}                         %Use: \Evaluate por Notation

    % =========================================
    % ========    GENERAL STYLE     ===========
    % =========================================
    
    % =====  COLORS ==================
    \definecolor{RedMD}{HTML}{F44336}                               %Use: Color :D        
    \definecolor{Red100MD}{HTML}{FFCDD2}                            %Use: Color :D        
    \definecolor{Red200MD}{HTML}{EF9A9A}                            %Use: Color :D        
    \definecolor{Red300MD}{HTML}{E57373}                            %Use: Color :D        
    \definecolor{Red700MD}{HTML}{D32F2F}                            %Use: Color :D 

    \definecolor{PurpleMD}{HTML}{9C27B0}                            %Use: Color :D        
    \definecolor{Purple100MD}{HTML}{E1BEE7}                         %Use: Color :D        
    \definecolor{Purple200MD}{HTML}{EF9A9A}                         %Use: Color :D        
    \definecolor{Purple300MD}{HTML}{BA68C8}                         %Use: Color :D        
    \definecolor{Purple700MD}{HTML}{7B1FA2}                         %Use: Color :D 

    \definecolor{IndigoMD}{HTML}{3F51B5}                            %Use: Color :D        
    \definecolor{Indigo100MD}{HTML}{C5CAE9}                         %Use: Color :D        
    \definecolor{Indigo200MD}{HTML}{9FA8DA}                         %Use: Color :D        
    \definecolor{Indigo300MD}{HTML}{7986CB}                         %Use: Color :D        
    \definecolor{Indigo700MD}{HTML}{303F9F}                         %Use: Color :D 

    \definecolor{BlueMD}{HTML}{2196F3}                              %Use: Color :D        
    \definecolor{Blue100MD}{HTML}{BBDEFB}                           %Use: Color :D        
    \definecolor{Blue200MD}{HTML}{90CAF9}                           %Use: Color :D        
    \definecolor{Blue300MD}{HTML}{64B5F6}                           %Use: Color :D        
    \definecolor{Blue700MD}{HTML}{1976D2}                           %Use: Color :D        
    \definecolor{Blue900MD}{HTML}{0D47A1}                           %Use: Color :D  

    \definecolor{CyanMD}{HTML}{00BCD4}                              %Use: Color :D        
    \definecolor{Cyan100MD}{HTML}{B2EBF2}                           %Use: Color :D        
    \definecolor{Cyan200MD}{HTML}{80DEEA}                           %Use: Color :D        
    \definecolor{Cyan300MD}{HTML}{4DD0E1}                           %Use: Color :D        
    \definecolor{Cyan700MD}{HTML}{0097A7}                           %Use: Color :D        
    \definecolor{Cyan900MD}{HTML}{006064}                           %Use: Color :D 

    \definecolor{TealMD}{HTML}{009688}                              %Use: Color :D        
    \definecolor{Teal100MD}{HTML}{B2DFDB}                           %Use: Color :D        
    \definecolor{Teal200MD}{HTML}{80CBC4}                           %Use: Color :D        
    \definecolor{Teal300MD}{HTML}{4DB6AC}                           %Use: Color :D        
    \definecolor{Teal700MD}{HTML}{00796B}                           %Use: Color :D        
    \definecolor{Teal900MD}{HTML}{004D40}                           %Use: Color :D 

    \definecolor{GreenMD}{HTML}{4CAF50}                             %Use: Color :D        
    \definecolor{Green100MD}{HTML}{C8E6C9}                          %Use: Color :D        
    \definecolor{Green200MD}{HTML}{A5D6A7}                          %Use: Color :D        
    \definecolor{Green300MD}{HTML}{81C784}                          %Use: Color :D        
    \definecolor{Green700MD}{HTML}{388E3C}                          %Use: Color :D        
    \definecolor{Green900MD}{HTML}{1B5E20}                          %Use: Color :D

    \definecolor{AmberMD}{HTML}{FFC107}                             %Use: Color :D        
    \definecolor{Amber100MD}{HTML}{FFECB3}                          %Use: Color :D        
    \definecolor{Amber200MD}{HTML}{FFE082}                          %Use: Color :D        
    \definecolor{Amber300MD}{HTML}{FFD54F}                          %Use: Color :D        
    \definecolor{Amber700MD}{HTML}{FFA000}                          %Use: Color :D        
    \definecolor{Amber900MD}{HTML}{FF6F00}                          %Use: Color :D

    \definecolor{BlueGreyMD}{HTML}{607D8B}                          %Use: Color :D        
    \definecolor{BlueGrey100MD}{HTML}{CFD8DC}                       %Use: Color :D        
    \definecolor{BlueGrey200MD}{HTML}{B0BEC5}                       %Use: Color :D        
    \definecolor{BlueGrey300MD}{HTML}{90A4AE}                       %Use: Color :D        
    \definecolor{BlueGrey700MD}{HTML}{455A64}                       %Use: Color :D        
    \definecolor{BlueGrey900MD}{HTML}{263238}                       %Use: Color :D        

    \definecolor{DeepPurpleMD}{HTML}{673AB7}                        %Use: Color :D

    \newcommand{\Color}[2]{\textcolor{#1}{#2}}                      %Simple color environment
    \newenvironment{ColorText}[1]                                   %Use: \begin{ColorText}
        { \leavevmode\color{#1}\ignorespaces }                      %That's is!

    % =====  CODE EDITOR =============
    \lstdefinestyle{CompilandoStyle} {                              %This is Code Style
        backgroundcolor     = \color{BlueGrey900MD},                %Background Color  
        basicstyle          = \tiny\color{white},                   %Style of text
        commentstyle        = \color{BlueGrey200MD},                %Comment style
        stringstyle         = \color{Green300MD},                   %String style
        keywordstyle        = \color{Blue300MD},                    %keywords style
        numberstyle         = \tiny\color{TealMD},                  %Size of a number
        frame               = shadowbox,                            %Adds a frame around the code
        breakatwhitespace   = true,                                 %Style   
        breaklines          = true,                                 %Style   
        showstringspaces    = false,                                %Hate those spaces                  
        breaklines          = true,                                 %Style                   
        keepspaces          = true,                                 %Style                   
        numbers             = left,                                 %Style                   
        numbersep           = 10pt,                                 %Style 
        xleftmargin         = \parindent,                           %Style 
        tabsize             = 4,                                    %Style
        inputencoding       = utf8/latin1                           %Allow me to use special chars
    }
 
    \lstset{style = CompilandoStyle}                                %Use this style









% =====================================================
% ============        COVER PAGE       ================
% =====================================================
\begin{document}
\begin{titlepage}
    
    % ============ TITLE PAGE STYLE  ================
    \definecolor{TitlePageColor}{cmyk}{1,.60,0,.40}                 %Simple colors
    \definecolor{ColorSubtext}{cmyk}{1,.50,0,.10}                   %Simple colors
    \newgeometry{left=0.25\textwidth}                               %Defines an Offset
    \pagecolor{TitlePageColor}                                      %Make it this Color to page
    \color{white}                                                   %General things should be white

    % ===== MAKE SOME SPACE =========
    \vspace                                                         %Give some space
    \baselineskip                                                   %But we need this to up command

    % ============ NAME OF THE PROJECT  ============
    \makebox[0pt][l]{\rule{1.3\textwidth}{3pt}}                     %Make a cool line
    
    \href{https://compilandoconocimiento.com}                       %Link to project
    {\textbf{\textsc{\Huge Facultad de Ciencias - UNAM}}}\\[2.7cm]  %Name of project   

    % ============ NAME OF THE BOOK  ===============
    \href{\ProjectNameLink/LibroAlgebraLineal}                      %Link to Author
    {\fontsize{65}{78}\selectfont \textbf{2 Tarea-Examen}\\[0.5cm]  %Name of the book
    \textcolor{ColorSubtext}{\textsc{\Huge Algebra Lineal 1 }}}     %Name of the general theme
    
    \vfill                                                          %Fill the space
    
    % ============ NAME OF THE AUTHOR  =============
    \href{\ProjectAuthorLink}                                       %Link to Author
    {\LARGE \textsf{Oscar Andrés Rosas Hernandez}}                  %Author

    % ===== MAKE SOME SPACE =========
    \vspace                                                         %Give some space
    \baselineskip                                                   %But we need this to up command
    
    {\large \textsf{Abril 2018}}                                    %Date

\end{titlepage}


% =====================================================
% ==========      RESTORE TO DOCUMENT      ============
% =====================================================
\restoregeometry                                                    %Restores the geometry
\nopagecolor                                                        %Use to restore the color to white




% =====================================================
% ========                INDICE              =========
% =====================================================
\tableofcontents{}
\label{sec:Index}

\clearpage




% ==============================================================
% =================          PROBLEMA 1       ==================
% ==============================================================
\clearpage
\section{1 Problema}



    % ==============================================================
    % =============  TEOREMAS QUE OCUPAR P1       ==================
    % ==============================================================
        \begin{itemize}
            \item Sea $B$ una base de $\VectorSet$ entonces $R[\LinealTransformation] = <\LinTrans[B]>$ 

                % ======== DEMOSTRACION ========
                \begin{SmallIndentation}[1em]
                    \textbf{Demostración}:
                    
                    A fin de cuentas es la igualdad entre 2 conjuntos, así que vamos por doble contención
                    para hacerlo. Sea $B = \Set{\vec v_1, \dots, \vec v_n}$, entonces:

                    \begin{itemize}
                        \item 
                            Por un lado, sea $\vec u \in R[\LinealTransformation]$ entonces
                            tenemos que existe un $\vec x \in \VectorSet$ que al $\FnLinTrans{\vec x} = \vec u$
                            donde tenemos que $\vec x = \sum_{i=1}^n a_i \vec v_i$, entonces:
                            \begin{align*}
                                \FnLinTrans{\vec x} 
                                    &= \FnLinTrans{\sum_{i=1}^n a_i \vec v_i} 
                                    &= \sum_{i=1}^n \FnLinTrans{a_i \vec v_i} 
                                    &= \sum_{i=1}^n a_i \FnLinTrans{\vec v_i} 
                            \end{align*}

                                Y nota que $\sum_{i=1}^n a_i \FnLinTrans{\vec v_i} \in <\LinTrans[B]>$ 

                       \item
                        La otra contención es .... es basicamente lo mismo

                    \end{itemize}

                \end{SmallIndentation}

            \item \textbf{Teorema de la Dimensión}

                Sea $\VectorSet$ y $\SubVectorSet$ espacios vectoriales sobre el mismo campo, sea 
                $\LinTrans: \VectorSet \to \SubVectorSet$
                una transformación lineal y las dimensiónes de ambos espacios finitos, entonces
                tenemos que:
                $dim(\VectorSet) = dim(K[\LinTrans]) + dim(R[\LinTrans])$

                % ======== DEMOSTRACION ========
                \begin{SmallIndentation}[1em]
                    \textbf{Demostración}:
                    
                    Fijemos la dimensión de $\VectorSet$ a ser $n$, un natural.
                    Ahora, por el mero hecho de que $K[\LinTrans]$ es un subespacio de $\VectorSet$
                    tenemos que $dim(K[\LinTrans]) \leq dim(\VectorSet)$.

                    Ahora, sea $\Set{\vec v_1, \dots, \vec v_k}$ una base de $K[\LinTrans]$, ahora, como es un conjunto
                    linealmente independiente de $\VectorSet$ podemos extenderlo hasta que sea base del mismo
                    $\VectorSet$.

                    Es decir, sea $B = \Set{\vec v_1, \dots, \vec v_k, \vec v_{k+1}, \dots, \vec v_n}$.

                    Ahora veamos que pasa al aplicarle la transformación lineal a ese conjunto, es
                    decir $\LinTrans$.
                    Ahora, ya habiamos demostrado el generado de la transformación lineal de una base
                    es $R[\LinTrans]$.
                    Ahora, yo te digo, que $S = \Set{ \LinTrans(\vec v_{k+1}), \dots, \LinTrans(\vec v_n)}$
                    es base de $R[\LinTrans]$.

                    Y te lo voy a demostrar:
                    \begin{itemize}
                        \item 
                            Por un lado $S$ genera a $R[\LinTrans]$ porque sabemos que $<\LinTrans[B]>$.

                            Pero, 
                            $\Generate{\LinTrans[B]} 
                                = \Generate{\vec 0, \LinTrans(\vec v_{k+1}), \dots, \LinTrans(\vec v_n)}$

                            Pero espera, todos los primeros $k$ elementos de $B$ por definición son mapeados
                            al cero, pero $R[\LinTrans]$ es ya un espacio por lo cual ya tienen al cero, y no 
                            aporta nada.

                        \item
                            S es linealmente independiente:

                            % ======== DEMOSTRACION ========
                            \begin{SmallIndentation}[1em]
                                \textbf{Demostración}:
                                \begin{align*}
                                    \sum_{k+1}^n b_i \LinTrans(\vec v_i) = \vec 0
                                    \FnLinTrans{\sum_{k+1}^n b_i \vec v_i} = \vec 0
                                \end{align*}

                                Pero $B$ es un base, por lo tanto es linealmente independiente, por
                                lo tanto tenemos que $\sum_{k+1}^n b_i \vec v_i = \vec 0$ implica que
                                todas las $b_i = 0$.

                                Además recuerda que $\Set{\vec v_1, \dots, \vec v_k}$ es base del Kernel
                                es decir a todos los elementos que $\LinTrans(\vec x) = \vec 0$, por 
                                lo tanto (y ya que $B$ es base, es decir tiene que ser linealmente independiente)
                                por obliga a que todas las $b_i$ sean ceros, es decir, si que era linealmente
                                independiente
                            
                            \end{SmallIndentation}


                            Ahora, ya vimos que $dim(\VectorSet) = n$, $dim(K[\LinTrans]) = k$
                            y $dim(R[\LinTrans]) = n - k$
                                
                    \end{itemize}
                
                \end{SmallIndentation}
        \end{itemize}

    % ==============================================================
    % =============  TEOREMAS QUE OCUPAR P1       ==================
    % ==============================================================
    \clearpage
    \subsection{Problema en si}
        
        \begin{itemize}

            \item
                Encontrar una base para el rango y el kernel de:
                $T: \Polynomials_2(\Reals) \to \Polynomials_3(\Reals)$
                dada por $T(f(x)) := xf(x) + f'(x)$

                % ======== DEMOSTRACION ========
                \begin{SmallIndentation}[1em]
                    \textbf{Demostración}:
                    
                    Primero, antes que nada vamos a demostrar que $T$ es una transformación lineal
                    para eso tomemos arbitrariamente $f(x), g(x) \in \Polynomials_2(\Reals)$
                    y $c \in \Reals$, entonces tenemos que:
                    \begin{align*}
                        T(cf(x) + g(x))
                            &= x(cf(x) + g(x)) + (cf(x) + g(x))'         \\
                            &= xcf(x) + xg(x) + (cf(x))' + g'(x)         \\
                            &= xcf(x) + xg(x) + cf'(x) + g'(x)           \\
                            &= xcf(x) + xg(x) + cf'(x) + g'(x)           \\
                            &= xcf(x) + cf'(x) + xg(x) + g'(x)           \\
                            &= c(xf(x) + f'(x)) + xg(x) + g'(x)          \\
                            &= c(T(f(x))) + T(g(x))
                    \end{align*}

                    Ok, ahora veamos que la pasa a una base al transformarla:
                    \begin{align*}
                        T\Brackets{\Wrap{1, x, x^2}}
                            &= \Set{T(1), T(x), T(x^2)}             \\ 
                            &= \Set{(x), (x^2 + 1), (x^3 + 2x)}     \\ 
                    \end{align*}

                    Creo que es más que obvio que son independientes linealmente (sobretodo por el grado
                    del polinomio)
                    y más aún hemos demostrado que el generado del conjunto de las transformados
                    de una base de $\VectorSet$ nos da el Rango de la transformación, por lo tanto
                    cumple todas las características de una base.

                    Ahora, por el otro lado, y por el teorema de la dimensión tenemos que el Kernel
                    solo contiene al polinomio cero por lo tanto tenemos que:

                    \begin{itemize}
                        \item Una base para $R[T]$ es $\Set{x, x^2+1, x^3+2x}$ otra
                        por ejemplo puede ser $\Set{x, x^2 + 1, x^3}$
                        \item Una base para $K[T]$ es $\emptyset$ es decir el Kernel es
                            $\Set{0}$
                    \end{itemize}

                
                \end{SmallIndentation}

            \item
                Encontrar una base para el rango y el kernel de:
                $T: M_{n \times n}(\Reals) \to \Reals$
                dada por $T(A) := tr(A)$

                % ======== DEMOSTRACION ========
                \begin{SmallIndentation}[1em]
                    \textbf{Demostración}:
                    
                    Primero, antes que nada vamos a demostrar que $T$ es una transformación lineal
                    para eso tomemos arbitrariamente $A, B$
                    y $c \in \Reals$, entonces tenemos que:
                    \begin{align*}
                        T(cA + B)
                            &= tr(cA + B)                           \\
                            &= \sum_{i = 0}^n(c[A]_{i,i} + B{i,i})  \\
                            &= \sum_{i = 0}^n(c[A]_{i,i}) 
                               +
                               \sum_{i = 0}^n([B]_{i,i})            \\
                            &= c\sum_{i = 0}^n([A]_{i,i}) 
                               +
                               \sum_{i = 0}^n([B]_{i,i})            \\
                            &= c tr(A) + tr(B)
                    \end{align*}

                    Ok entonces, ya sabemos que es una transformación lineal
                    ahora, claro que podemos llegar a cualquier elemento del campo,
                    es decir $T(E_{1, 1}) = 1$, por lo tanto $T(kE_{1, 1}) = k$
                    entonces la base del Rango es claramente {1}.

                    Ahora, el Kernel, el Kernel es otra historia, para empezar podemos
                    pensar en todas las matrices que tienen cero a lo largo de la diagonal
                    es decir $\Set{E_{i, j} \Such i \neq j}$.

                    Ahora hay que pensar en las que suman cero, su base claramente son:
                    $\Set{E_{i, i} + E_{n, n} \Such i \in [1, 2, \dots, n - 1]}$

                    Por lo tanto tenemos que:
                    \begin{itemize}
                        \item Una base para $R[T]$ es $\Set{1}$
                        \item Una base para $K[T]$ es $\Set{E_{i, j} \Such i \neq j} \cup 
                            \Set{E_{i, i} + E_{n, n} \Such i \in [1, 2, \dots, n - 1]}$
                    \end{itemize}
                
                \end{SmallIndentation}

        \end{itemize}



% ==============================================================
% =================          PROBLEMA 2       ==================
% ==============================================================
\clearpage
\section{2 Problema}

    Sea $\VectorSet, \SubVectorSet$ espacios vectoriales con subespacios 
    $\VectorSet_1, \SubVectorSet_1$, respectivamente. 

    Si $T : \VectorSet \to \SubVectorSet$ es lineal, entonces:
    \begin{align*}
        T[\VectorSet_1] \leq_{\GenericField} \SubVectorSet
        \Space \Also \Space     
        \Set{x \in \VectorSet \Such T(x) \in \SubVectorSet_1} \leq_{\GenericField} \VectorSet   
    \end{align*} 

    % ======== DEMOSTRACION ========
    \begin{SmallIndentation}[1em]
        \textbf{Demostración}:
        
        Primero vamos a ver que $T[\VectorSet_1] \leq_{\GenericField} \SubVectorSet$
        esto se hace en 2 pasos:
        \begin{itemize}
            \item 
                Nota que $\VectorSet_1$ es un subespacio entonces ya tiene al cero
                vector simplemente por ser un subespacio, ahora como $T$ es una transformación
                lineal, ya sabemos que $T(\vec 0) = \vec 0$, por lo tanto este también 
                esta en $T[\VectorSet_1]$, por lo tanto $\vec 0 \in T[\VectorSet_1]$

            \item
                Vamos tomemos $c \in \GenericField$ y $\vec y_1, \vec y_2 \in T[\VectorSet_1]$
                entonces tenemos $\vec x_1, \vec x_2 \in \VectorSet_1$ tal que 
                $T(\vec x_1) = \vec y_1$ y $T(\vec x_2) = \vec y_2$.

                Entonces tenemos que $T(x_1 + x_2) = y_1 + y_2$ y $T(cx_1) = cy_1$, por lo tanto
                $y_1 + y_2, cy_1 \in T[\VectorSet_1]$
        \end{itemize}

        Ahora vamos a probar que $\Set{x \in \VectorSet \Such T(x) \in \SubVectorSet_1} 
        \leq_{\GenericField} \VectorSet$.
        \begin{itemize}
            \item 
                Nota que $\SubVectorSet_1$ es un subespacio entonces ya tiene al cero
                vector simplemente por ser un subespacio, ahora como $T$ es una transformación
                lineal, ya sabemos que $T(\vec 0) = \vec 0$, por lo tanto este también 
                esta en $\Set{x \in \VectorSet \Such T(x) \in \SubVectorSet_1}$, por lo
                tanto $\vec 0 \in \Set{x \in \VectorSet \Such T(x) \in \SubVectorSet_1}$

            \item
                Vamos tomemos $c \in \GenericField$ y 
                $\vec y_1, \vec y_2 \in \Set{x \in \VectorSet \Such T(x) \in \SubVectorSet_1}$
                entonces tenemos $\vec x_1, \vec x_2 \in \SubVectorSet_1$ tal que 
                $T(\vec y_1) = \vec x_1$ y $T(\vec y_2) = \vec x_2$.

                Entonces tenemos que $T(y_1 + y_2) = x_1 + x_2$ y $T(cy_1) = cx_1$, por lo tanto
                $y_1 + y_2, cy_1 \in \Set{x \in \VectorSet \Such T(x) \in \SubVectorSet_1}$
        \end{itemize}

    \end{SmallIndentation}
                            
     


% ==============================================================
% =================          PROBLEMA 3       ==================
% ==============================================================
\clearpage
\section{3 Problema}


    Sean $\VectorSet, \SubVectorSet$ espacios vectoriales y sean 
    $T, U \in \Laplace(\VectorSet, \SubVectorSet)$ no nulas. 

    Si $R[T] \cap R[U] = \Set{\vec 0}$, entonces ${T, U}$ es un
    subconjunto linealmente independiende de $\Laplace(\VectorSet, \SubVectorSet)$. 

    % ======== DEMOSTRACION ========
    \begin{SmallIndentation}[1em]
        \textbf{Demostración}:
        
        Este deberia ser sencillo, primero supongamos que no son linealmente independientes
        es decir que podemos expresar a $T = kU$, entonces tomemos a un vector en el rango 
        de $T$ (que no nos de el cero vector, ni que sea el cero vector), podemos hacer esto
        porque el dijimos que ninguna de las transformaciones es nula.

        Ahora ve que $T(x) = kU(x) = U(kx)$, entonces si te das cuenta encontramos un vector
        en el rango de ambos que comparten, ahora, como $\vec x \neq 0$ y ademas especificamente
        seleccionamos a $\vec x$ para que su transformada no sea cero.

        Pero eso es imposible, dijimos que $R[T] \cap R[U] = \Set{\vec 0}$, por lo tanto
        contradicción.

        ${T, U}$ es un subconjunto linealmente independiende de $\Laplace(\VectorSet, \SubVectorSet)$

    \end{SmallIndentation}
                            


% ==============================================================
% =================          PROBLEMA 4       ==================
% ==============================================================
\vspace{1em}
\section{4 Problema}


    Sea $\VectorSet$ un espacio vectorial y sea $T \in \Laplace(\VectorSet)$. Entonces 
    $T^2 = T_0$(la transformación cero) si y solo si $R[T] \subseteq N[T]$.

    % ======== DEMOSTRACION ========
    \begin{SmallIndentation}[1em]
        \textbf{Demostración}:
        
        Ok, vamos paso por paso, por un lado:

        Supongamos que $T^2 = T_0$ entonces tomemos $\vec y \in R[T]$ entonces 
        tenemos que $\vec y = T(\vec x)$ para alguna $\vec x$.
        Y $T(\vec y) = T(T(\vec x)) = T^2(\vec x) = \vec 0$, por lo tanto $\vec y \in N[T]$

        Por otro lado tenemos que:
        Si $R[T] \subseteq N[T]$, tenemos que $T^2(\vec x) = T(T(\vec x)) = \vec 0$ 
        y ya que $T(\vec x)$ es un elemento de $R[T]$ y como vimos de $N(T)$.
    
    \end{SmallIndentation}




% ==============================================================
% =================          PROBLEMA 5       ==================
% ==============================================================
\clearpage
\section{5 Problema}


    % ==============================================================
    % =============  TEOREMAS QUE OCUPAR P5       ==================
    % ==============================================================
    \subsection{Teoremas que Ocupar}

        \begin{itemize}
            \item 
                Sea $B$ una base de $\VectorSet$ entonces $R[\LinealTransformation] = <\LinTrans[B]>$ 

                % ======== DEMOSTRACION ========
                \begin{SmallIndentation}[1em]
                    \textbf{Demostración}:
                    
                    A fin de cuentas es la igualdad entre 2 conjuntos, así que vamos por doble contención
                    para hacerlo. Sea $B = \Set{\vec v_1, \dots, \vec v_n}$, entonces:

                    \begin{itemize}
                        \item 
                            Por un lado, sea $\vec u \in R[\LinealTransformation]$ entonces
                            tenemos que existe un $\vec x \in \VectorSet$ que al $\FnLinTrans{\vec x} = \vec u$
                            donde tenemos que $\vec x = \sum_{i=1}^n a_i \vec v_i$, entonces:
                            \begin{align*}
                                \FnLinTrans{\vec x} 
                                    &= \FnLinTrans{\sum_{i=1}^n a_i \vec v_i} 
                                    &= \sum_{i=1}^n \FnLinTrans{a_i \vec v_i} 
                                    &= \sum_{i=1}^n a_i \FnLinTrans{\vec v_i} 
                            \end{align*}

                                Y nota que $\sum_{i=1}^n a_i \FnLinTrans{\vec v_i} \in <\LinTrans[B]>$ 

                       \item
                        La otra contención es .... es basicamente lo mismo

                    \end{itemize}

                \end{SmallIndentation}

            \item
                Hablando de espacios finitos decimos que
                $\VectorSet \cong_{\GenericField} \SubVectorSet$ si y solo si 
                $dim(\VectorSet) = dim(\SubVectorSet)$

                % ======== DEMOSTRACION ========
                \begin{SmallIndentation}[1em]
                    \textbf{Idea Demostración}:

                    Por un lado es sencillo, si suponemos que $\VectorSet \cong_{\GenericField} \SubVectorSet$
                    entonces se que existe una función invertible entre los dos espacios, dicha
                    función si es invertible entonces es biyectiva, entonces tenemos que
                    es una función inyectiva y una función suprayectiva, entonces
                    $dim(\VectorSet) = dim(\SubVectorSet)$

                    Por el otro lado es casi lo mismo
                
                \end{SmallIndentation}



        \end{itemize}

    % ==============================================================
    % =============           PROBLEMA 5          ==================
    % ==============================================================
    \vspace{1em}
    \subsection{Problema en si}

        Sean $\VectorSet, \SubVectorSet$ espacios vectoriales y sea 
        $T: \VectorSet \cong_{\GenericField} \SubVectorSet$. Si $\beta$ es una base
        para $\SubVectorSet$, entonces que $T[\beta]$ es una base para $\SubVectorSet$.

        % ======== DEMOSTRACION ========
        \begin{SmallIndentation}[1em]
            \textbf{Demostración}:
            
            Ahora, sabemos de otra demostración que $<T[\beta]> = R[T]$, por lo tanto lo unico que nos
            falta por ver es que son linealmente independiente, pues de serlo y por ser base de $\VectorSet$
            (y por otro teorema pasado) tienen la cantidad de vectores necesarios para ser base
            de $\SubVectorSet$.

            Ahora, como $\beta$ es una base entonces $\sum_{i=1}^n a_i \beta_i = \vec 0$ implica
            que $a_i = 0$ para $i \in [1, n]$.

            Ahora, nota que $\sum_{i=1}^n a_i T(\beta_i) = T(\sum_{i=1}^n a_i \beta_i) = T(\vec 0) = \vec 0$.
            Por lo tanto, también que la combinación lineal de el conjunto de las transformadas de la base
            sea cero, implica que todos los escalares son cero, por lo tanto, tenemos que son linealmente
            independientes y son también $n$ vectores, por lo tanto son base  
        
        \end{SmallIndentation}




% ==============================================================
% =================          PROBLEMA 6       ==================
% ==============================================================
\clearpage
\section{6 Problema}

    Sea $B \in M_{n \times n}(\GenericField)$ invertible. 
    Defina $\eta : M_{n \times n}(\GenericField) to M_{n \times n}(\GenericField)$
    dada por $\eta(A) = B^{-1}AB$. Demuestre que $\eta$ es un isomorfismo.

    % ======== DEMOSTRACION ========
    \begin{SmallIndentation}[1em]
        \textbf{Demostración}:
        
        Primero que nada hay que demostrar que $\eta$ es lineal:
        \begin{align*}
            \eta(cA + D)
                &= B^{-1}(cA + D)B                  \\
                &= B^{-1}(cAB + DB)                 \\
                &= (B^{-1}cAB) + (B^{-1}DB)         \\
                &= c(B^{-1}AB) + (B^{-1}DB)         \\
                &= c \eta(A) + \eta(D)              
        \end{align*}

        Ahora para probar que es inyectiva lo unico que vamos a ver
        que es inyectiva demostrando que su kernel solo contiene al cero.

        Ya que si $\eta(A) = 0$ entonces quiere decir que $\eta(A) = B^{-1}0B$
        porque después de todo $B$ es invertible, por lo tanto ni ella ni su inversa
        puede ser el cero vector, por lo tanto no nos queda mas que $A = 0$.

        Por otro lado es subrayectiva, es decir, puedo llegar a cualquier matriz
        del espacio con esta función. Para una matriz arbitraria $A$ tenemos que
        $\eta(D) = B^{-1}DB = D$.

        Por lo tanto $\eta$ es inyectiva y subrayectiva, por lo tanto es invertible
        por $\eta$ es un isomorfismo por definición.

    \end{SmallIndentation}




% ==============================================================
% =================          PROBLEMA 7       ==================
% ==============================================================
\clearpage
\section{7 Problema}

    Sea $\beta = \Set{1, x, x,^2}$ y sea $\gamma = \Set{1+x+x^2, 1-x, 1}$.

    Hallemos la matriz de cambio de coordenadas de $\gamma$ a $\beta$.

    Primero veamos cada uno de los elementos de $\gamma$ como combinación lineal de $\beta$:
    \begin{itemize}
        \item $(1+x+x^2) = 1(1) + 1(x) + 1(x^2)$
        \item $(1-x) = 1(1) + (-1)(x) + 0(x^2)$
        \item $(1) = 1(1) + 0(x) + 0(x^2)$
    \end{itemize}

    Por lo tanto tenemos que nuestra matriz de cambio es:
    $Q = \bVector{1 & 1 & 1 \\ 1 & -1 & 0 \\ 1 & 0 & 0}$

    Por ejemplo el vector $x^2$ se ve en gamma como $\bVector{1 \\ 1 \\ -2}$ entonces tenemos que:
    \begin{align*}
        Q\bVector{1, 1, -2}
            &= \bVector{1 & 1 & 1 \\ 1 & -1 & 0 \\ 1 & 0 & 0} \bVector{1 \\ 1 \\ -2}        \\
            &= \bVector{0 \\ 0 \\ 1}
    \end{align*}


% ==============================================================
% =================          PROBLEMA 8       ==================
% ==============================================================
\clearpage
\section{8 Problema}

    \begin{itemize}

        \item
            Si $A, B \in M_{n \times n}(\GenericField)$ entonces $traza(AB) = traza(BA)$
            % ======== DEMOSTRACION ========
            \begin{SmallIndentation}[1em]
                \textbf{Demostración}:
                
                Veamos como sale esto:
                \begin{align*}
                    traza(AB)
                        &= \sum_{k = 1}^n [AB]_{k, k}                                \\
                        &= \sum_{k = 1}^n \sum_{k' = 1}^n [A]_{k, k'} [B]_{k', k}    \\
                        &= \sum_{k = 1}^n \sum_{k' = 1}^n [B]_{k', k} [A]_{k, k'}    \\
                        &= \sum_{k' = 1}^n \sum_{k = 1}^n [B]_{k', k} [A]_{k, k'}    \\
                        &= \sum_{k' = 1}^n [BA]_{k', k'}                             \\
                        &= traza(BA)
                \end{align*}
            
            \end{SmallIndentation}

        \item
            Si $A \in M_{n \times n}(\GenericField)$ entonces $traza(A) = traza(A^T)$
            % ======== DEMOSTRACION ========
            \begin{SmallIndentation}[1em]
                \textbf{Demostración}:
                
                Veamos como sale esto:
                \begin{align*}
                    traza(A)
                        &= \sum_{k = 1}^n [A]_{k, k}                                 \\
                        &= \sum_{k' = 1}^n [A^T]_{k, k}                              \\
                        &= traza(A^T)
                \end{align*}

                Así de sencillo

            \end{SmallIndentation}


        \item
            Si $A, B \in M_{n \times n}(\GenericField)$ y $A$ similiar a $B$ entonces $traza(A) = traza(B)$
            % ======== DEMOSTRACION ========
            \begin{SmallIndentation}[1em]
                \textbf{Demostración}:

                Ahora como tenemos que son similares tenemos que $B = P^{-1}AP$
                
                Veamos como sale esto:
                \begin{align*}
                    traza(B)
                        &= traza(P^{-1}AP)                                          
                            && \Remember{Ahora como sabemos que son similiares}     \\
                        &= traza(P^{-1}(AP))                                          
                            && \Remember{Ahora agrupamos}                           \\
                        &= traza((AP)P^{-1})                                          
                            && \Remember{$traza(AB) = traza(BA)$}                   \\
                        &= traza(A(PP^{-1}))                                          
                            && \Remember{Agrupamos ahora si}                        \\
                        &= traza(AId_n)                                          
                            && \Remember{Definición}                                \\
                        &= traza(A)                                        
                            && \Remember{Definición de indentidad}
                \end{align*}
            
            \end{SmallIndentation}

    \end{itemize}



% ==============================================================
% =================          PROBLEMA 9       ==================
% ==============================================================
\clearpage
\section{9 Problema}

    Sea $M_{n \times n}(\GenericField)$ entonces si $A^2 = 0$ entonces $A$ no es invertible

        % ======== DEMOSTRACION ========
        \begin{SmallIndentation}[1em]
            \textbf{Demostración}:

            Si $A$ es invertible entonces $I = AB$ para alguna $B$, quien sabe cual sea, pero existe.

            Entonces $A = AI = A(AB) = A^2B = 0B = 0$, y por lo tanto $A = 0$, pero $0$ no es invertible

            ¡Contradicción! Entonces solo nos queda por admitir que $A$ no es invertible
        
        \end{SmallIndentation}



% ==============================================================
% =================          PROBLEMA 10      ==================
% ==============================================================
\clearpage
\section{10 Problema}

    Sea $\VectorSpace$ un espacio vectorial con $\beta = \Set{\vec x_1, \dots , \vec x_n}$ 
    como base ordenada.

    Dado que $\vec x_0 = \vec 0$

    Entonces debe existir una transformación lineal $T \in \Laplace(\VectorSet)$ tal
    que $T(\vec x_j) = \vec x_j - \vec x_{j - 1}$ para toda $j \in \Set{1, \dots, n}$.

    Encontremos $[T]_\beta$

    % ======== DEMOSTRACION ========
    \begin{SmallIndentation}[1em]
        \textbf{Solución}:
        
        Vamos a tomar a cada uno de los $n$ vectores de la base $\beta$ y transformarlo entonces:
        \begin{itemize}
            \item Para el primer vector tenemos que $T(\vec x_1) = \vec x_1$
            \item Para el segundo vector tenemos que $T(\vec x_2) = \vec x_2 - \vec x_1$
            \item Para el tercer vector tenemos que $T(\vec x_3) = \vec x_3 - \vec x_2$
            \item \dots
        \end{itemize}

        Entonces tenemos los vectores columna serán:
        \begin{itemize}
            \item Primera Columna \bVector{1 \\ 0 \\ 0 \\ \dots \\ 0}
            \item Segunda Columna \bVector{-1 \\ 1 \\ 0 \\ \dots \\ 0}
            \item Tercera Columna \bVector{0 \\ -1 \\ 1 \\ \dots \\ 0}
        \end{itemize}

        Entonces en general:
        \begin{align*}
            [T]_\beta
                &= \bVector{
                    1 & -1 &  0 & \dots & 0 & 0   \\
                    0 &  1 & -1 & \dots & 0 & 0   \\
                    0 &  0 &  1 & \dots & 0 & 0   \\
                    0 &  0 &  0 & \dots & 0 & 0   \\
                    \dots                         \\
                    0 &  0 &  0 & \dots & 1 & -1  \\
                    0 &  0 &  0 & \dots & 0 & 1   \\
                }
        \end{align*}
    
    \end{SmallIndentation}

\end{document}